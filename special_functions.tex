\documentclass[10pt,dvipdfmx,letterpaper,twoside]{article}
\usepackage{marginnote}
\usepackage[left=1in,right=1in,top=1in,bottom=1in]{geometry}
\usepackage{amsmath}
\usepackage{amssymb}
\usepackage{framed}
\usepackage{color}
%\usepackage{cite}
%\usepackage{accents}

\newcommand{\F}[1]{{\mathtt{#1}}}

\DeclareMathOperator{\erf}{erf}
\DeclareMathOperator{\erfc}{erfc}
\DeclareMathOperator{\acos}{acos}
\DeclareMathOperator{\asin}{asin}
\DeclareMathOperator{\atan}{atan}
\DeclareMathOperator{\acosh}{acosh}
\DeclareMathOperator{\asinh}{asinh}
\DeclareMathOperator{\atanh}{atanh}
\let\O=\operatorname
\newcommand{\Ob}[1]{\operatorname{\mathbf{#1}}}

\newcommand{\RR}{{\mathbb{R}}}
\newcommand{\CC}{{\mathbb{C}}}
\newcommand{\ZZ}{{\mathbb{Z}}}
\newcommand{\NN}{{\mathbb{N}}}
\newcommand{\NNo}{{\mathbb{N}_0}}
\newcommand{\ii}{{\hat{\imath}}}
\newcommand{\Hyper}[5]{{\mathop{{}_{#1}\!\O{F\!}_{#2}\left({\genfrac{}{}{0pt}{}{#3}{#4}}\middle|{#5}\right)}}}

%\newenvironment{implementation}{\noindent\begin{framebox}\begin{minipage}{0.98\textwidth}}{\end{minipage}\end{framebox}}
\newenvironment{implementation}{\noindent\begin{framed}}{\end{framed}}
\let\DEF=\marginnote

\def\Xint#1{\mathchoice
{\XXint\displaystyle\textstyle{#1}} 
{\XXint\textstyle\scriptstyle{#1}} 
{\XXint\scriptstyle\scriptscriptstyle{#1}} 
{\XXint\scriptscriptstyle\scriptscriptstyle{#1}} 
\!\int}
\def\XXint#1#2#3{{\setbox0=\hbox{$#1{#2#3}{\int}$ }
\vcenter{\hbox{$#2#3$ }}\kern-.5\wd0}}
\def\ddashint{\Xint=}
\def\dashint{\Xint-}

\let\Beta=B
\let\al=\alpha
\let\gam=\gamma
\let\Gam=\Gamma
\let\eps=\varepsilon
\let\kap=\kappa
\let\lam=\lambda
\let\sig=\sigma
\let\theta=\vartheta
\let\phi=\varphi
\let\om=\omega

\DeclareMathAccent{\wtilde}{\mathord}{largesymbols}{"65}

% to use titled-frame, must define TFFrameColor and TFTitleColor
\definecolor{TFFrameColor}{rgb}{0.9,0.9,0.9}
\definecolor{TFTitleColor}{rgb}{0.0,0.0,0.0}

\begin{document}

{\LARGE Special Functions}\hfill Apollo Hogan\hfill\today

\tableofcontents

%%%%%%%%%%%%%%%%%%%%%%%%%%%%%%%%%%%%%%%%%%%%%%%%%%%%%%%%%%%%%%%%%%%%%%%%%%%%%%%%
\section{Constants, Basic definitions}

INDEX
GLOSSARY
SYMBOLS
(MARGIN NOTES)
EQUATION NUMBERING?

[Generally need {\em zeros} as well as {\em derivatives} of all functions... and possibly {\em integrals}!]

\begin{itemize}
\item Mathematics
  \begin{itemize}
  \item definitions
  \item relations / expansions / extensions / etc.
  \item discussion of contxt, usage, other notations, etc.
  \item graphics
  \end{itemize}
\item Computer science
  \begin{itemize}
  \item routines (API)
  \item implementation notes
  \item pseudo-code / code
  \item example usage
  \item error plots, etc.
  \end{itemize}
\end{itemize}
Coding style: CamelCase as basic function names, but underscore to separate modifications?
For example, Exp, Exp\_m1, StruveH, StruveH\_ddx; double underscore for internal variations
(e.g. different methods, etc.), e.g. Exp\_m1\_\_naive, Exp\_m1\_\_series.

%\[3 + \underaccent{\tilde}{\Sigma} - 100 \qquad 3 + {\underaccent{\tilde}{\Sigma}}^0_n - 100\]
%Test out this $\underaccent{\tilde}{\Sigma}$ and stuff ${\underaccent{\tilde}{\Sigma}}^0_n$ fooferah.
%\[ \O{\mathbf{H}}_\nu(x) = \underaccent{\tilde}{\O{H}}_\nu(x) = \mathbf{H}_\nu(x) \]
%\[ \O{\mathbf{H}}_\nu(x) = \underaccent{\wtilde}{\O{H}}_\nu(x) = \mathbf{H}_\nu(x) \]
%\[ \O{\mathbf{H}}_\nu(x) = \undertilde{\O{H}}_\nu(x) = \mathbf{H}_\nu(x) \]

\begin{itemize}
\item $\NN = \{ 1, 2, \dots \}$
\item $\NNo = \{ 0, 1, 2, \dots \} = \NN\cup\{0\}$
\item $\ZZ = \{ \dots, -2, -1, 0, 1, 2, \dots \}$
\end{itemize}

\begin{itemize}
\item Pi%
  \[ \pi = 3.1415926535897932384626433\dots  \DEF{$\pi$}\]
  \[ \frac1\pi = 12\sum_{k=0}^\infty(-)^k\frac{(6k)!(13591409+545140134k)}{(3k)!(k!)^3640320^{3k+3/2}} \qquad\text{Chudnovsky algorithm}\]
\item Euler's gamma%
  \[ \gam = \lim_{n\to\infty}\left(\sum_{k=1}^n\frac1k - \log n)\right) = -\int_0^\infty e^{-t}\ln t\,dt = 0.57721566490153286060\dots
      \DEF{$\gam$}\]
\item Khintchine constant
  \[ K_0 = \prod_{k=1}^\infty\left[1 + \frac{1}{k(k+2)}\right]^{\log_2 k} = 2.6854520010\dots
      \DEF{$K_0$}\]
\item misc
  \[ \log(2) = \dots \]
  \[ \sqrt{2} = \dots \]
  \[ \zeta(2) = \frac{\pi^2}{6} = \dots \]
  \[ \Gam(1/3) = 2.6789385347077\dots \]
\end{itemize}

Basic operations expected of any numeric type:
$+, -, *, /, +=, -=, *=, /=, ++, --, ==, !=, <, >, <=, >=,
 \O{pow}, \O{pow_int}, \O{sqrt}, \O{cbrt}, \O{ln}, \O{exp},
 \O{abs}, \O{arg}, \O{signum}, \O{floor}, \O{ceil}, \O{truncate}, \O{mod}$

Use double-doubles just for more accurate summations (and range-reductions?)

From~\cite{precise-numerical}:
given a number $x$ and an uncertainty in $x$ of $\delta_x$,
we define the {\em accuracy} as $\O{Accuracy}(x,\delta_x) = (-\log_{10}|\delta_x|$ (corresponding to the absolute error),
the {\em precision} as $\O{Precision}(x,\delta_x) = -\log_{10}|\frac{\delta_x}{x}|$ (corresponding to the relative error),
and the {\em scale} as $\O{Scale} = \O{Precision} - \O{Accuracy}$.  We can visualize different cases as:

If {\em accuracy}, {\em precision}, and {\em scale} are all positive, we have:
\[ \overbrace{\underbrace{x_{1}x_{2}\cdots x_{s}\vphantom{x_{s+1}}}_{\O{Scale}}.\underbrace{x_{s+1}x_{s+2}\cdots x_{s+m}}_{\O{Accuracy}}}^{\O{Precision}} \]
If {\em accuracy} and {\em precision} are positive and {\em scale} negative, we have:
\[ \overbrace{\underbrace{.00\cdots0\vphantom{x_p}}_{\O{Scale}}\underbrace{x_{1}x_{2}\cdots x_{p}}_{\O{Precision}}}^{\O{Accuracy}} \]
If {\em precision} and {\em scale} are positive and {\em accuracy} negative, we have:
\[ \overbrace{\underbrace{x_{1}x_{2}\cdots x_{p}}_{\O{Precision}}\underbrace{\vphantom{x_p}00\cdots0.}_{\O{Accuracy}}}^{\O{Scale}} \]

error diagrams; diagrams for techniques (colored nicely) used in 2-parameters, e.g.

test of continued fraction typesetting:
\[ \cfrac{1}{x + \cfrac{2}{x^2 + \cfrac{3}{x^3 + \cdots }}} \]

\[(1-\cos x)/\sin x = \sin x / (1+\cos x) \]
\[ e^{a+b} = e^{a}e^{b} = e^{a}(1+(e^{b}-1)) = e^{a} + e^{a}(e^{b}-1) \]

%%%%%%%%%%%%%%%%%%%%%%%%%%%%%%%%%%%%%%%%%%%%%%%%%%%%%%%%%%%%%%%%%%%%%%%%%%%%%%%%
\section{Existing systems}

[Check Matlab special function support!]

%%%%%%%%%%%%%%%%%%%%%%%%%%%%%%%%%%%%%%%%
\subsection{Octave}

IDEA: build out the special function support for Octave!?  Issue is lack of
long double type to help with getting full precision.
(Can create objects + overloading to maybe create double-double types?)

Note that Octave includes a gsl package for GSL bindings (but GSL generally has poor support for complex numbers...)
And, the statistics package includes various probability distributions.

\begin{itemize}
\item Built-in {\tt version 3.6.2}:
  \begin{itemize}
  \item {\tt [a,ierr] = airy(k,z,opt)} --- $\O{Ai}(z), \O{Ai}'(z), \O{Bi}(z), \O{Bi}'(z)$ with optional scaling; $z$ can be real or complex
  \item {\tt [r,ierr] = besselj(alpha,x,opt)/bessely/besseli/besselk/besselh} --- Bessel functions
      $\O{J}_\al(x)$, $\O{Y}_\al(x)$, $\O{I}_\al(x)$, $\O{K}_\al(x)$, $\O{H}_\al(x)$ with optional scaling; $\al$ must be real, $x$ can be complex
  \item {\tt beta(a,b), betaln(a,b)} --- Beta function $B(a,b)$ or natural logarithm of Beta function $\ln(B(a,b))$; with $a, b$ real
  \item {\tt betainc(x,a,b)} --- incomplete Beta function with $x, a, b$ real
  \item {\tt bincoeff(n,k)} --- binomial coefficient; $n, k$ integers;
      [accepts some non-integers but returns suspicious values]
  \item {\tt erf(z), erfc(z), erfcx(z)} --- error function $\O{erf}(z)$, complementary error function $\O{erfc}(z) = 1-\O{erf}(z)$,
      scaled complementary error function $e^{z^2}\O{erfc}(z)$; $z$ real
  \item {\tt erfinv(x)} --- inverse error function; $x$ real
  \item {\tt gamma(z), lgamma(z)/gammaln(z)} --- gamma function $\Gam(z)$ or natural logarithm of gamma function $\ln\Gam(z)$; $z$ real;
      [looks like iffy/low precision implementation: {\tt gamma(6)=119.99999999999997}]
  \item {\tt gammainc(x,a,opt)} --- normalized incomplete gamma function $P(a,x)$ or complementary version; $x, a$ real
  \item {\tt legendre(n,x,opt)} --- Legendre functions of degree $n$ and order $m=0\dots n$, $P^m_n(x)$ optionally normalized
  \item elementary functions; $x, y$ real, $z$ real or complex
    \begin{itemize}
    \item {\tt exp(z)}, {\tt expm1(z)}, {\tt log(z)}, {\tt reallog(z)}, {\tt log1p(z)}, {\tt log10(z)}, {\tt log2(z)}
    \item {\tt pow2(z)}, {\tt pow2(f,e)}, {\tt nextpow2(x)}, {\tt realpow(x,y)}, {\tt sqrt(z)}, {\tt cbrt(x)}, {\tt nthroot(x,n)}
    \item {\tt cos(z)}, {\tt sin(z)}, {\tt tan(z)}, {\tt cosh(z)}, {\tt sinh(z)}, {\tt tanh(z)},
        {\tt sec(z)}, {\tt csc(z)}, {\tt cot(z)}, {\tt sech(z)}, {\tt csch(z)}, {\tt coth(z)}
    \item {\tt acos(z)}, {\tt asin(z)}, {\tt atan(z)}, {\tt acosh(z)}, {\tt asinh(z)}, {\tt atanh(z)},
        {\tt asec(z)}, {\tt acsc(z)}, {\tt acot(z)}, {\tt asech(z)}, {\tt acsch(z)}, {\tt acoth(z)}
    \item {\tt atan2(y,x)}, {\tt abs(z)}, {\tt arg(z)}, {\tt conj(z)}, {\tt real(z)}, {\tt imag(z)}
    \item {\tt sind(z)}, {\tt cosd(z)}, {\tt tand(z)}, {\tt secd(z)}, {\tt cscd(z)}, {\tt cotd(z)},
        {\tt asind(z)}, {\tt acosd(z)}, {\tt atand(z)}, {\tt asecd(z)}, {\tt acscd(z)}, {\tt acotd(z)}
    \item {\tt ceil(z)}, {\tt fix(z)}, {\tt floor(z)}, {\tt round(z)}, {\tt roundb(z)},
        {\tt max(z1...zn)}, {\tt min(z1...zn)}, {\tt hypot(z1...zn)}
    \end{itemize}
  \end{itemize}
\item Package {\tt specfun-1.1.0}
  \begin{itemize}
  \item {\tt cosint(z) = Ci(z)} --- cosine integral function $\O{Ci}(z)=\int_x^\infty\frac{\cos(t)}{t}\,dt$; $z$ real or complex;
      [uses expression in terms of exponential integrals $\O{E}_1$ or $\O{Ei}$]
      [iffy for some cases: {\tt Ci(0) = NaN - NaN*i}]
  \item {\tt sinint(z) = Si(z)} --- sine integral function $\O{Si}(z)=\int_0^x\frac{\sin(t)}{t}\,dt$; $z$ real or complex;
      [uses summation of Bessel $\O{J}$ functions]
      [iffy for some cases: {\tt Si(Inf) = Si(1e10) = NaN + NaN*i}]
  \item {\tt heaviside(x,threshhold)} --- Heaviside function; $z$ real or complex
      (but poor behaviour for complex (all non-real give 1) --- this raises the question of what {\it should} be the extension of
      the heaviside to $\CC$?)
  \item {\tt dirac(z)} --- Dirac delta function: {\tt Inf} iff $x=0$ else {\tt 0}; $z$ real or complex
  \item {\tt zeta(z)} --- Riemann zeta function; $z$ real or complex;
      [uses quadrature $\int_0^\infty\frac{x^{z-1}}{\Gam(z)(e^{x}-1)}\,dx$ for real $z>1$;
      summation $\frac{1}{1-2^{1-z}}\sum_{k=1}^\infty \frac{(-)^{k-1}}{k^{z}}$ for values with $\Re z\geq0$;
      and otherwise recursively computes $2^{z}\pi^{z-1}\sin(\pi z/2)\Gam(1-z)\zeta(1-z)$]
  \item {\tt lambertw(b,z)} --- computes Lambert W function $\O{W}(z)$ for branch $b$; $z$ real or complex, $b$ integer;
      [uses series around $-1/e$ and asymptotic expansion at $0$ and $\infty$ to get initial guess, then
      Halley iteration to refine]
  \item {\tt psi(x)} --- computes $\tfrac{d}{dx}\ln\Gam(x)$; real $x$;
      [does simple finite centered difference approximation]
  \item {\tt expint(z)=expint\_E1(z), } --- computes exponential integral $\int_x^\infty e^{t}/t\,dt$; $z$ real or complex;
      [uses {\tt expint\_Ei(-z)+C}]
  \item {\tt expint\_Ei(z)} --- computes exponential integral $-\int_{-x}^\infty e^{t}/t\,dt$; $z$ real or complex;
      [uses quadrature for real $z>2$ or $z<0$; if $|z|\geq10$, $\Im z\leq0$ uses asympotic expansion (with conjugation for $\Im z>0$);
      otherwise uses series expansion]
  \item {\tt erfcinv(z)} --- inverse complementary error function; $z$ real or complex;
      [simply returns {\tt erfinv(1-x)} --- which will have precision issues for $x\sim1$]
  \item {\tt [y,p] = laguerre(z,n)} --- computes the Laguerre polynomial of order $n$ at $z$; $z$ real or complex, $n>0$ integer;
      [computes polynomial coefficients via recursion and then evaluates resulting polynomial at $z$;
      sounds iffy to me...]
  \item {\tt [k,e] = ellipke(m)} --- returns complete elliptic integrals of first $\O{K}(m)$ and second $\O{E}(m)$ kind; real $m\leq1$;
      [uses AGM algorithm and a transform for $m<0$]
  \end{itemize}
\item Package {\tt miscellaneous-1.1.0}
  \begin{itemize}
  \item {\tt chebyshevpoly(kind,order,z)} --- computes first $T_n(x)$ or second $U_n(x)$ Chebyshev polymials
      (coefficients or value at $z$); $z$ real or complex;
      [uses recursion for coefficients; value is evaluated from polymomial; seems iffy to me]
  \item {\tt hermitepoly(order,z)} --- computes Hermite polynomials $H_n(z)$ (coefficients or value at $z$); $z$ real or complex;
      [uses low-precision (8-digits) table for $0\leq\text{order}\leq50$; otherwise uses recursion to compute polynomial
      coefficients; value is evaluated from polynomial coefficients; seems iffy]
  \item {\tt laguerrepoly(order,z)} --- computes Laguerre polynomials $L_n(z)$ (coefficients or value at $z$); $z$ real or complex;
      [uses recursion to compute polynomial coefficients; value is evaluated from polynomial coefficients; seems iffy]
  \item {\tt legendrepoly(order,z)} --- computes Legendre polynomials $L_n(z)$ (coefficients or value at $z$); $z$ real or complex;
      [uses recursion to compute polynomial coefficients; value is evaluated from polynomial coefficients; seems iffy]
  \end{itemize}
\end{itemize}


%%%%%%%%%%%%%%%%%%%%%%%%%%%%%%%%%%%%%%%%%%%%%%%%%%%%%%%%%%%%%%%%%%%%%%%%%%%%%%%%
\section{Elementary functions}

\[\F{ldexp}\] \[\F{frexp}\] \[\F{mod}\] \[\F{abs}\] \[\F{hypot}\] \[\F{hypot3}\]

Also an $\O{fmod}(z,a,b)$ which reduces $z$ with high-accuracy modulo $a+b$ (a ``pseudo-quad-double'' functionality)
which is useful for trigonometric range-reduction, for example.

A handy function is a multi-precision constant factory: takes decimal string, say, and returns $n$ doubles whose sum
is equal to the constant.  (Case $n=2$ is just the constructor for qdouble...)

%%%%%%%%%%%%%%%%%%%%%%%%%%%%%%%%%%%%%%%%
\subsection{Polynomials}

%%%%%%%%%%%%%%%%%%%%
\subsubsection{sqrt}
Assume that $a<0$ or $b\neq0$, then we have:
\[ \sqrt{a+b\ii} = \frac{b}{2d} + d\ii \qquad d=\sqrt{\frac{-a\pm\sqrt{a^2+b^2}}{2}} \]
We could also use a trigonometric approach: $\sqrt{a+b\ii} = \sqrt{\sqrt{a^2+b^2}}\left(\cos(\atan(\tfrac{b}{a})/2) + \ii\sin(\atan(\tfrac{b}{a})/2)\right)$.

\begin{implementation}
Implementation notes for $\sqrt{z}$ for $z\geq0$:
\begin{itemize}
\item Use the Newton iteration $x_{n+1} = \frac12(x_n + \frac{z}{x_n})$
\item This converges (for doubles) in only a couple more iterations than Halley's method (and each iteration is cheaper)
\item Gives exact answers (for gcc) in 4 or 5 iterations
\item For starting value, let $z=2^N\cdot f$ with $N$ even and $f\in[1/2, 2)$ (use {\tt frexp}),
  and use the starting guess of $2^{N/2}\widehat{f}$ where $\widehat{f}$ is the linearly interpolated value of $\sqrt{\ }$ on $[1/2,2)$
  for the value $f$.
\item Brent claims that it is better (optimal?) to use $\sqrt{a} = a/\sqrt{a}$ and compute $a^{-1/2}$ directly via the 3rd-order iteration
  $x_{n+1} = x_n - \frac{x_n}{2}(\eps_n - \tfrac34 \eps_n^2)$ where $\eps_n = a x_n^2 - 1$.  But for the case of a double, this
  converges no faster than Newton.
\end{itemize}
\end{implementation}

%%%%%%%%%%%%%%%%%%%%
\subsubsection{cbrt}
For complex values, use trigonometric approach.  (Use a newton-step to clean digits?)
Also, note that if $x=z^{1/3}$ is a cube-root, then so are $\zeta x$, $\zeta^2 x$, where $\zeta=-\frac12+\ii\frac{\sqrt{3}}{2}$ is a cube root of unity.
\begin{implementation}
Implementation notes for $\sqrt[3]{z}$ for $z\in\RR$:
\begin{itemize}
\item Use the Newton iteration $x_{n+1} = \frac{x_n^4 + 2 z x_n}{2 x_n^3 + z}$
\item This converges (for doubles) in only a couple more iterations than Halley's method (and each iteration is cheaper)
\item For starting value, let $z=2^N\cdot f$ with $N=3k$ and $f\in[1/2, 4)$ (use {\tt frexp}),
  and use the starting guess of $2^{N/3}\widehat{f}$ where $\widehat{f}$ is the linearly interpolated value of $\sqrt[3]{\ }$ on $[1/2,4)$
  for the value $f$.
\item Gives exact answers (for gcc) in 5 or 6 iterations (Halley takes about 4 iterations)
\end{itemize}
\end{implementation}

%%%%%%%%%%%%%%%%%%%%
\subsubsection{pow}
\[ \F{pow}(z,r) = z^{r} \qquad z\in\RR, r\in\RR \]
\[ \F{pow\_int}(z,n) = z^{n} \qquad z\in\RR, r\in\ZZ \]
integer, real, complex versions (branches?), fully general (integer)
$x^2$, $x^3$, $x^y-1$, $\sqrt{1+x}-1$, $\sqrt{1+x}-x$

\begin{implementation}
Implementation notes for $z^n$, $n\in\ZZ$:
\begin{itemize}
\item Assume $n>0$ (compute $z^{-n}$ and use $z^n = 1/z^{-n}$)
\item Use repeated doubling algorithm - takes $\log n$ steps
\item Gives exact answers
\end{itemize}
\end{implementation}

%%%%%%%%%%%%%%%%%%%%
\subsubsection{polynomial roots}
\[ \F{solve\_linear}(a,b) = \rho \text{\quad s.t.\quad } a\rho+b=0 \qquad a,b\in\RR \]
\[ \F{solve\_quadratic}(a,b,c) = \rho \text{\quad s.t.\quad } a\rho^2+b\rho+c=0 \qquad a,b,c\in\RR \]
\[ \F{solve\_cubic}(a,b,c,d) = \rho \text{\quad s.t.\quad } a\rho^3+b\rho^2+c\rho+d=0 \qquad a,b,c,d\in\RR \]
\[ \F{solve\_quartic}(a,b,c,d,e) = \rho \text{\quad s.t.\quad } a\rho^4+b\rho^3+c\rho^2+d\rho+e=0 \qquad a,b,c,d,e\in\RR \]
\[ \F{polynomial\_root}(\vec{a},n) = \rho \text{\quad s.t.\quad } \sum_{k=0}^{n}a^{n-k}\rho^k=0 \qquad \vec{a}\in\RR^{n+1} \]
Complex versions also...

For $n$-th root of $r$, newton looks like $x_{k+1} = \frac{n-1}{n}x_k + \frac{r}{n x_k^{n-1}}$.

There are direct methods here, trigonometric (using complex numbers), etc.

For general polynomial root-finding, NR recommends ``Laguerre'' method + deflation.  (Which works decently as long as there are not repeated roots...)

Remark: to transform a general monic quadratic $x^4 + bx^3 + cx^2 + dx + e$ to a ``depressed'' quadratic we take $x=y - \tfrac{b}{4}$ and get
\[ y^4 + (c-\tfrac{3b^2}{8})y^2 + (\tfrac{b^3}{8} - \tfrac{bc}{2} + d)y + (e - \tfrac{3b^4}{256} + \tfrac{b^2c}{16} - \tfrac{bd}{4}) \]


%%%%%%%%%%%%%%%%%%%%
\subsubsection{polynomial\_value}
\[ \F{polynomial\_value}(\vec{a},n,z) = \sum_{k=0}^{n}a^{n-k}z^k \qquad \vec{a}\in\RR^{n+1}, z\in\RR \]

%%%%%%%%%%%%%%%%%%%%
\subsubsection{general notes on root-finding for basic functions}
This technique can be useful for implementing basic functions, especially for new types.
Starting from an initial, easily computed, approximation, you refine via Newton's method, $x'=x-f(x)/f'(x)$ to get the desired precision.
One can also use Halley's method for theoretically improved convergence rate, but in practice, for a given precision, Newton's method may
actually provide better performance.
\begin{itemize}
  \item For $\sqrt{a}$, we use $f(x)=x^2-a$, then
    \[ x' = \frac12(x+a/x) \]
  \item For $1/\sqrt{a}$, we use $f(x)=x^{-2}-a$, then
    \[ x' = \frac{x}2(3-a x^2) \]
  \item For $\sqrt[n]{a}$, we use $f(x)=x^n-a$, then
    \[ x' = \frac1n((n-1)x+a/x^{n-1}) \]
  \item For $1/\sqrt[n]{a}$, we use $f(x)=x^{-n}-a$, then
    \[ x' = \frac{x}n((n+1) - a x^n) \]
\end{itemize}


%%%%%%%%%%%%%%%%%%%%%%%%%%%%%%%%%%%%%%%%
\subsection{Exponentials}
Exponential function:
\[ \exp(z) = e^{z} = \sum_{n=0}^\infty\frac{z^n}{n!} \qquad z\in\CC   \DEF{$\exp(z), e^{z}$}\]
\[ \F{exp}(z) = e^z \]
\[ \F{exp\_m1}(z) = e^z-1 \]
\[ \F{exp\_m1\_dx}(z) = \frac{e^z-1}{z} \]
\[ \F{e\_n}(z) = e_{n}(z) = \sum_{j=0}^n\frac{z^j}{j!} \]
\[ \F{ex}(z,n) = e^z - e_{n-1}(z) = \sum_{j=n}^\infty \frac{z^j}{j!} \]
\[ \F{exd}(z,n) = \F{ex}(z,n)\frac{n!}{z^n} = \sum_{j=n}^\infty\frac{n!}{j!}z^{j-n} = \sum_{k=0}^\infty\frac{n!}{(n+k)!}z^k = \sum_{k=0}^\infty\frac{x^k}{(n+1)_k} \]
Other useful functions: $2^x$, $10^x$, $e^z-1-z$, $\frac{e^z-1-z}{z^2}$, $\frac{ex(z,n)}{z^{n}}$

Remark: $e^{x+y\ii} = e^x\cos y + \ii e^x\sin y$ for $x,y\in\RR$.

Continued fraction expansions:
\[ e^z = \frac{1}{1-{}} \frac{z}{1+{}} \frac{z}{2-{}} \frac{z}{3+{}} \frac{z}{2-{}} \frac{z}{5+{}} \cdots\]
(The following might be useful for computing $(e^z-1)/z$?)
\[ e^z = 1+\frac{z}{1-} \frac{z}{2+} \frac{z}{3-} \frac{z}{2+} \frac{z}{5-} \frac{z}{2+} \frac{z}{7-} \cdots \]
\[ e^z = 1+\frac{z}{1-(z/2)+} \frac{z^2/(4\cdot3)}{1+} \frac{z^2/(4\cdot15)}{1+} \frac{z^2/(4\cdot35)}{1+} \cdots \frac{z^2/(4(4n^2-1))}{1+} \cdots \]
\[ \F{ex}(z,n) = \frac{z^n}{n!-{}} \frac{n! z}{(n+1)+{}} \frac{z}{(n+2)-{}} \frac{(n+1)z}{(n+3)+{}} \frac{2z}{(n+4)-{}} \frac{(n+2)z}{(n+5)+{}} \frac{3z}{\cdots} \]
The following works well:
\[ \F{exd}(z,n) = \frac{1}{1-} \frac{z}{(n+1)+} \frac{z}{(n+2)-} \frac{(n+1)z}{(n+3)+} \frac{2z}{(n+4)+} \frac{(n+2)z}{(n+5)-} \frac{3z}{(n+6)+} \cdots\]


\begin{implementation}
Implementation notes for $\exp(z)$ for $z\in\RR$:
\begin{itemize}
\item For $z<0$, use $\exp(z)=\frac{1}{\exp(-z)}$ and compute $\exp(-z)$ (to avoid negative cancellation in series)
\item For $|z|>\tfrac12$, use $\exp(2^N\cdot x) = \exp(x)^{2^N}$ and compute $\exp(z\cdot 2^N)$ with $z\cdot 2^N\leq\tfrac12$
\item Finally, use standard power-series $\exp(z)=\Hyper{0}{0}{}{}{z} = \sum_{n=0}^\infty\frac{z^n}{n!}$ for the argument $0<z\leq\tfrac12$ to compute.
\item In tests, this gives results which are within 1 ulp of the (gcc math library computed) actual values, many values are exactly the same.
  (Note that this is actually depending on intermediate values being computed with higher precision!  using the {\tt -fexcess-precision=standard} option
  in gcc gives much higher error rates!  Explicitly using {\tt long double} variables (20-byte) brings us back to the low error-rate.)
\item Luke claims that convergence is better for $\Re(z)>-1$ computing:
  \[ e^{-z} = (1+z)^{-1} \Hyper{1}{1}{z}{2+z}{-z} = \frac{1}{1+z}\sum_{k=0}^\infty \frac{(-z)^k}{k!} \frac{(z)_k}{(2+z)_k} \]
  (If $\al_n$ is the $n$th term in the sum, then $\al_{n+1}=-\al_{n}\frac{z(z+n)}{(n+1)(z+n+2)}$.)
\item Another idea: write $z=(\ln2)^N\xi$ so that $e^z=2^N e^\xi$
\item Idea: use high-precision range-reduction to extract multiples of $\ln 2$, for better results
\item For double-precision, note that $e^{710}=\infty$ already, so in $e^{2^n f} = (e^f)^{2^n}$, we have~$n<10$.
\end{itemize}

Implementation notes for $\exp(z)-1$ for $z\in\RR$:
\begin{itemize}
\item If $|z|\geq1/2$, just compute $\exp(z) - 1$ directly
\item Otherwise use a power series $\sum_{n=1}^\infty \frac{x^n}{n!}$
\item This takes up to 18 terms in the series to converge for double precision
\item There are sporadic errors of 1 or 2 ulps (vs gcc) in both branches of the above
\item for the series, this can be ameliorated a lot by using long double
\item can we use the trick $e^x-1 = (e^{x/2}+1)(e^{x/4}+1)(e^{x/8}+1)(e^{x/8}-1)$ (etc.) somehow?
\end{itemize}

Note that for $z\in\CC$ there can be cancellation of the real-part away from $z=1$ also
(in particular when $(\Re z)\cos(\Im z)\sim 1$ ...)
\end{implementation}

Alternative notes:
For approximating $e^x$, we define $g(x)=\frac{e^x-1}{e^x+1}$ so that $e^x=\frac{1+g(x)}{1-g(x)} = 1+\frac{2g(x)}{1-g(x)}$.
Remark: $g(x)$ is an odd function, so that $g(x)=x f(x^2)$.  Suppose that $g(x)=p(x)/q(x)$, then $e^x = 1+\frac{2p(x)}{q(x)-p(x)}$.
One such approximation which gives (nearly) double-precision accuracy for $x\in(-0.45,0.45)$ is given by
\begin{eqnarray*}
  e^x &\approx& 1+2\frac{x p(x^2)}{q(x^2) - xp(x^2)} \\
  p(y) &=& \frac{1}{2} + \frac{y}{72} + \frac{y^2}{30240} = \frac{1}{2}\left(1+\frac{y}{72}\left(1+\frac{y}{420}\right)\right) \\
  q(y) &=& 1+\frac{y}{9}+\frac{y^2}{1008} = \left(\frac{y}{112}+1\right)\frac{y}{9} + 1
\end{eqnarray*}
Another approximation covering $x\in(-1.2,1.2)$ is given by
\begin{eqnarray*}
  p(y) &=& \frac{1}{2} + \frac{5}{312}y + \frac{1}{11440}y^2 + \frac{1}{17297280}y^3 \\
  q(y) &=& 1 + \frac{3}{26}y + \frac{5}{3432}y^2 + \frac{1}{308880}y^3
\end{eqnarray*}



%%%%%%%%%%%%%%%%%%%%%%%%%%%%%%%%%%%%%%%%
\subsection{Logarithms}

\[ \log(z) = \rho \text{\quad s.t. \quad} e^\rho = z \]
\[ \log_b(z) = \log(z)/\log(b) = \rho \text{\quad s.t. \quad} b^\rho = z \]

\[ \F{log}(z) = \log(z) \]
\[ \F{Log}(z,w) = \O{Log}_{[w]}(z) \qquad\text{$w$th branch}\]
\[ \F{log\_b}(z, b) = \log_{b}(z) \]
\[ \F{log\_p1}(z) = \log(1+z) \]
\[ \F{log\_p1\_mx}(z) = \log(1+z)-z \]
\[ \F{log\_2}(z) = \log_{2}(z) \]
\[ \F{log\_10}(z) = \log_{10}(z) \]
$z\ln z$, $\frac{\ln(1+z)}{z}$, $\frac{\ln(1+z)-z}{-z^2/2}$, etc.
partial series, etc.

Remark: $\ln x+y\ii = \ln r + \theta\ii$, where $x,y\in\RR$, $x+y\ii=r e^{\theta\ii}$, thus
$r=|x+y\ii|=\sqrt{x^2+y^2}$ and $\theta=\arg x+y\ii = \O{atan2}(y,x)$.

\[ \log(1+z) = \sum_{n=1}^\infty (-)^{n-1} \frac{z^n}{n} \qquad |z|\leq1, z\neq-1 \]
\[ \log(z) = 2\sum_{n=1}^\infty\left(\frac{z-1}{z}\right)^n\frac{1}{n} \qquad \Re z\geq\frac12 \]
\[ \log(\frac{z+1}{z-1}) = 2\sum_{n=0}^\infty\frac{z^{-2n-1}}{2n+1} \qquad |z|\geq1, z\neq\pm1 \]
\[ \log(z) = 2\sum_{n=0}^\infty\left(\frac{z-1}{z+1}\right)^{2n+1} \frac{1}{2n+1} \qquad \Re z\geq0, z\neq0 \]
\[ \log(z+a) = \log(a) + 2\sum_{n=0}^\infty \left(\frac{z}{2a+z}\right)^{2n+1} \frac{1}{2n+1} \qquad a>0, \Re z\geq-a, z\neq-a \]

Continued fraction for $z\in\CC\setminus(-\infty,-1]$:
\[ \log(1+z) = \frac{z}{1+} \frac{z}{2+} \frac{z}{3+} \frac{4z}{4+} \frac{4z}{5+} \frac{9z}{6+} \cdots \]
Continued fraction on slit-plane:
\[ \log(\frac{1+z}{1-z}) = \frac{2z}{1-} \frac{z^2}{3-} \frac{4z^2}{5-} \frac{9z^2}{7-} \cdots \]

\begin{implementation}
Implementation notes for $\log(z)$ for $z>0$:
\begin{itemize}
\item For $z>1$, use $\log(z) = -\log\frac1z$ and compute $\log\frac1z$ (to get $z\leq1$)
\item Next, use $\log(2^N x) = N\log 2 + \log x$, so can reduce to get $\tfrac12\leq z\leq\tfrac32$
\item Finally, use series $\log(1+z) = -\sum_{n=1}^\infty(-)^n\frac{z^n}{n}$
\end{itemize}
In tests, this gives results which are within 3 ulp of the (gcc math library computed) actual values.
\end{implementation}

%%%%%%%%%%%%%%%%%%%%%%%%%%%%%%%%%%%%%%%%
\subsection{Trigonometric functions}

polar to/from rectangular, angle, radians to/from degrees to/from gradians, etc.

{\tt
sin, cos, tan, sec, csc, cot,
sinh, cosh, tanh, seca, csca, cota,
asin, acos, atan, asec, acsc, acot,
asinh, acosh, atanh, aseca, acsca, acota,
sind, cosd, ..., atan2, gudermannian,
versine, haversine, coversine, hacoversine, exsecant, excosecant,
sinc, sinc\_a}
\[ \O{sinc}(x) = \frac{\sin \pi x}{\pi x} \]
\[ \O{sin}_\pi(x) = \sin(\pi x) \]
(and etc.  Allows for reduced rounding error in many usages of $\sin$.)

%%%%%%%%%%%%%%%%%%%%
\subsubsection{Circular}
\[ \cos(z) = \sum_{n=0}^\infty (-)^n\frac{z^{2n}}{(2n)!} \]
\[ \sin(z) = \sum_{n=0}^\infty (-)^n\frac{z^{2n+1}}{(2n+1)!} \]
\[ \tan(z) = \frac{\sin z}{\cos z} \]
\[ \sec(z) = \frac{1}{\cos z} \]
\[ \cot(z) = \frac{1}{\tan z} = \frac{\cos z}{\sin z} \]
\[ \csc(z) = \frac{1}{\sin z} \]

Remark: $\cos x+y\ii = \cos x \cosh y - \ii \sin x \sinh y$, and 
$\sin x+y\ii = \sin x \cosh y + \ii \cos x \sinh y$, when $x,y\in\RR$.

\begin{implementation}
\begin{itemize}
\item continued fraction $\tan z = \frac{z}{1-} \frac{z}{3-} \frac{z}{5-} \cdots$, ($z\neq\frac\pi2\pm n\pi$)
\item $\cot z = \frac12 + 2z\sum_{k=1}^\infty\frac{1}{z^2 - k^2\pi^2}$
\item $\asin z = z + \frac{1}{2}\frac{z^3}{3} + \frac{1\cdot3}{2\cdot4}\frac{z^5}{5} + \frac{1\cdot3\cdot5}{2\cdot4\cdot6}\frac{z^7}{7} + \cdots$
\item continued fraction$\frac{\asin z}{\sqrt{1 - z^2}} = \frac{z}{1 - } \frac{1\cdot2 z^2}{3-} \frac{1\cdot2 z^2}{5-} \frac{3\cdot4 z^2}{7-} \frac{3\cdot4 z^2}{9-} \cdots$
\item Expansions for $\atan z$
  \begin{itemize}
  \item continued fraction $\atan z = \frac{z}{1+} \frac{z^2}{3+} \frac{4z^2}{5+} \frac{9z^2}{7+} \cdots$
  \item for $|z|\leq 1$, $\atan z = z - \frac{z^3}{3} + \frac{z^5}{5} - \frac{z^7}{7} \cdots$
  \item for $|z|> 1$, $\atan z = \frac{\pi}{2} - \frac{1}{z} + \frac{1}{3z^2} - \frac{1}{5z^5} \cdots$
  \item for any $z\neq\pm\ii$, $\atan z = \frac{z}{1+z^2}\left(1 + \frac{2}{3}\frac{z^2}{1+z^2} + \frac{2\cdot4}{3\cdot5}(\frac{z^2}{1+z^2})^2+\cdots\right)$
  \end{itemize}
\end{itemize}
\end{implementation}

Remarks:
\[ 1-\cos(z)=2\sin^2(z/2) \qquad\text{since $\sin(z/2)=\pm\sqrt{(1-\cos(z))/2}$} \]
\[ 1+\cos(z)=2\cos^2(z/2) \qquad\text{since $\cos(z/2)=\pm\sqrt{(1+\cos(z))/2}$} \]
\[ \tan(z/2) = \pm\sqrt{\frac{1-\cos z}{1+\cos z}} = \frac{1-\cos z}{\sin z} = \frac{\sin z}{1+\cos z} \]

%%%%%%%%%%%%%%%%%%%%
\subsubsection{Hyperbolic}
\[ \cosh(z) = \cos(\ii z) = \sum_{n=0}^\infty \frac{z^{2n}}{(2n)!} = \frac{e^{z} + e^{-z}}{2} \]
\[ \sinh(z) = -\ii\sin(\ii z) = \sum_{n=0}^\infty \frac{z^{2n+1}}{(2n+1)!} = \frac{e^{z} - e^{-z}}{2} \]
\[ \tanh(z) = -\ii\tan(\ii z) = \frac{\sinh z}{\cosh z} = \frac{e^{z} - e^{-z}}{e^{z} + e^{-z}} \]
\[ \O{sech}(z) = \frac{1}{\cosh z} \]
\[ \O{coth}(z) = \frac{1}{\tanh z} = \frac{\cosh z}{\sinh z} \]
\[ \O{csch}(z) = \frac{1}{\sinh z} \]
Power series:
\[ \cosh(z) = \sum_{n=0}^\infty\frac{z^{2n}}{(2n)!} \]
\[ \sinh(z) = \sum_{n=0}^\infty\frac{z^{2n+1}}{(2n+1)!} \]
Continued fraction:
\[ \tanh(z) = \frac{z}{1+} \frac{z^2}{3+} \frac{z^2}{5+} \frac{z^2}{7+} \cdots \qquad z\neq\frac\pi2\ii\pm n\pi\ii \]

%%%%%%%%%%%%%%%%%%%%
\subsubsection{Inverse Circular}
\[ \acos(z) = \rho \text{\quad s.t.\quad} \cos\rho = z \]
\[ \asin(z) = \rho \text{\quad s.t.\quad} \sin\rho = z \]
\[ \atan(z) = \rho \text{\quad s.t.\quad} \tan\rho = z \qquad \rho\in(-\pi/2, +\pi/2) \]
\[ \O{atan2}(y,x) = \atan(y/x) \text{\quad s.t. correct for $x\sim0$ and signs} \]
Power series:
\[ \asin(z) = \sum_{n=0}^\infty z^{2n+1} \frac{1\cdot3\cdot5\cdots(2n-1)}{(2n+1)\cdot2\cdot4\cdot6\cdots(2n)} \qquad |z|<1 \]
\[ \atan(z) = \sum_{n=0}^\infty(-)^n\frac{z^{2n+1}}{2n+1} \qquad |z|\leq1, z^2\neq-1 \]
\[ \atan(z) = \frac{\pi}{2} - \sum_{n=0}^\infty(-)^n\frac{z^{-2n-1}}{2n+1} \qquad |z|>1 \]
Continued fraction:
\[ \atan(z) = \frac{z}{1+} \frac{z^2}{3+} \frac{4z^2}{5+} \frac{9z^2}{7+} \frac{16z^2}{9+} \cdots
  \qquad z\notin (-\ii\infty,-\ii]\cup[\ii,\ii\infty) \]
\[ \frac{\asin(z)}{\sqrt{1-z^2}} = \frac{z}{1-} \frac{1\cdot2\cdot z^2}{3-} \frac{1\cdot2\cdot z^2}{5-} \frac{3\cdot4\cdot z^2}{7-} \frac{3\cdot4\cdot z^2}{9-} \cdots
  \qquad z\notin (-\infty,-1]\cup[1,\infty) \]

Remark: $\acos x+y\ii = \acos\beta + \ii\O{sign}_+(y)\ln(\al+\sqrt{\al^2-1})$ and
$\asin x+y\ii = \asin\beta + \ii\O{sign}_+(y)\ln(\al+\sqrt{\al^2-1})$;
where $x,y\in\RR$, $\al = \frac{\sqrt{(x+1)^2+y^2} + \sqrt{(x-1)^2+y^2}}{2}$,
$\beta = \frac{\sqrt{(x+1)^2+y^2} - \sqrt{(x-1)^2+y^2}}{2}$,
and \[\O{sign}_+(y) = \begin{cases}+1&y\geq0\\-1&y<0\end{cases}\]

For $z\notin(-\infty,-1]\cup[1,\infty)$,
\[ \acos(z) = \frac{\pi}{2} - \asin(z) = -\ii \O{Log}(z + \ii(1-z^2)^{1/2}) \]
and
\[ \asin(z) = -\ii \O{Log}((1-z^2)^{1/2} + \ii z) \]
We also have
\[ \O{acsc}(z)=\asin(1/z) \qquad z\notin[-1,1] \]
and $\O{asec}(z)$ is defined for $z\notin[-1,1]$ with $\O{asec}(z)+\O{acsc}(z)=\pi/2$.

For $z\notin(-\infty\ii,-\ii]\cup[\ii,\ii\infty)$, we have
\[ \atan(z) = \frac{\ii}{2}\O{Log}(\frac{1-\ii z}{1+\ii z}) = \frac{\ii}{2}\O{Log}(\frac{\ii+z}{\ii-z}) \]
We also have $\O{acot}(z) = \atan(1/z)$ defined for $z\notin[-\ii,\ii]$, and $\atan(z)+\O{acot}(z)=\O{sign}_+(z)\pi/2$.

%%%%%%%%%%%%%%%%%%%%
\subsubsection{Inverse Hyperbolic}
\[ \acosh(z) = \rho \text{\quad s.t.\quad} \cosh\rho = z \]
\[ \asinh(z) = \rho \text{\quad s.t.\quad} \sinh\rho = z \]
\[ \atanh(z) = \rho \text{\quad s.t.\quad} \tanh\rho = z \]

%%%%%%%%%%%%%%%%%%%%
\subsubsection{Misc}
\[ \O{gud}(z) = 2\atan(e^{z}) - \frac\pi2 \qquad\text{Gudermannian} \DEF{$\O{gud}(z)$}\]
\[ \O{gud}^{-1}(z) = \log(\tan(\frac{\pi}{4}+\frac{z}{2})) = \log(\sec(z) + \tan(z)) \]

Note: for a function $\O{F}(z)$, we define
$\O{verF}(z) = 2 \O{F}(z/2)^2$, 
$\O{coF}(z) = \O{F}(\tfrac{\pi}{2} - z)$, 
$\O{haF}(z) = \O{F}(z)/2$, 
$\O{exF}(z) = \O{F}(z) - 1$, giving us
\[ \O{vsin}(z) = \O{versine}(z) =  2\sin(z/2)^2 = 1 - \cos(z) = \O{excosine}(z) \]
\[ \O{hvsin}(z) = \O{haversine}(z) = \frac12\O{versine}(z) = \frac{1-\cos(z)}{2} = \sin(z/2)^2 \]
\[ \O{cvsin}(z) = \O{coversine}(z) = 1 - \sin(z) = \O{versine}(\tfrac\pi2 - z) = 2\sin(\tfrac\pi4 - \tfrac{x}{2}) \]
\[ \O{chvsin}(z) = \O{hcvsin}(z) = \O{hacoversine}(z) = \O{cohaversine}(z) = \frac12\O{coversine}(z) = \frac{1-\sin(z)}{2} \]
\[ \O{vcos}(z) = \O{vercosine}(z) =  2\cos(z/2)^2 = 1 + \cos(z) \]
\[ \O{hvcos}(z) = \O{havercosine}(z) = \frac12\O{vercosine}(z) = \frac{1+\cos(z)}{2} \]
\[ \O{cvcos}(z) = \O{covercosine}(z) = 1 + \sin(z) = \O{vercosine}(\tfrac\pi2 - z) = 2\sin(\tfrac\pi4 + \tfrac{x}{2}) \]
\[ \O{hcvcos}(z) = \O{chvcos}(z) = \O{hacovercosine}(z) = \O{cohavercosine}(z) = \frac12\O{covercosine}(z) = \frac{1+\sin(z)}{2} \]
\[ \O{exsec}(z) = \O{exsecant}(z) = \sec(z) - 1 = \frac{1-\cos(z)}{\cos(z)} = \frac{\O{versine}(z)}{\cos(z)} = 2\sin(z/2)^2\sec(z) \]
\[ \O{excsc}(z) = \O{excosecant}(z) = \O{exsecant}(\tfrac\pi2 - z) = \csc(z) - 1\]
[The equalities for $\O{excsc}$ need to be verified... they appear to be incorrect]

%%%%%%%%%%%%%%%%%%%%%%%%%%%%%%%%%%%%%%%%
\subsection{Miscellaneous}

%%%%%%%%%%%%%%%%%%%%
\subsubsection{AGM}

{\tt agm(a,b), agm(a,b,out c)}

Given $a_0=\al$, $b_0=\beta$, let 
\begin{eqnarray*}
  a_{n+1} &=& (a_n + b_n)/2 \\
  b_{n+1} &=& (a_n \cdot b_n)^{1/2} \\
  c_{n+1} &=& (a_n - b_n)/2
\end{eqnarray*}
then $\O{agm}(\al,\beta) = \lim_{n\to\infty}a_n = \lim_{n\to\infty}b_n$.
For complex values, we choose the square-root such that $\O{Ph}b_{n+1} = (\O{Ph}a_n + \O{Ph}b_n)/2$ with
the phases of $a_n$,~$b_n$ chosen such that the difference is less than $\pi$.

%%%%%%%%%%%%%%%%%%%%%%%%%%%%%%%%%%%%%%%%%%%%%%%%%%%%%%%%%%%%%%%%%%%%%%%%%%%%%%%%
\section{Gamma and related functions}

What about computation of {\em ratios} of Gamma functions?
(To handle cancellation of poles, etc. nicely; avoid scaling issues, etc.)
$\F{gamma\_ratio}([a_1,...,a_n], [b_1,...,b_m])$

Note that
\[ \lim_{\eps\to0}\frac{\Gam(-n+\eps)}{\Gam(-m+\eps)} = (-)^{m+n}\frac{m!}{n!} \qquad m,n\in\NN \]


%%%%%%%%%%%%%%%%%%%%%%%%%%%%%%%%%%%%%%%%
\subsection{Gamma functions}
\[ \Gam(z) = \int_0^\infty t^{z-1}e^{-t}\,dt \qquad z\in\CC\setminus\NNo  \DEF{$\Gam(z)$}\]
Note that that integral definition only applies for $\Re z>0$, but the function can be extended
to a meromorphic function with simple poles at $z=0,-1,-2,\dots$
\[ \Gam^*(z) = \frac{x^{-x+1/2} e^{x}}{\sqrt{2\pi}} \Gam(z)   \DEF{$\Gam^*(z)$}\]
\[ \frac{1}{\Gam(z+1)} = e^{\gamma z}\prod_{n=1}^\infty e^{-z/n}(1 + z/n) \]

Binet function:
\[ \O{J}(z) = \ln\Gam(z) + z - (z-\frac12)\ln z - \frac12\ln2\pi    \DEF{$\O{J}(z)$}\]

\[ \F{gamma}(z) = \Gam(z) \]
\[ \F{gamma\_p1\_m1}(z) = \Gam(1+z)-1 \]
\[ \F{log\_gamma}(z) = \log(\Gam(z)) \]
\[ \F{log\_abs\_gamma}(z) = \log(\Gam(|z|)) \]
\[ \F{gammastar}(z) = \Gam^*(z) \]
\[ \F{gamma\_inv}(z) = 1/\Gam(z) \]
log-gamma+sign also

We have, for $\nu\in(-1,1)$
\[ \log(\Gam(1+\nu)) = -\gam\nu + \frac12\log(\frac{\pi\nu(1-\nu)}{(1+\nu)\sin(\pi\nu)}) - \sum_{j=3,5,7,\cdots}\frac{\zeta(j)-1}{j}\nu^j\]
but initial testing gives mediocre results.

\[ \log(\Gam(1+\nu)) = -\gam\nu + \sum_{j=2}^\infty\frac{\zeta(j)}{j}(-\nu)^j \qquad \nu\in(-1,1] \]

Amusing fact: $\zeta(3) = \lim_{\eps\to0}\frac{1}{2\eps^3}\frac{\Gam^3(1+\eps)\Gam(1-\eps)}{\Gam(1+\eps)}$ with error~$\sim2\eps$.

Spouge's approximation (a modification of Stirling's):
\[ \Gam(z+1) = (z+a)^{z+1/2}e^{-(z+a)}\left( c_0 + \sum_{k=1}^{a-1}\frac{c_k}{z+k} + \eps_a(z) \right) \]
where $a$ is arbitrary positive integer and $c_0=\sqrt{2\pi}$ with (for $k=1,2,\dots,a-1$):
\[ c_k = \frac{(-)^{k-1}}{(k-1)!}(-k+a)^{k-1/2}e^{-k+a} \]
If $\Re z>0$ and $a>2$, then the relative error in discarding $\eps_a(z)$ is bounded by $a^{-1/2}(2\pi)^{-(a+1/2)}$.
(Note that this has controllable error but the large coefficients have cancellative properties...)
This is a convenient ``work-horse'' algorithm --- it's conveniently modified to directly compute $\Gam^*$ nicely.

The Lanczos approximation (from Wikipedia):
\[ \Gam(z+1) = \sqrt{2\pi}(z+g+\tfrac12)^{z+1/2} e^{-(z+g+1/2)} A_g(z) \]
with $g$ an arbitrary constant (with $\Re(z+g+1/2)>0$) and
\[ A_g(z) = \frac12 p_0(g) + \frac{z}{z+1} p_1(g) + \frac{z(z-1)}{(z+1)(z+2)} p_2(g) + \cdots \]
... valid only for $\Re z>0$ ... (test of implementation from wikipedia page gives bad results, not clear why)

\begin{implementation}
A few useful relations
\begin{itemize}
\item[\bf(A)] $\Gam(z)\Gam(1-z) = \frac{\pi}{\sin\pi z}$; equivalently, $\ln|\Gam(z)| + \ln|\Gam(1-z)| = \ln\pi - \ln|\sin\pi z|$
\item[\bf(B)] $\Gam(z+1) = z\Gam(z)$; equivalently $\ln|\Gam(z+1)| = \ln|z| + \ln|\Gam(z)|$
\item[\bf(C)] Stirling's approximation for large $n$:
  \[n! \sim \sqrt{2\pi n}\left(\frac{n}{e}\right)^n\left(
      1 + \frac{1}{12 n} + \frac{1}{288 n^2} - \frac{139}{51840 n^3} - \frac{571}{2488320 n^4} \cdots\right)\]
\item[\bf(D)] For $|z|<1$ ($<2$?),
  \[ \ln\Gam(1+z) = -\ln(1+z) + z(1-\gamma) + \sum_{k=2}^\infty(-)^k(\zeta(k)-1)\frac{z^k}{k} \]
  Use this for tiny $z$, especially to directly compute $\Gam(1+z)-1$ --- it works beautifully.
\item[\bf(E)] A power-series at zero, $\frac{1}{\Gam(z)} = \sum_{k=1}^\infty c_k z^k$, where $c_1=1$, $c_2=\gamma$,
  and $(k-1)c_k = \gamma c_{k-1} - \zeta(2) c_{k-2} + \zeta(3) c_{k-3} - \cdots + (-)^k \zeta(k-1) c_1$ for $k\geq3$
\end{itemize}
For implementation, we use {\bf(C)} for $x>90$ and {\bf(D)} for $x<2$ and otherwise use {\bf(B)} repeatedly to bring $x<2$.  This seems to
give 15 digits precision for all $x>0$.  Values of $\zeta(k)-1$ are precomputed at high-precision and stored in a table --- 100 values are enough.
For $x<0$ we use {\bf(A)} to transform to a positive value, but naive implementation gives poor results for $-90<x<0$.

Note that the above relations apply for complex $z$ also, so can be the basis for complex $\Gam$ implementation.
\end{implementation}

%%%%%%%%%%%%%%%%%%%%%%%%%%%%%%%%%%%%%%%%
\subsection{Factorial, Pochammer symbol}
\[ n! = \Gam(n+1) \]
\[ (\al)_\nu = \al^{\uparrow \nu} = \frac{\Gam(\al+\nu)}{\Gam(\al)} \]
\[ \al^{\downarrow \nu} = \frac{\Gam(\al)}{\Gam(\al-\nu)} \]
Thus $(\al)_0 = 1$ and $(\al)_{n+1} = (\al+n)\cdot(\al)_n$.
n!!, n!!!, \dots (+logs), log-factorial, rising/falling\_factorial, $x^n/n!$,
ratio of pochhammers $\frac{(a_1)_n\cdots(a_p)_n}{(b_1)_n\cdots(b_q)_n}$,
integer/general pochhammers, etc.

%%%%%%%%%%%%%%%%%%%%%%%%%%%%%%%%%%%%%%%%
\subsection{Incomplete gamma functions}

\[ \gam(\al,z) = \int_0^z t^{\al-1}e^{-t}\,dt   \DEF{$\gam(\al,z)$}\]
\[ \gam^*(\al,z) = \al^{\dots}\gam(\al,z) \]
\[ \O{P}(\al,z) = \frac{\gam(\al,z)}{\Gam(\al)}   \DEF{$\O{P}(\al,z)$}\]
\[ \O{Q}(\al,z) = 1 - \O{P}(\al,z) = \frac{\Gam(\al,z)}{\Gam(\al)}    \DEF{$\O{Q}(\al,z)$}\]
\[ \Gam(\al,z) = \Gam(\al) - \gam(\al,z) = \int_z^\infty t^{\al-1}e^{-t}\,dt \DEF{$\Gam(\al,z)$}\]
For a fixed $z$, $\gamma(\al,z)$ is a meromorphic function of $\al$ with simple poles at $\al=0,-1,-2,\dots$,
while $\Gam(\al,z)$ is an entire function of~$\al$.

Remarks: $\gam(\al,-z)=\int_0^{-z}t^{\al-1}e^{-t}\,dt$, $\Gam(\al,-z)=\int_{-z}^\infty t^{\al-1}e^{-t}\,dt$
\[ \gam(\al,x) = \frac{x^\al}{\al} e^{-x} \Hyper{1}{1}{1}{1+\al}{x} = \frac{x^\al}{\al}\Hyper{1}{1}{\al}{1+\al}{-x} \]
Recalling that $M(a,b,z)=e^{z} M(b-a,b,-z)$ and $\Hyper{1}{1}{1}{1+\al}{x}=e^{x}\Hyper{1}{1}{\al}{1+\al}{-x}$.

Inverse functions ...

\begin{implementation}
Various potentially useful relations
\begin{itemize}
\item[\bf(A)] For $\al\neq-1,-2,\dots$,
\[ \Gam(\al,z) = \Gam(\al) - \sum_{n=0}^\infty\frac{\Gam(\al)e^{-z}z^{n+\al}}{\Gam(\al+n+1)} = \Gam(\al) - \sum_{n=0}^\infty\frac{(-)^nz^{n+\al}}{(\al+n)n!} \]
\item[\bf(B)] As $z\to\infty$,
\[ \Gam(\al,z)\sim z^{\al-1}e^{-z}\sum_{n=0}^\infty\frac{\Gam(\al)}{\Gam(\al-n)z^n} \]
\item[\bf(C)] Continued fraction expansion: (best when $\al>>z$ ?? should this be $z>>\al$ instead??)
\[ \Gam(\al,z) = \frac{e^{-z}z^\al}{z+} \frac{1-\al}{1+} \frac{1}{z+} \frac{2-\al}{1+} \frac{2}{z+} \cdots \]
\item[\bf(C')] Equivalent fraction (converges for $\Re z>0$), where $v=z^{-1}$
\[ e^{z}z^{1-\al}\Gam(\al,z) = \frac{1}{1+} \frac{(1-\al)v}{1+} \frac{v}{1+} \frac{(2-\al)v}{1+} \frac{2v}{1+} \cdots \]
\item[\bf(D)] For $n\geq0$; (though claimed to be of limited use unless $\al\sim-n$)
\[ \Gam(\al+n+1, z) = \Gam(\al+n, z) + \frac{\Gam(\al)e^{-z}z^{n+\al}}{\Gam(\al+n+1)} \]
\end{itemize}
\end{implementation}

%%%%%%%%%%%%%%%%%%%%%%%%%%%%%%%%%%%%%%%%
\subsection{Digamma functions and related}
The digamma function is defined via
\[ \psi(z) = \frac{d}{dz}\log\Gam(z) = \frac{\Gam'(z)}{\Gam(z)}   \DEF{$\psi(z)$}\]
meromorphic with simple poles at $z=0,-1,-2,\dots$

Series expansion:
\[ \psi(z) = -\gam - \frac1z + \sum_{n=1}^\infty\frac{z}{n(n+z)} = -\gam+\sum_{n=0}^\infty\left(\frac{1}{n+1}-\frac{1}{n+z}\right) \]

Asymptotic:
\[ \psi(z) \sim \log(z) - \frac{1}{2z} - \sum_{m=1}^\infty \frac{B_{2m}}{2m} z^{-2m} \qquad\text{as $z\to\infty$, $|\arg z|<\pi$} \]

Notes on Euler-Maclaurin expansion:
\[ \sum_{n=a}^\infty \frac{z}{n(n+z)} \approx \log(\frac{a+z}{z}) - \frac{z}{z(a)(a+z)} - \sum_{j=2}^k \frac{B_j}{j!} f^{(j-1)}(a) \]
where we note that $f^{(k)}(n) = (-1)(-2)\cdots(-k)(n^{-k-1}-(n+z)^{-k-1})$.  This expansion works well in testing.

And the polygamma function (for $n=1,2,3,\dots$)
\[ \psi^{(n)}(z) = \frac{d^n}{dz^n}\psi(z) = \frac{d^{n+1}}{dz^{n+1}}\ln\Gam(z)
    = (-)^{n+1}\int_0^\infty\frac{t^n e^{-zt}}{1-e^{-t}}\,dt \DEF{$\psi^{(n)}(z)$}\]
which is meromorphic and single-valued with poles of order $n+1$ at $z=-m$ ($m=0,1,2,\dots$).
\[ \psi^{(n)}(z+1) = \psi^{(n)}(z) + (-)^n n! z^{-n-1} \]
\[ \psi^{(n)}(1-z) - (-)^n\psi^{(n)}(z) = (-)^n \pi \frac{d^n}{dz^n}\cot(\pi z) \]
\[ \psi^{(n)}(1+z) = (-)^{n+1}\sum_{k=n}^\infty (-)^k k!\zeta(k+1)\frac{z^{k-n}}{(k-n)!} \]
\[ \psi^{(n)}(z) = (-)^{n+1} n! \sum_{k=0}^\infty (z+k)^{-n-1} \qquad(z\neq0,-1,-2,\dots) \]
Using Euler-Maclaurin summation with this last series works quite well.

\begin{implementation}
This approach for digamma works pretty well, but get loss of precision as $x\to^-2$ or $x\to^+0$.
\begin{itemize}
\item[\bf(A)] Use the following to reduce the (real) argument to get $0.5<z\leq1.5$
    \[ \psi(z+1) = \frac{1}{z} + \psi(z) \]
\item[\bf(B)] Then use this series for the resulting $z$.
    \[ \psi(1+z) = -\gamma + \sum_{k=2}^\infty (-)^k \zeta(k) z^{k-1} \]
\end{itemize}
\end{implementation}

Remark: for reflection of polygamma function, we need $\frac{d^n}{dz^n}\cot(\pi z)$.  From Mathematica we have the following table:
perhaps some pattern can be gleaned for arbitrary $n$... [TODO: solve this puzzle]
{\small
\[\begin{array}{r|l}
n & \frac{(-)^n}{\csc(x)^{n+1}}\frac{d^n}{dx^n}\cot(x) \\\hline
 0 & \cos (x) \\
 1 & 1 \\
 2 & 2 \cos (x) \\
 3 & 2 \cos (2 x)+4 \\
 4 & 22 \cos (x)+2 \cos (3 x) \\
 5 & 52 \cos (2 x)+2 \cos (4 x)+66 \\
 6 & 604 \cos (x)+114 \cos (3 x)+2 \cos (5 x) \\
 7 & 2382 \cos (2 x)+240 \cos (4 x)+2 \cos (6 x)+2416 \\
 8 & 31238 \cos (x)+8586 \cos (3 x)+494 \cos (5 x)+2 \cos (7 x) \\
 9 & 176468 \cos (2 x)+29216 \cos (4 x)+1004 \cos (6 x)+2 \cos (8 x)+156190 \\
 10 & 2620708 \cos (x)+910384 \cos (3 x)+95680 \cos (5 x)+2026 \cos (7 x)+2 \cos (9 x) \\
 11 & 19476228 \cos (2 x)+4406976 \cos (4 x)+305274 \cos (6 x)+4072 \cos (8 x)+2 \cos (10 x)+15724248 \\
 12 & 325024572 \cos (x)+132636948 \cos (3 x)+20375370 \cos (5 x)+956542 \cos (7 x)+8166 \cos (9 x)+2 \cos (11 x) \\
\end{array}\]
}

Remarks on polygamma function
\begin{itemize}
  \item Definition, $k=1,2,\dots$, $z\neq0,-1,-2,\dots$
    \[ \psi^{(m)}(z) = (-)^{m+1} m! \sum_{k=0}^\infty \frac{1}{(z+k)^{m+1}} \]
  \item Euler-Maclaurin:
    \[ \sum_{k=a}^\infty \frac{1}{(z+k)^{m+1}} = \frac{(z+a)^{-m}}{m} - \frac{(z+a)^{-m-1}}{2}
        - \sum_{j=2}^\infty \frac{B_j}{j!}(z+a)^{-m-j}(-m-1)(-m-2)\cdots(-m-(j-1)) \]
  \item Asymptotic as $z\to\infty$ for $|\arg z|<\pi$, $k\geq1$,
    \[ \psi^{(m)}(z) \sim (-)^{m-1}\left(\frac{(m-1)!}{z^m} + \frac{m!}{2 z^{m+1}} + z^{-2}\sum_{k=0}^\infty \frac{B_{2k+2}}{(2k+2)!}(2k+m+1)! z^{-(2k+m)} \right)\]
  \item Reflection, $k=1,2,3,\dots$
    \[ \psi^{(m)}(1-z) + (-)^{m+1}\psi^{(m)}(z) = (-)^m \pi \frac{d^k}{dz^k} \cot(\pi z) \]
\end{itemize}

%%%%%%%%%%%%%%%%%%%%%%%%%%%%%%%%%%%%%%%%
\subsection{Beta functions}

\[ \Beta(a,b) = \int_0^1 t^{a-1}(1-t)^{b-1}\,dt = \frac{\Gam(a) \Gam(b)}{\Gam(a+b)}     \DEF{$\Beta(a,b)$}\]
{\tt beta}

%%%%%%%%%%%%%%%%%%%%%%%%%%%%%%%%%%%%%%%%
\subsection{Incomplete beta functions}

\[ \Beta_\xi(a,b) = \int_0^\xi t^{a-1}(1-t)^{b-1}\,dt     \DEF{$\Beta_\xi(a,b)$}\]
\[ \O{I}_\xi(a,b) = \frac{\Beta_\xi(a,b)}{\Beta(a,b)}     \DEF{$\O{I}_\xi(a,b)$}\]
+complementary incomplete beta (\& scaled)

Gil-Segura-Temme book gives the continued fraction
\[ \Beta_x(p,q) = \frac{x^p(1-x)^q}{p}\left( \frac{1}{1+} \frac{d_1}{1+} \frac{d_2}{1+} \frac{d_3}{1+} \cdots \right) \]
where
\[ d_{2n+1} = -x\frac{(p+n)(p+q+n)}{(p+2n)(p+2n+1)} \qquad d_{2n+2} = x\frac{(n+1)(q-n-1)}{(p+2n+1)(p+2n+2)} \]
with the note that when $p,q>1$ the best numerical results are obtained when $x\leq x_0 = (p-1)/(p+q-2)$ and for $x_0<x\leq1$
you should use the reflection $\Beta_x(p,q) = \Beta(p,q) - \Beta_{1-x}(q,p)$.
Note also that for this continued fraction,
the convergents $C_{4n}$ and $C_{4n+1}$ are less than
and the convergents $C_{4n+2}$ and $C_{4n+3}$ are greater than
the value of the continued fraction.
[This works quite well (with a few domain issues for complex/negative parameters).]

%%%%%%%%%%%%%%%%%%%%%%%%%%%%%%%%%%%%%%%%%%%%%%%%%%%%%%%%%%%%%%%%%%%%%%%%%%%%%%%%
\section{Error and related functions}
Dawson's integral, Fadeeva function (zeros?) [normal cdf, etc.]
Black-Scholes/Implied-Volatility core: $e^\al\Phi[\al/s + s/2] - \Phi[\al/s - s/2]$,
{\tt erf, erfc, inverse\_erf, inverse\_erfc, erfc\_star, Dawson, Fadeeva},

Iterated error functions $i^n = \int i^{n-1}\erfc$

\[ \erf(z) = \frac{2}{\sqrt\pi}\int_0^z e^{-t^2}\,dt \qquad z\in\CC \]
\[ \erfc(z) = 1 - \erf(z) \qquad z\in\CC \]
\[ \erfc^*(z) = e^{z^2} \erfc(z) \]
\[ \O{erfi}(z) = -\ii\erf(\ii z) \]
(useful for $z\to\infty$, $z>0$)

\[ \F{erf}(z) = \erf(z) \]
\[ \F{erfc}(z) = \erfc(z) \]

An approximation from A\&S is 
\[ \O{inverf}(x) = t + \frac{1}{3}t^3 + \frac{7}{30}t^5 + \frac{127}{630}t^7 + \cdots \]
where $t=x\sqrt{\pi}/2$.  This can give the first guess to an iterative root-finder (such as Halley's method).

Note that Lebedev defines $\Phi(z)=\text{our $\erf$}$ and $\O{Erf}(z)=\int_0^z e^{-t^2}\,dt$.
He also defines $F(z) = e^{-z^2}\int_0^z e^{u^2}\,du$.

\[ \erf(\ii x) = \frac{2x\ii}{\sqrt\pi} \sum_{n=0}^\infty (-)^n \left( \O{i}^{(1)}_{2n}(x^2) + \O{i}^{(1)}_{2n+1}(x^2) \right) \qquad x\in\RR \]

\begin{implementation}
\begin{itemize}
\item A series that works reasonably well for small $z$ is the following; note that we ensure $z\geq0$ by using
  the fact that $\erf(-z)=-\erf(z)$ (to avoid cancellation)
  \[ \erf(z) = \frac{2}{\sqrt\pi}e^{-z^2}\sum_{n=0}^\infty\frac{2^n z^{2n+1}}{1\cdot3\cdot5\cdots(2n+1)} \]
\item A continued fraction that works well the following (evaluated with, say, the Wallis recurrence)
  \[ \erfc(z) = \frac{e^{-z^2}}{\sqrt\pi} \frac{2z}{z^2+1-} \frac{1\cdot2}{z^2+5-} \frac{3\cdot4}{z^2+9-} \frac{5\cdot6}{z^2+13-}\cdots \]
\item An asymptotic expansion that generally underperforms the continued fraction is
  \[ \erfc(z) \sim \frac{e^{-z^2}}{z\sqrt\pi}\sum_{m=0}^\infty(-)^m\frac{1\cdot3\cdot5\dots(2m-1)}{(2z^2)^m} \]
\item Marsaglia~\cite{marsaglia} points out (for the normal cdf) that given the value of $\erfc(x_0)$ at a point $x_0$, there is a simple recurrence
  relation to generate the coefficients of the Taylor series around that point, that is:
  \[ \erf^{(n+2)}(z) = -2z\erf^{(n+1)}(z) - 2n\erf^{(n)}(z) \qquad n\geq0 \]
\item A reasonable approach is to use the continued fraction for $|x|>3$, the series for $|x|<1$, and the local Taylor expansions
  centered at $\pm1\tfrac12, \pm2, \pm2\tfrac12$ for other points.  (Using a table to store the ``seeds'' for those select points.)
\end{itemize}
\end{implementation}

Other notes:
\[ erf(x+\ii y) = \erf(x) + \frac{e^{-x^2}}{2\pi x}\left[(1-\cos(2xy)) + \ii\sin(2xy)\right]
    +\frac2\pi e^{-x^2}\sum_{n=1}^\infty \frac{e^{-n^2/4}}{n^2+4x^2}\left[ f_n(x,y) + \ii g_n(x,y)\right] + \eps \]
with $\eps\sim 10^{-16}\cdot|erf(z)|$ where
$f_n(x,y) = 2x - 2x\cosh(ny)\cos(2xy) + n\sinh(ny)\sin(2xy)$ and
$g_n(x,y) = 2x\cosh(ny)\sin(2xy) + n\sinh(ny)\cos(2xy)$.
This seems to work pretty well for small(ish) $z$, but gives NaN for, say, $z=17(1+\ii)$.  (Note that $x=0$ is bad, though
this case reduces to...)


%%%%%%%%%%%%%%%%%%%%%%%%%%%%%%%%%%%%%%%%
\subsection{Fadeeva and Dawson integrals}
Dawson's integral:
\[ \O{F}(x) = e^{-x^2}\int_0^x e^{t^2}\,dt = -\ii\frac{\sqrt\pi}{2}e^{-z^2}\erf(\ii z) = \ii\frac{\sqrt{\pi}}{2}\left(e^{-z^2} -\O{w}(z)\right)\]
Generalized Dawson's integral:
\[ \O{F}_p(x) = e^{-x^p}\int_0^x e^{t^p}\,dt \]

Fadeeva's integral:
\[ \O{w}(z) = e^{-z^2}\erfc(-\ii z) = e^{-z^2}(1 + \frac{2\ii}{\sqrt\pi}\int_0^z e^{t^2}\,dt \]

Series:
\begin{eqnarray*}
\O{F}(x) &=& x\frac{\sqrt{\pi}}{2}\sum_{k=0}^\infty(-)^k\frac{x^{2k}}{\Gam(k+3/2)} \\
         &=& x\sum_{k=0}^\infty (-)^k \frac{x^{2k}}{(3/2)_k} \\
         &=& \sum_{n=0}^\infty (-)^n \frac{2^n x^{2n+1}}{(2n+1)!!}
\end{eqnarray*}

Series of spherical Bessel functions
\begin{eqnarray*}
\O{F}(x) &=& e^{-x^2} x \sum_{n=0}^\infty (-)^n \left( \O{i}^{(1)}_{2n}(x^2) + \O{i}^{(1)}_{2n+1}(x^2) \right) \\
         &=& e^{-x^2} x \left( \O{i}_0 + \O{i}_1 - \O{i}_2 - \O{i}_3 + \O{i}_4 + \O{i}_5 - \cdots \right) \\
         &=& e^{-x^2} x \left( \frac{3}{z}\O{i}_1 + \frac{5}{z}\O{i}_2 + \frac{11}{z}\O{i}_5 + \frac{13}{z}\O{i}_6 + \frac{19}{z}\O{i}_9 + \frac{21}{z}\O{i}_{10} + \cdots \right)
\end{eqnarray*}
where we use the recurrence relation $\O{i}_{n-1} - \O{i}_{n+1} = \tfrac{2n+1}{z}\O{i}_n$ for the last equation.
Unfortunately, this approach is not numerically stable and we lose much accuracy from bad cancellation.

Continued fractions:
\[ F(z) = \frac{z}{1+} \frac{2z^2}{3-} \frac{4z^2}{5+} \frac{6z^2}{7-} \frac{8z^2}{9+} \cdots \]
which is good for $|z|$ smaller.
\[ F(z) = \frac{z}{1+2z^2-} \frac{4z^2}{3+2z^2-} \frac{8z^2}{5+2z^2-} \frac{12z^2}{7+2z^2-} \cdots \]
which is good for $|z|$ larger.


%%%%%%%%%%%%%%%%%%%%%%%%%%%%%%%%%%%%%%%%
\subsection{Fresnel integrals}
Need {\em zeros} also
\[ \O{C}(z) = \int_0^z \cos(t^2 \pi/2)\,dt \]
\[ \O{S}(z) = \int_0^z \sin(t^2 \pi/2)\,dt \]
{\tt Fresnel\_C, Fresnel\_S}

We have the series expansions
\[ \O{C}(z) = \sum_{n=0}^\infty(-)^n\frac{(\pi/2)^{2n}}{(2n)!(4n+1)}z^{4n+1}
    = \sum_{n=0}^\infty \O{j}_{2n}(z^2\pi/2) \]
\[ \O{S}(z) = \sum_{n=0}^\infty(-)^n\frac{(\pi/2)^{2n+1}}{(2n+1)!(4n+3)}z^{4n+3}
    = \sum_{n=0}^\infty \O{j}_{2n+1}(z^2\pi/2) \]
(for using the spherical Bessel expansions, should extract the whole sequence of needed values for efficiency...)
Note that DLMF and A\&S include a $z$ multiplier on the Bessel expansions but numeric tests seem to indicate that that
is wrong!
\begin{implementation}
Use the two series above as well as asymptotic expansion...
\end{implementation}

``modified Fresnel? (see Z \& J)''

Note: in Lebedev's notation, we have
$\Phi(\sqrt\ii x)/\sqrt\ii = \frac{2}{\sqrt\pi}\int_0^x\cos u^2\,du -  \ii\frac{2}{\sqrt\pi}\int_0^x\sin u^2\,du$

\[ \O{Ci}(z) = \int_\infty^z\frac{\cos t}{t}\,dt = \gam + \log(z) + \int_0^z \frac{\cos t - 1}{t}\,dt \]
\[ \O{Si}(z) = \int_0^z \frac{\sin t}{t}\,dt \]
$\O{Si}(z)$ is an entire function, while $\O{Ci}(z)$ is analytic in the split plane (deleting the negative real axis).
Note that $\O{Ei}(\ii x) = \O{Ci}(x) - \ii(\frac\pi2 - \O{Si}(x))$ for $x>0$.

When $|\O{ph} z| < \pi/2$, we have
\[ \O{Ci}(z) = -\frac12\left( \O{E}_1(\ii z) + \O{E}_1(-\ii z) \right) \]
\[ \O{Si}(z) = \frac\ii2\left( \O{E}_1(-\ii z) - \O{E}_1(\ii z) \right) + \frac\pi2 \]
Using these with continued-fraction expansion for $\O{E}_1(z) = e^{-z}/(z+1/(1+1/(z+2/(1+2/(z+\cdots)))))$ works well.
(Note that for real $x$, we have $\O{Ci}(x) = -\Re \O{E}_1(\ii x)$.)

[{\bf Move much of this to Exponential Integrals section}]

\[ \O{Cin}(z) = \int_0^z\frac{1-\cos t}{t}\,dt \]
thus $\O{Ci}(z) = -\O{Cin}(z) + \ln z + \gam$

Series expansions:
\[ \O{Ci}(z) = \gamma + \ln(z) + \sum_{n=1}^\infty (-)^n \frac{z^{2n}}{(2n)!(2n)} \]
\[ \O{Si}(z) = \sum_{n=0}^\infty (-)^n \frac{z^{2n+1}}{(2n+1)!(2n+1)} \]

\[ \O{Si}(z) = z\sum_{n=0}^\infty \O{j}^2_n(z/2) \]
which works well (and can use recurrence for fast computation...)
\[ \O{Ci}(z) = \sum_{n=0}^\infty a_n\O{j}^2_n(z/2) \]
where $a_n = (2n+1)\left(1 - (-)^n + \psi(n+1) - \psi(1)\right)$
\[ \O{Ei}(x) = \gam + \ln|x| + \sum_{n=0}^\infty (-)^n (x-a_n) \O{i}^{(1)}(x/2)^2 \qquad (x\neq 0) \]
where $a_n = (2n+1)\left(1 - (-)^n + \psi(n+1) - \psi(1)\right)$
\[ \O{Ein}(x) = z e^{-z/2}\left( \O{i}^{(1)}_0(z/2) + \sum_{n=1}^\infty \frac{2n+1}{n(n+1)} \O{i}^{(1)}_n(z/2) \right) \]
Also, $\O{E}_1(z) = \Gam(0,z)$ and $\O{E}_1(z) = e^{-z} U(1,1,z)$

Asymptotic expansions ($|z|=25$ seems to be a reasonable transition point from the series to asymptotic)
\begin{eqnarray*}
\O{Si}(z) &=& \frac\pi2 - \cos(z) f(z) - \sin(z) g(z) \\
\O{Ci}(z) &=& \sin(z) f(z) - \cos(z) g(z) \\
f(z) &\sim& \frac1z\left(1 - \frac{2!}{z^2} + \frac{4!}{z^4} - \frac{6!}{z^6} + \cdots \right) \\
g(z) &\sim& \frac1{z^2}\left(1 - \frac{3!}{z^2} + \frac{5!}{z^4} - \frac{7!}{z^6} + \cdots \right)
\end{eqnarray*}

For $x>0$ we have the following (actually for any $z$ in the slit plane $\CC\setminus(-\infty,0]$):
\[ \O{Shi}(x) = (\O{Ei}(z) + \O{E}_1(x))/2 \]
\[ \O{Chi}(x) = (\O{Ei}(z) - \O{E}_1(x))/2 \]

We have
\[ \O{li}(z) = \O{Ei}(\ln z) \]

Asymptotic expansion for Fresnel integrals
\begin{eqnarray*}
\O{C}(x) &=& \O{sgn}(x)\frac12 + \frac{1}{\pi x}\left( f(x)\sin(x^2\pi/2) - g(x)\cos(x^2\pi/2) \right) \\
\O{S}(x) &=& \O{sgn}(x)\frac12 - \frac{1}{\pi x}\left( f(x)\cos(x^2\pi/2) + g(x)\sin(x^2\pi/2) \right) \\
\end{eqnarray*}
where
\begin{eqnarray*}
f(x) &\sim& 1 + \sum_{n=1}^\infty (-)^n \frac{1\cdot3\cdot5\cdots(4n-1)}{(\pi x^2)^{2n  }} \\
g(x) &\sim&     \sum_{n=0}^\infty (-)^n \frac{1\cdot3\cdot5\cdots(4n+1)}{(\pi x^2)^{2n+1}} \\
\end{eqnarray*}
[quality of this needs to be validated...]

%%%%%%%%%%%%%%%%%%%%
\subsubsection{Arctangent integral}

\[ \O{atanint}(z) = \int_0^z\frac{\atan(t)}{t}\,dt \]
then since $\atan(z)=z-z^3/3+z^5/5-\cdots$ (for $|z|\leq1$, $z\neq\pm\ii$), we have
\[ \O{atanint}(z) = \sum_{n=0}^\infty(-)^n\frac{z^{2n+1}}{(2n+1)^2} \]
(compare dilogarithm...)

In terms of the Lerch $\Phi$ function we have
\[ \O{atanint}(z) = \frac{z}{4}\sum_{n=0}^\infty\frac{(-z^2)^n}{(n+1/2)^2} = \frac{z}{4}\Phi[-z^2, 2, 1/2] \]

We have the asymptotic series
\[ \O{atanint}(z) = \frac\pi2\ln z + \sum_{n=0}^\infty \frac{(-)^n}{z^{2n+1}(2n+1)^2} \qquad|z|\geq1, \Re z\neq0\]
(from similar series for $\atan$)

We can expand in Taylor series at $1$.  We get
\begin{eqnarray*}
  \O{ataint}(1+x) &=& C + \frac\pi4 x + \left(\frac12-\frac\pi4\right) \frac{x^2}{2!} + \left(\frac\pi4-\frac32\right) \frac{x^3}{3!} + \left(5-\frac{3\pi}2\right) \frac{x^4}{4!} \\
      && {} + \left(6\pi-20\right) \frac{x^5}{5!} + \left(97-30\pi\right) \frac{x^6}{6!} + \left(180\pi-567\right) \frac{x^7}{7!} + \left(3924-1260\pi\right) \frac{x^8}{8!} + \cdots
\end{eqnarray*}
where $C=0.9159655942\dots$ is Catalan's constant.

One can attempt Euler-Maclaurin expansion in the series, but you end up with $\Gam(-1,x)$ term for the integral (which can be written in terms of $\O{Ei}(x)$), but the
computation becomes more expensive... (integral term in EM is $\tfrac{-\ln(z)}{4z}\Gam(-1, (4n+1)\ln z)$
and we have $\Gam(-1,x) = \O{Ei}(-x) + \frac{e^{-x}}{x} + \frac12(\ln(-1/x) - \ln(-x)) + \ln x$.)

\begin{implementation}
To implement for real $x$, use series, asymptotic series, reflection, and local series at~$1$.
\end{implementation}


%%%%%%%%%%%%%%%%%%%%%%%%%%%%%%%%%%%%%%%%
\subsection{Iterated error functions}
Define
\begin{eqnarray*}
i^{-1}\erfc(z) &=& \frac{2}{\sqrt\pi} e^{-z^2} \\
i^{0}\erfc(z) &=& \erfc(z) \\
i^{n}\erfc(z) &=& \int_z^\infty i^{n-1}\erfc(t)\,dt
\end{eqnarray*}
We have the recurrence
\[ i^{n}\erfc(z) = -\frac{z}{n}i^{n-1}\erfc(z) + \frac{1}{2n}i^{n-2}\erfc(z) \]

%%%%%%%%%%%%%%%%%%%%%%%%%%%%%%%%%%%%%%%%%%%%%%%%%%%%%%%%%%%%%%%%%%%%%%%%%%%%%%%%
\section{Bessel, Hankel, Airy functions}

Need {\em zeros} also

{\tt Jn, Jnu, Yn, Ynu, In, Inu, Kn, Knu, dJn, ..., Integrate\_Jn, ...
jn, yn, in, kn, ..., (array versions...)
} --- real / imaginary / complex arguments, integer / real / imaginary / complex parameters
$\Lambda_\nu(x) = $?, ``Riccati Bessel functions''?

%%%%%%%%%%%%%%%%%%%%%%%%%%%%%%%%%%%%%%%%
\subsection{Bessel functions}
Bessel functions are also known as ``cylinder functions''

Bessel's differential equation $z^2 y'' + z y' + (z^2-\nu^2) y = 0$.

Bessel function of the first kind, regular at~$0$; if $\nu\in\ZZ$ then we can have any $z\in\CC$, but
for arbitrary $\nu\in\CC$ we need $|\arg z|<\pi$.
\[ \O{J}_\nu(z) = \sum_{k=0}^\infty\frac{(z/2)^{2k+\nu}}{k!(k+\nu)!} \]
Note that
\[ \O{J}_\nu(z) = \frac{1}{\pi}\int_0^\pi\cos(z \sin\theta - \nu\theta)\,d\theta
  - \frac{\sin \nu\pi}{\pi}\int_0^\infty e^{-z \sinh t - \nu t}\,dt \]

Bessel function of the second kind, irregular at~$0$:
\[ \O{Y}_\nu(z) = \frac{\O{J}_\nu(z)\cos\nu\pi - \O{J}_{-\nu}(z)}{\sin \nu\pi} \]
Some people use the notation $\O{N}_\nu(z) = \O{Y}_\nu(z)$.
For $\nu=n\in\ZZ$, define $\O{Y}_n(z) = \lim_{\nu\to n}\O{Y}_\nu(z)$.

%%%%%%%%%%%%%%%%%%%%%%%%%%%%%%%%%%%%%%%%
\subsection{Modified Bessel functions}

Hyperbolic or modified Bessel function (of the first kind), regular at~$0$:
\[ \O{I}_\nu(z) = e^{-\nu\pi\ii/2} \O{J}_\nu(\ii z) \qquad z\in\CC, \nu\in\CC \]
Hyperbolic or modified Bessel function (of the third kind), irregular at~$0$, also called Bassett's function or Macdonald's function:
\[ \O{K}_\nu(z) = xxx \qquad z\in\CC\setminus\RR_{\leq0}, \nu\in\CC \]

Notes sketching one approach for computation of $I_m(z)$:
\[ I_n(x) = \frac{1}{\pi} \int_0^\pi e^{z\cos\theta} \cos(n\theta)\,d\theta \]
\[ e^{x} = \frac12 c_0 + \sum_{k=1}^\infty c_k T_k(x) \qquad x\in(-1,1), c_k = I_k(1) \]
\[ I_n(z) \sim \frac{1}{\sqrt{2\pi n}}\left(\frac{ez}{2n}\right)^n \qquad\text{as $n\to\infty$, fixed $z$} \]
\[ I_{n-1}(z) - I_{n+1}(z) = \frac{2n}{z} I_n(z) \qquad\text{(15 steps enough)} \]
\[ \left(\frac{z}{2}\right)^{\nu} = \sum_{k=0}^\infty(-)^k \frac{(\nu+2k)\Gam(\nu+k)}{k!} I_{\nu+2k}(z) \]
\[ e^{z(t+1/t)/2} = \sum_{m=-\infty}^\infty t^m I_m(z) \]
\[ \text{therefore\ } e^1 = \sum_{m=-\infty}^\infty I_m(z) \]
\[ T_0(x) = 1, T_1(x) = x, T_{n+1}(x) = (2-\delta_{n,0})x T_{n}(x) - T_{n-1}(x) \]

Misc. notes on $I_\nu(z)$:
\[ I_\nu(z) = \left(\frac{z}{2}\right)^\nu \sum_{k=0}^\infty \frac{(z/2)^{2k}}{k! \Gam(\nu+k+1)} \]
\[ I_\nu(0) = \begin{cases} 1 & \nu=0\\ 0 & \nu\in\ZZ, n\neq0\\ 0 & \nu>0,\nu\notin\ZZ\\ \infty & \nu<0,\nu\notin\ZZ\end{cases} \]
\[ I_{-n}(z) = I_{n}(z) \qquad n\in\ZZ \]
\[ I_{-\nu}(z) = I_{\nu}(z) + \frac{2}{\pi} \sin(\nu\pi) K_{\nu}(z) \qquad \nu\notin\ZZ \]

%%%%%%%%%%%%%%%%%%%%%%%%%%%%%%%%%%%%%%%%
\subsection{Incomplete Bessel functions}

\[ \O{I}_\nu(z; \xi) = \dots \]

%%%%%%%%%%%%%%%%%%%%%%%%%%%%%%%%%%%%%%%%
\subsection{Spherical Bessel functions}

\[ \O{j}_n(z) = \sqrt{\frac{\pi}{2z}} \O{J}_{n+1/2}(z) \]
\[ \O{y}_n(z) = \sqrt{\frac{\pi}{2z}} \O{Y}_{n+1/2}(z) \]
\[ \O{i}^{(1)}_n(z) = \]
\[ \O{i}^{(2)}_n(z) = \]
\[ \O{k}_n(z) = \]
\[ \O{h}^{(1)}_n(z) = \]
\[ \O{h}^{(2)}_n(z) = \]

Some explicit formul\ae\ for spherical Bessel functions:
let $a_k(n+\tfrac12) = \begin{cases}\frac{(n+k)!}{2^k k! (n-k)!} & k=0,1,\dots,n\\0&k>n\end{cases}$
\[ \O{j}_n(z) = \sin(z - n\frac\pi2)\sum_{k=0}^{\lfloor n/2\rfloor}(-)^k\frac{a_{2k}(n+\tfrac12)}{z^{2k+1}}
    + \cos(z - n\frac\pi2)\sum_{k=0}^{\lfloor (n-1)/2\rfloor}(-)^k\frac{a_{2k+1}(n+\tfrac12)}{z^{2k+2}} \]
(The Bessel polynomial of degree $n$ is given by $\sum_{k=0}^n a_k(n+\tfrac12)z^{n-k}$.)
\begin{eqnarray*}
\O{j}_0(z) &=& \frac{\sin z}{z} \\
\O{j}_1(z) &=& \frac{\sin z}{z^2} - \frac{\cos z}{z} = \sum_{n=0}^\infty (-)^{n+1}\frac{z^{2n+1}}{(2n+1)!(2n+3)} \\
\O{y}_0(z) &=& -\frac{\cos z}{z} \\
\O{y}_1(z) &=& -\frac{\cos z}{z^2} - \frac{\sin z}{z} \\
\O{k}_0(z) &=& \frac{\pi}{2} e^{-z} \frac{1}{z} \\
\O{k}_1(z) &=& \frac{\pi}{2} e^{-z} \left( \frac{1}{z} + \frac{1}{z^2} \right) \\
\O{i}^{(1)}_0(z) &=& \frac{\sinh z}{z} = \sum_{n=0}^\infty\frac{z^{2n}}{(2n+1)!} \\
\O{i}^{(1)}_1(z) &=& -\frac{\sinh z}{z^2} + \frac{\cosh z}{z} = \frac{1}{z}\sum_{n=0}^\infty\frac{2n}{(2n+1)!}z^{2n} \\
\O{i}^{(2)}_0(z) &=& \frac{\cosh z}{z} \\
\O{i}^{(2)}_1(z) &=& -\frac{\cosh z}{z^2} + \frac{\sinh z}{z}
\end{eqnarray*}

Reflection relations:
\[ \O{j}_n(-z) = (-)^n\O{j}_n(z) \]
\[ \O{y}_n(-z) = (-)^{n+1}\O{y}_n(z) \]
\[ \O{i}^{(1)}_n(-z) = (-)^n\O{i}^{(1)}_n(z) \]
\[ \O{i}^{(2)}_n(-z) = (-)^{n+1}\O{i}^{(2)}_n(z) \]
\[ \O{k}_n(-z) = -\frac{\pi}{2}\left( \O{i}^{(1)}_n(-z) \O{i}^{(2)}_n(-z) \right) \]

Recurrence relations:
Let $f_n(z)$ denote $\O{j}_n(z)$, $\O{y}_n(z)$, or $\O{h}^{(j)}_n(z)$, then
\[ f_{n-1}(z) + f_{n+1}(z) = \frac{2n+1}{z} f_n(z) \]
and
\[ f'_n(z) = f_{n-1}(z) - \frac{n+1}{z} f_n(z) \]
For $\O{y}_n$, use forward.  For $\O{j}_n$, use forward when $|z|>n$ and backward when $|z|<n$.

Let $g_n(z)$ denote $(-)^n\O{k}_n(z)$, $\O{i}^{(j)}_n(z)$, then
\[ g_{n-1}(z) - g_{n+1}(z) = \frac{2n+1}{z} g_n(z) \]
and
\[ g'_n(z) = g_{n-1}(z) - \frac{n+1}{z} g_n(z) \]
For $\O{k}_n$, use forward.  For $\O{i}^{(2)}_n$, use forward.  For $\O{i}^{(1)}$, use forward when $|z|>n$ and backward when $|z|<n$.

For scaling in recurrence, can use:
\[ \sum_{n=0}^\infty (2n+1)\O{j}_n^2(z) = 1 \]
(or can use, for example, $\O{j}_0(z)=\sin z / z$.)

%Additional notes: for $\O{j}_n(z) = \sqrt{\pi/(2z)} \O{J}_{n+1/2}(z)$, we have
%\[ \O{j}_n(z) = \sin(z-n\frac{\pi}{2})\sum_{k=0}^{\lfloor n/2 \rfloor}(-)^k \frac{a_{2k}(n+1/2)}{z^{2k+1}}
%            + \cos(z-n\frac{\pi}{2})\sum_{k=0}^{\lfloor(n-1)/2\rfloor}(-)^k\frac{a_{2k+1}(n+1/2)}{z^{2k+2}} \]
%where
%\[ a_k(n+1/2) = \begin{cases} \frac{(n+k)!}{2^k k! (n-k)!} & k=0,1,2,\dots,n \\ 0 & k=n+1,n+2,\dots \end{cases} \]
Perhaps it makes sense to define
\[ \widetilde{\O{J}}_{n+1/2}(z) = z^{n+1/2} \O{J}_{n+1/2}(z) \]
Then
\[ \widetilde{\O{J}}_{n+1/2}(z) = \sqrt{\frac2\pi}\sin(z-n\frac{\pi}{2})\sum_{k=0}^{\lfloor n/2 \rfloor}(-)^k a_{2k}(n+1/2)z^{n-2k}
            + \sqrt{\frac2\pi}\cos(z-n\frac{\pi}{2})\sum_{k=0}^{\lfloor(n-1)/2\rfloor}(-)^k a_{2k+1}(n+1/2)z^{n-2k-1} \]

%%%%%%%%%%%%%%%%%%%%%%%%%%%%%%%%%%%%%%%%
\subsection{Integral Bessel functions}

\[ \O{Ji}_\nu(z) = \int_0^z \O{J}_\nu(t)\,dt \]
\[ \O{Ii}_\nu(z) = \int_0^z \O{I}_\nu(t)\,dt \]
etc.

Note that Lebedev defines $\O{Ji}_\nu(z) = \int_0^z\frac{\O{J}_\nu(t)}{t}\,dt$.

%%%%%%%%%%%%%%%%%%%%%%%%%%%%%%%%%%%%%%%%
\subsection{Hankel, (Whittaker?)}

Hankel functions are also known as Bessel functions of the third kind:
\[ \O{H}^{(1)}_\nu(z) = \O{J}_\nu(z) + \ii\O{Y}_\nu(z) \]
\[ \O{H}^{(2)}_\nu(z) = \O{J}_\nu(z) - \ii\O{Y}_\nu(z) \]

%%%%%%%%%%%%%%%%%%%%%%%%%%%%%%%%%%%%%%%%
\subsection{Airy functions}
Solutions of Airy DE $y'' - z y = 0$.
[Need derivatives also]

\[ \O{Ai}(z) = \frac1\pi \int_0^\infty \cos(t^3/3 + xt)\,dt
  = \frac{\sqrt{z/3}}{\pi}K_{1/3}(\zeta)
  = \frac{\sqrt{z}}{3}\left( \O{I}_{-1/3}(\zeta) - \O{I}_{1/3}(\zeta) \right)
  \DEF{$\O{Ai}(z)$} \]
\[ \O{Bi}(z) = \frac1\pi \int_0^\infty e^{-t^3/3 + xt}\sin(t^3/3 + xt)\,dt
  = \frac{\sqrt{z}}{3}\left( \O{I}_{-1/3}(\zeta) + \O{I}_{1/3}(\zeta) \right)
  \DEF{$\O{Bi}(z)$} \]
where $\zeta = (2/3)z^{3/2}$.

[For positive $z$ use continued fraction from power series??]

We also have
\[ \O{Ai}(-z) = \frac{\sqrt{z}}{3}\left( \O{J}_{1/3}(\zeta) - \O{J}_{-1/3}(\zeta) \right) \]
\[ \O{Bi}(-z) = \sqrt{\frac{z}{3}}\left( \O{J}_{-1/3}(\zeta) - \O{J}_{1/3}(\zeta) \right) \]

Plus scaled versions, derivatives, integrals, ...
(joint versions)

Remark: one could attempt to do "Taylor stepping" (repeated expansion of power-series as analytic continuation...)
One can derive a recurrence for the derivatives at an arbitrary point from the ordinary differential equation satisfied
by Airy functions $w'' - zw = 0$ gives $w^{(n+3)} = z\cdot w^{(n+1)} + n\cdot w^{(n)}$.  However, in practice this works
poorly: backwards is unstable, forwards gives low accuracy, relative error is $O(h^3)$ for step-sizes of $h$, etc.

\begin{implementation}
Atlas~\cite{atlas:thompson} recommends to compute these as
$\O{Ai}(z) = c_1 f(x) - c_2 g(x)$, $\O{Bi}(z) = \sqrt3[c_1 f(x) + c_2 g(x)]$,
where $c_1 = (3^{2/3}\Gam(2/3))^{-1}$, $c_2 = (3^{1/3}\Gam(1/3))^{-1}$, and
\[ f(x) = \sum_{k=0}^\infty \frac{1\cdot4\cdot7\cdots(3k-2)}{(3k)!}x^{3k} = 1 + \frac{x^3}{3!} + 4\frac{x^6}{6!} + \cdots \]
\[ g(x) = \sum_{k=0}^\infty \frac{2\cdot5\cdot8\cdots(3k-1)}{(3k+1)!}x^{3k} = x(1 + 2\frac{x^3}{4!} + 2\cdot5\frac{x^6}{7!} + \cdots) \]

For $x>>0$, use the asymptotics
\[ \O{Ai}(x) \sim \frac{e^{-\zeta}}{2\sqrt{\pi x^{1/2}}}\left(\sum_{k=0}^\infty(-)^k\frac{t_k}{\zeta^k}\right) \]
\[ \O{Bi}(x) \sim \frac{e^{\zeta}}{2\sqrt{\pi x^{1/2}}}\left(\sum_{k=0}^\infty\frac{t_k}{\zeta^k}\right) \]
where $t_k = \frac{(2k+1)(2k+3)\cdots(6k-1)}{216^k k!}$ (for $k>0$), $t_0=1$.

For $x<<0$, use the asymptotics
\[ \O{Ai}(x) \sim \frac{1}{\sqrt{\pi x^{1/2}}}\left[ \sin(\zeta+\pi/4) F(\zeta) - \cos(\zeta+\pi/4) G(\zeta) \right] \]
\[ \O{Bi}(x) \sim \frac{1}{\sqrt{\pi x^{1/2}}}\left[ \cos(\zeta+\pi/4) F(\zeta) + \sin(\zeta+\pi/4) G(\zeta) \right] \]
where $F(\zeta) = \sum_{k=0}^\infty (-)^k c_{2k} / \zeta^{2k}$ and $G(\zeta) = \sum_{k=0}^\infty (-)^k c_{2k+1} / \zeta^{2k+1}$.
(In practice, he solves for $F, G$ in terms of $\O{Ai}, \O{Bi}$ and computes a rational approximation to them.)
\end{implementation}

Discussion based on Gil {\it et al.}:
For scaling purposes, convenient to define $\widetilde{\O{Ai}}(z) = e^{\zeta}\O{Ai}(z)$ where $\zeta=\frac23 z^{3/2}$.
To compute $\O{Ai}(z),\O{Bi}(z)$ for complex $z=x+y\ii$, we divide into regions:
\begin{enumerate}
  \item[(A)] Use a series when $|y|<3$ and $-2.6<x<1.3$
  \item[(B)] Use asymptotic expansion for $|z|>15$
  \item[(C)] Use Gauss-Legendre integration when $|z|<15$ (and not case (A))
  \item[Note:] For cases (B),(C) compute normally when $|\arg z|\leq2\pi/3$, otherwise use
    \[ \O{Ai}(z) = -e^{-2\pi\ii/3}\O{Ai}(e^{-2\pi\ii/3}z) -e^{2\pi\ii/3}\O{Ai}(e^{2\pi\ii/3}z) \]
\end{enumerate}
The integral for case (C) is
\[ \O{Ai}(z) = a(z) \int_0^\infty (2+t/\zeta)^{-1/6} t^{-1/6} e^{-t} \,dt \]
where
\[ a(z) = \frac{e^{-\zeta}\zeta^{-1/6}}{\sqrt{\pi} 48^{1/6} \Gam(5/6)} \]
note for $z\neq0$, $|\arg\zeta|<\pi$.
Similarly,
\[ \O{Ai}'(z) = b(z) \int_0^\infty (2+t/\zeta)^{1/6} t^{1/6} e^{-t} \,dt \]
where
\[ b(z) = \frac{-z e^{-\zeta}}{\sqrt{3\pi} 2^{2/3} \Gam(7/6)\sqrt\zeta} \]

%%%%%%%%%%%%%%%%%%%%
\subsubsection{Scorer functions}
\[ \O{Gi}(z) \text{\ is solution of inhomogeneous Airy DE $y''-z y = -\frac1\pi$ with $\O{Gi}(0)=\O{Bi}(0)/3$, $\O{Gi}'(0)=\O{Bi}'(0)/3$, }
  \DEF{$\O{Gi}(z)$} \]
Note that $\O{Gi}(0) = \frac{1}{3^{7/6}\Gam(2/3)}$ and $\O{Gi}'(0) = \frac{1}{3^{5/6}\Gam(1/3)}$.
\[ \O{Hi}(z) \text{\ is solution of inhomogeneous Airy DE $y''-z y = +\frac1\pi$ with $\O{Hi}(0)=2\O{Bi}(0)/3$, $\O{Hi}'(0)=2\O{Bi}'(0)/3$, }
  \DEF{$\O{Hi}(z)$} \]
Similarly, $\O{Hi}(0) = \frac{2}{3^{7/6}\Gam(2/3)}$ and $\O{Hi}'(0) = \frac{2}{3^{5/6}\Gam(1/3)}$.

\[ \O{Gi}(z) = \O{Bi}(z)\int_z^\infty\O{Ai}(t)\,dt + \O{Ai}(z)\int_0^z\O{Bi}(t)\,dt \]
\[ \O{Hi}(z) = \O{Bi}(z)\int_{-\infty}^z\O{Ai}(t)\,dt - \O{Ai}(z)\int_{-\infty}^z\O{Bi}(t)\,dt \]

For small $|z|$ we can use series
\[ \O{Gi}(z) = \frac{3^{-2/3}}{\pi}\sum_{k=0}^\infty \cos(\pi\tfrac{2k-1}{3})\Gam(\tfrac{k+1}{3})\frac{(3^{1/3} z)^k}{k!} \]
\[ \O{Hi}(z) = \frac{3^{-2/3}}{\pi}\sum_{k=0}^\infty                         \Gam(\tfrac{k+1}{3})\frac{(3^{1/3} z)^k}{k!} \]
Note that from periodicity of $\cos$, we don't need to compute for each term.
Similarly, $\Gam(n/3)$ can be computed via recurrence from $\Gam(1/3)$ and $\Gam(2/3)$.

For Mathematica use, these integral forms were useful:
\[ \O{Gi}(z) = \frac{1}{\pi}\int_0^\infty\sin(\tfrac{t^3}{3} + xt)\,dt \qquad x\in\RR \]
\[ \O{Hi}(z) = \frac{1}{\pi}\int_0^\infty e^{-t^3/3 + zt}\,dt \]

Another integral form (useful for intermediate $|z|$?) is
\[ \O{Gi}(z) = -\frac{1}{\pi}\int_0^\infty e^{-t^3/3 - zt/2}\cos(\tfrac{\sqrt3}{2}zt + \tfrac23\pi)\,dt \]

For large $|z|\to\infty$ we have asymptotic forms
\[ \O{Gi}(z) \sim \frac{1}{\pi z}\sum_{k=0}^\infty \frac{(3k)!}{k!(3z^3)^k} \qquad |\O{ph}z|\leq\tfrac\pi3 - \delta \]
\[ \O{Hi}(z) \sim -\frac{1}{\pi z}\sum_{k=0}^\infty \frac{(3k)!}{k!(3z^3)^k} \qquad |\O{ph} {-z}|\leq\tfrac23\pi - \delta \]
For other phase ranges, use the connection formulas
\[ \O{Gi}(z) + \O{Hi}(z) = \O{Bi}(z) \]
\[ \O{Gi}(z) = \frac12 e^{\pi\ii/3}\O{Hi}(z e^{-2\pi\ii/3}) + \frac12 e^{-\pi\ii/3}\O{Hi}(z e^{2\pi\ii/3}) \]
\[ \O{Gi}(z) = e^{\mp\pi\ii/3}\O{Hi}(z e^{\pm2\pi\ii/3}) \pm \ii \O{Ai}(z) \]
\[ \O{Hi}(z) = e^{\pm2\pi\ii/3}\O{Hi}(z e^{\pm2\pi\ii/3}) + 2 e^{\mp\pi\ii/6}\O{Ai}(z e^{\mp2\pi\ii/3}) \]

Finally a few other integral representations:
\[ \O{Gi}(x) = \frac{4x^2}{3^{3/2}\pi} \dashint_0^\infty\frac{\O{K}_{1/3}(t)}{\zeta^2 - t^2}\,dt \qquad x>0, \text{CHECK!} \]
\[ \O{Hi}(-z) = \frac{4z^2}{3^{3/2}\pi^2} \int_0^\infty\frac{\O{K}_{1/3}(t)}{\zeta^2 + t^2}\,dt \qquad |\O{ph} z|<\tfrac\pi3 \]
where $\zeta=\tfrac23z^{3/2}$
\[ \O{Hi}{z} = \frac{3^{-2/3}}{2\pi^2\ii} \int_{-\ii\infty}^{+\ii\infty} \Gam(\tfrac13+\tfrac13t)\Gam(-t)(3^{1/3}e^{\pi\ii}z)^t\,dt \]
where the integration contour separates the poles of $\Gam(\tfrac13+\tfrac13t)$ (which are $-1, -4, -7, -10, \dots$)
from the poles of $\Gam(-t)$ (which are $0, 1, 2, 3, \dots$).

%%%%%%%%%%%%%%%%%%%%%%%%%%%%%%%%%%%%%%%%
\subsection{Kelvin functions}
Regular at 0
\[ \O{ber}_\nu(z) = \sum_{n=0}^\infty \frac{(z/2)^{2n+\nu}}{n!(n+\nu)!}\cos( (3\nu/4 + n/2) \pi)\]
\[ \O{bei}_\nu(z) = \sum_{n=0}^\infty \frac{(z/2)^{2n+\nu}}{n!(n+\nu)!}\sin( (3\nu/4 + n/2) \pi)\]
Note $\O{ber}_\nu(z) - \ii\O{bei}_\nu(z) = \O{J}_\nu(\ii^{-3/2}z)$.
Irregular at 0
\[ \O{ker}_\nu(z) = \]
\[ \O{kei}_\nu(z) = \]
Note $\O{ker}_\nu(z) \pm \ii\O{kei}_\nu(z) = \ii^{\mp\nu}\O{K}_\nu(\ii^{\pm1/2}z)$.
{\tt ber, bei, ker, kei, d\_ber, d\_bei, ..., Int\_ber, ...}

%%%%%%%%%%%%%%%%%%%%%%%%%%%%%%%%%%%%%%%%
\subsection{Struve functions}
\[ \Ob{H}_\nu(z) = \sum_{n=0}^\infty (-)^n \frac{(z/2)^{2n+\nu+1}}{\Gamma(n+3/2)\Gamma(n+\nu+3/2)} \]
\[ \Ob{L}_\nu(z) = -\ii e^{-\ii\nu\pi/2}\O{\mathbf{H}}_\nu(\ii z) = \sum_{n=0}^\infty \frac{(z/2)^{2n+\nu+1}}{\Gamma(n+3/2)\Gamma(n+\nu+3/2)} \]
{\tt Struve\_Hn, d\_Struve\_Hn, ..., Ln, Integrate\_Struve\_Hn, ...}

(Note: series/Bessel series + asymptotic-in-$z$ seems to do well, whereas asymptotic-in-$\nu$ does poorly).

From A\&S~\cite{a&s}:

The Struve function $\Ob{H}_\nu(z)$ is a particular solution to the non-homogeneous Bessel differential equation
$z^2w'' + zw' + (z^2-\nu^2)w = \frac{4(z/2)^{\nu+1}}{\sqrt{\pi}\Gam(\nu+1/2)}$
(which thus has a general solution $w=\al\O{J}_\nu(z) + \beta\O{Y}_\nu(z) + \Ob{H}_\nu(z)$)
and $z^{-\nu}\Ob{H}_\nu(z)$ is an entire function in $z$.
\[ \Ob{H}_\nu(z) = \left(\frac{z}{2}\right)^{\nu+1}\sum_{k=0}^\infty(-)^k\frac{(z/2)^{2k}}{\Gam(k+3/2)\Gam(k+\nu+3/2)} \]
thus
\[ \Ob{H}_0(z) = \frac2\pi\left(z-\frac{z^3}{1^2\cdot3^2}+\frac{z^5}{1^2\cdot3^2\cdot5^2}-\cdots\right) \]
and
\[ \Ob{H}_1(z) = \frac2\pi\left(\frac{z^2}{1^2\cdot3}-\frac{z^4}{1^2\cdot3^2\cdot5}+\frac{z^6}{1^2\cdot3^2\cdot5^2\cdot7}\cdots\right) \]
We can also write
\[ \Ob{H}_\nu(z) = \frac{2(z/2)^{\nu+1}}{\sqrt{\pi} \Gam(\nu+3/2)} \Hyper{1}{2}{1}{\tfrac32+\nu, \tfrac32}{-\frac{z^2}{4}} \]

We have the recurrence
\[ \Ob{H}_{\nu-1} + \Ob{H}_{\nu+1} = \frac{2\nu}{z}\Ob{H}_\nu + \frac{(z/2)^\nu}{\sqrt{\pi} \Gam(\nu+3/2)} \]
((upward recursion is unstable... power-series seems quite robust enough (for $|x|<20$...)))

And we have the asymptotic expansion for large $z$ with $|\arg z|<\pi$
\[ \Ob{K}_\nu(z) = \Ob{H}_\nu(z) - \O{Y}_\nu(z) = \frac{1}{\pi}\sum_{k=0}^{m-1}\frac{\Gam(k+1/2)}{\Gam(\nu-k+1/2)(z/2)^{2k-\nu+1}} + R_m \DEF{$\Ob{K}_\nu(z)$} \]
with $R_m=O(|z|^{\nu-2m-1})$; if $\nu$ is real, $z$ positive, and $m-\nu+\tfrac12\geq0$, the remainder after $m$ terms is of the
same sign and numerically less than the first term neglected.

Claimed: for large $|z|$ with $|\arg z|\leq\pi-\delta$, $\nu$ fixed, that
\[ \Ob{K}_\nu(z) \sim \frac{1}{\pi} \sum_{k=0}^\infty \frac{\Gam(k+1/2)(z/2)^{\nu-2k-1}}{\Gam(\nu+1/2-k)} \]
For large $\nu$ with $|\arg\nu|\leq\pi/2-\delta$, fixed $\lam=z/\nu>0$,
\[ \Ob{K}_\nu(\lam\nu) \sim \frac{(\lam\nu/2)^{\nu-1}}{\sqrt{\pi}\Gam(\nu+1/2)} \sum_{k=0}^\infty \frac{k! c_k(\lam)}{\nu^k} \]
where $c_0=1$, $c_1=2\lam^{-2}$, $c_2=6\lam^{-4}-\frac12\lam^{-2}$, $c_3=20\lam^{-6}-4\lam^{-4}$,
$c_4=70\lam^{-8}-\frac{45}{2}\lam^{-6}+\frac{3}{8}\lam^{-4}$, etc.

The modified Struve function is
\[ \Ob{L}_\nu(z) = -\ii e^{-\ii\nu\pi/2} \Ob{H}_\nu(\ii z)
    = \left(\frac{z}{2}\right)^{\nu+1}\sum_{k=0}^\infty\frac{(z/2)^{2k}}{\Gam(k+3/2)\Gam(k+\nu+3/2)} \]

\[ \Ob{M}_\nu(z) = \Ob{L}_\nu(z) - I_\nu(z) \]

Other miscellaneous notes:
\[ \Ob{H}_\nu' = \frac{\Ob{H}_{\nu-1} - \Ob{H}_{\nu+1}}{2} + \frac{(z/2)^\nu}{2\sqrt{\pi} \Gamma(\nu+3/2)} \]
\[ \Ob{H}_0' = (2/\pi) - \Ob{H}_1 \]
\[ \Ob{L}_{\nu-1} - \Ob{L}_{\nu+1} = \frac{2\nu}{z}\Ob{L}_\nu + \frac{(z/2)^\nu}{\sqrt{\pi} \Gam(\nu+3/2)} \]
\[ \Ob{L}_\nu' = \frac{\Ob{L}_{\nu-1} + \Ob{L}_{\nu+1}}{2} + \frac{(z/2)^\nu}{\sqrt{\pi} \Gamma(\nu+3/2)} \]
[double check for typos here...]
\[ \int_0^z \Ob{H}_\nu(t)\,dt = \sum_{k=0}^\infty\frac{(-)^k(z/2)^{2k+2+\nu}}{(2k+2+\nu)\Gamma(k+3/2)\Gamma(\nu+k+3/2)}
    = \left(\frac{z}{2}\right)^{\nu+2}\sum_{k=0}^\infty\frac{(-)^k(z/2)^{2k}}{(2k+2+\nu)\Gamma(k+3/2)\Gamma(\nu+k+3/2)} \]
\[ \int_0^z \Ob{H}_0(t)\,dt = \frac{2}{\pi}\left(\frac{z^2}{2} - \frac{z^4}{1^2\cdot3^2\cdot4} + \frac{z^6}{1^2\cdot3^2\cdot5^2\cdot6} - \cdots\right) \]
(just integrate term-by-term the power-series expansion)

Strategy:
\begin{itemize}
  \item series/Bessel-series for small to moderate $|z|$
  \item $z$-asymptotics for large $|z|$
  \item ?? use $\nu$-asymptotics by using $\nu$-recurrence to get large $\nu$?  (Recur down from large $\nu$-asymptotic values)
\end{itemize}

\begin{implementation}
From~\cite{atlas:thompson}:
To compute the Struve function $\Ob{H}_n(x)$ for $n=0,\dots,N$ (for real $x$), we:
\begin{itemize}
\item if $x=0$, then all values are $0$
\item if $x<N/2$, use a power-series to compute $\Ob{H}_{N+8}(x)$, $\Ob{H}_{N+7}(x)$ and then use the recursion
    downward to $n=0$
\item if $N/2\leq x\leq25$, use power-series to compute $\Ob{H}_0(x)$, $\Ob{H}_1(x)$ and use the recursion upward to $n=N$
\item else ($N/2,25<x$) use the asymptotic expansion to get $\Ob{H}_0(x)$, $\Ob{H}_1(x)$ and use the recursion upward to $n=N$
\end{itemize}
To compute the modified Struve function $\Ob{L}_n(x)$ for $n=0,\dots,N$ (for real $x$), we (for $x>0$) use the power-series or
asymptotic expansion to compute $\Ob{L}_{N}(x)$, $\Ob{L}_{N-1}(x)$, then use the backwards recursion.

From~\cite{zj}:
\begin{itemize}
\item for $\Ob{H}_0(x)$, $\Ob{H}_1(x)$ --- use power-series for $x<20$, asymptotic (with $Y_\nu$) for $x>20$
\item for $\Ob{H}_\nu(x)$, ($\nu$ arbitrary), use power-series for $x<20$, asymptotic o.w.
\item *OR* use
    \[ \Ob{H}_{-n-1/2}(z) = (-)^nJ_{n+1/2}(z) \]
    \[ \Ob{H}_{\nu}(z) = \frac{1}{\Gam(\nu+1/2)}\sum_{k=0}^\infty \frac{(z/2)^{k+\nu+1/2}}{k!(\nu+k+1/2)}J_{k+1/2}(z)
        = \left(\frac{z}{2\pi}\right)^{1/2} \sum_{k=0}^\infty \frac{(z/2)^k}{k!(k+1/2)} J_{k+\nu+1/2}(z) \]
    and note that the $J_{n+1/2}$ are easy to evaluate using the recurrence for Bessel functions,
    $J_{\nu-1}(z)+J_{\nu+1}(z)=\frac{2\nu}{z}J_{nu}(z)$ and $J_\nu(z)\sim\frac{1}{\sqrt{2\pi\nu}}(\frac{ez}{2\nu})^\nu$.
    (We can even get a recurrence for the full term in the summand.)
\item We also have the following, for $\nu\neq-1,-2,-3,\dots$, (which seems to work the best), 
    \[ \Ob{H}_{\nu}(z) = \frac{4}{\sqrt\pi \Gam(\nu+1/2)} \sum_{k=0}^\infty \frac{2k+\nu+1}{(2k+1)(2k+2\nu+1)} \frac{\Gam(k+\nu+1)}{k!} J_{2k+\nu+1}(z) \]
\end{itemize}
\end{implementation}

%%%%%%%%%%%%%%%%%%%%%%%%%%%%%%%%%%%%%%%%%%%%%%%%%%%%%%%%%%%%%%%%%%%%%%%%%%%%%%%%
\section{Spherical harmonics, Legendre functions}
spherical Legendre, toroidal Legendre, conical Legendre (Atlas~\cite{atlas:thompson} for Computing...)
{\tt Pn, Pnu, dPn, dPnu, assocPmn, assocPmnu, Qn, ..., Integrate\_Pn, ...} polynomials ...
``spherical legendre'', ``conical legendre'', ``toroidal legendre'', ...

\[ \O{P}_\nu(z) = \]
\[ \O{Q}_\nu(z) = \frac{\pi}{2} \frac{\O{P}_\nu(z) \cos\nu z - \O{P}_\nu(-z)}{\sin \nu\pi} \]
\[ \O{P}^m_\nu(z) = (-)^m(1-z^2)^{m/2} \frac{d^m}{dz^m} \O{P}_\nu(z) \]
\[ \O{Q}^m_\nu(z) = (-)^m(1-z^2)^{m/2} \frac{d^m}{dz^m} \O{Q}_\nu(z) \]

According to Lebedev, ``spherical harmonics'' are solutions of $(1-z^2)u'' - 2zu' + [\nu(\nu+1)-\mu^2/(1-z^2)]u = 0$
(typically $z\in(-1,1)$ or sometimes $z\in[-\infty,1]$; $\mu=0,1,2,\dots$; $\nu\in\RR$ or $\nu\in\CC$).

[Lebedev] Legendre functions take $\mu=0$ so we have $(1-z^2)u'' - 2zu' + \nu(\nu+1)u = 0$ and get
Legendre functions of the first kind:
\[ \O{P}_\nu(z) = F[-\nu,\nu+1; 1; \frac{1-z}{2}]  \qquad|z-1|<2 \]
and Legendre functions of the second kind:
\[ \O{Q}_\nu(z) = \]
Both of these can be analytically continued to a larger domain.  Note that $P_n(z)$ are the Legendre polynomials for $n\in\NNo$.

[Lebedev] The associated Legendre functions (of the first/second kinds) solve the more general equation with $\mu=m=0,1,2,\dots$
where
\[ \O{P}^m_\nu(z) = (z^2-1)^{m/2} \frac{d^m}{dz^m} \O{P}_\nu(z) \]
\[ \O{Q}^m_\nu(z) = (z^2-1)^{m/2} \frac{d^m}{dz^m} \O{Q}_\nu(z) \]
and the associated Legendre functions of the first/second kind {\em for the interval $(-1,1)$} are given by:
\[ \O{P}^m_\nu(x) = (-)^m(1-x^2)^{m/2} \frac{d^m}{dx^m} \O{P}_\nu(x) \]
\[ \O{Q}^m_\nu(x) = (-)^m(1-x^2)^{m/2} \frac{d^m}{dx^m} \O{Q}_\nu(x) \]

%%%%%%%%%%%%%%%%%%%%%%%%%%%%%%%%%%%%%%%%%%%%%%%%%%%%%%%%%%%%%%%%%%%%%%%%%%%%%%%%
\section{Exponential integrals}

\[ \O{Ei}(z) = \int_{-\infty}^z \frac{e^t}{t}\,dt  \qquad|\arg {-z}|<\pi  \qquad\text{(correct?) Lebedev agrees}\]
\[ \O{Ei}_1(x) = \gam + \log z + \sum_{k=1}^\infty \frac{x^k}{k! k} \qquad x>0\]
\[ \O{E}_n(z) = \int_1^\infty\frac{e^{-zt}}{t^n}\,dt \]
Generalized exponential-integral for $p\in\CC$:
\[ \O{E}_p(z) = z^{p-1}\Gam(1-p,z) = z^{p-1}\int_z^\infty\frac{e^{-t}}{t^p}\,dt \]

\[ \O{li}(z) = \int_{0}^z\frac{dt}{\log t} = \O{Ei}(\ln z) \qquad\text{$|\arg z|<\pi$ and $|\arg 1-z|<\pi$} \]
\[ \O{li}_1(z) = xxx  \qquad\text{modified log-integral (Lebedev)}\]

{\tt Ei, En, Ep, Li, Ci, Chi, Si, Shi, atan\_int} --- real / complex / adjusted real

\begin{implementation}
Implementation notes for $\O{Ei}(x)$:
\begin{itemize}
\item For small $x$, use $\O{Ei}(x) = \gamma + \ln x + \sum_{m=1}^\infty \frac{x^m}{m! m}$
  (this has some issues near the root at $0.3725074107813666\dots$ with relative precision, dropping to 14 digits
  using the expression $\O{Ei}(x) = \ln(x\cdot\exp(\gamma + \sum\cdots))$ helps with accuracy in that area)
\item For large $x$, use asympotics $\O{Ei}(x) \sim \frac{e^x}{x}(1 + \sum_{m=0}^\infty\frac{m!}{x^m})$
  (this only gives 15 digits accuracy for around $x>38$, so use, say, for $x>=40$)
\end{itemize}
\end{implementation}

Continued fraction for $\O{E}_1(z)$ for $|\O{ph} z|<\pi$:
\[ \O{E}_1(z) = \frac{e^{-z}}{z +} \frac{1}{1 +} \frac{1}{z +} \frac{2}{1 +} \frac{2}{z +} \frac{3}{1 +} \frac{3}{z +} \cdots \]
and factorial series for $\Re z>0$:
\[ \O{E}_1(z) = e^{-z}\left( \frac{c_0}{z} + \frac{c_1}{z(z+1)} + 2!\frac{c_2}{z(z+1)(z+2)} + 3!\frac{c_3}{(z)_4} + \cdots \right) \]
where $c_0=1$, $c_1=-1$, $c_2=1/2$, $c_3=-1/3$, $c_4=1/6$, and $c_k=-\sum_{j=0}^{k-1}\tfrac{c_j}{k-j}$ for $k\geq1$.

\begin{implementation}
Implementation notes for $\O{E}_n(x)$:
\begin{itemize}
\item For $n=0$, use $\O{E}_0(x) = e^{-x}/x$
\item For $n=1$, use $\O{E}_1(x) = -\gamma - \ln x - \sum_{k=1}^\infty\frac{(-x)^k}{k! k}$
\item For $x\leq1$, use series $\O{E}_n(x) = \frac{(-x)^{n-1}}{(n-1)!}(-\ln x + \psi(n)) - \sum_{m=0,m\neq n-1}^\infty \frac{(-x)^m}{(m-(n-1))m!}$
\item for $x>1$, use continued fraction $e^{-x}\left[\frac{1}{x+n-}\ \frac{1\cdot n}{x+n+2-}\ \frac{2\cdot(n+1)}{x+n+4-}\ \cdots\right]$
  (see~\cite{zj})
\item the recurrence $\O{E}_{n+1}(x) = \frac{1}{n}(e^{-x} - x \O{E}_n(x))$ $(n>1)$ is forward and backward unstable but works ok for small values
  (tested for $x<1$ and $n<10$ - should be fine for $x<1$ and any $n$; seems to work decent if $x<<n$)
\item asymptotic expansion $\O{E}_n(x) \sim \frac{e^{-x}}{x}(1 - \frac{n}{x} + \frac{n(n+1)}{x^2} - \frac{n(n+1)(n+2)}{x^3} + \cdots)$ gives quite
  disappointing results - works only when $x>>n$
\end{itemize}
\end{implementation}

%%%%%%%%%%%%%%%%%%%%%%%%%%%%%%%%%%%%%%%%%%%%%%%%%%%%%%%%%%%%%%%%%%%%%%%%%%%%%%%%
\section{Wave equation solutions}
(From Temme ...)

Spheroidal Wave Functions
- first/second kind, prolate/oblate, characteristic values, expansion coefficients, wave/angular functions,
radial/angular functions, spheroidal Bessel functions, ...

%%%%%%%%%%%%%%%%%%%%%%%%%%%%%%%%%%%%%%%%%%%%%%%%%%%%%%%%%%%%%%%%%%%%%%%%%%%%%%%%
\section{Theta functions}

%%%%%%%%%%%%%%%%%%%%%%%%%%%%%%%%%%%%%%%%
\subsection{Jacobian theta functions}

Notation: $\tau$ is the ``lattice parameter'', $q=e^{\ii\pi\tau}$ is the ``nome'', $z$ is the ``argument''
Assume that $\Im\tau > 0$ (thus $0<|q|<1$ and $q\notin[-\infty,0]$).

\[ \theta_1(z|\tau) = \theta_1(z,q) = 2\sum_{n=0}^\infty(-)^n q^{(n+1/2)^2}\sin (2n+1)z \]
\[ \theta_2(z|\tau) = \theta_2(z,q) = 2\sum_{n=0}^\infty q^{(n+1/2)^2}\cos (2n+1)z \]
\[ \theta_3(z|\tau) = \theta_3(z,q) = 1 + 2\sum_{n=1}^\infty q^{n^2}\cos 2nz \]
\[ \theta_4(z|\tau) = \theta_4(z,q) = 1 + 2\sum_{n=1}^\infty(-)^n q^{n^2}\cos 2nz \]

{\tt theta\_1, theta\_2, theta\_3, theta\_4, ...}, $\theta'_j/\theta_j$, Neville theta functions: $\theta_s$,n,d,c;
conversions $k$ to/from $q$, etc.; real/imaginary argument/complex, etc.

\begin{implementation}
\begin{itemize}
\item For $q\in\RR$, we can simply use the series definitions and we'll get fast convergence, though it slows down
  near $1$
\item Near $q=1$ use the transformations (where $\tau' = -1/\tau$ and square roots assume principal values)
  \[\sqrt{-\ii\tau} \theta_1(z|\tau) = -\ii e^{\ii\tau' z^2/\pi} \theta_1(z\tau'|\tau') \]
  \[\sqrt{-\ii\tau} \theta_2(z|\tau) = e^{\ii\tau' z^2/\pi} \theta_4(z\tau'|\tau') \]
  \[\sqrt{-\ii\tau} \theta_3(z|\tau) = e^{\ii\tau' z^2/\pi} \theta_3(z\tau'|\tau') \]
  \[\sqrt{-\ii\tau} \theta_4(z|\tau) = e^{\ii\tau' z^2/\pi} \theta_2(z\tau'|\tau') \]
\item For example, letting $\phi=-\ln q/\pi$ and $q'=e^{-\pi/\phi}$, we end up with the very rapidly converging (near $q=1$):
  \[ \theta_3(z,q) = \frac{e^{-z^2/\pi\phi}}{\sqrt\phi}\left[ 1 + 2\sum_{n=1}^\infty (q')^{n^2} \cosh 2nz/\phi \right] \]
  (actually, in practice it seems to work generally better (for $z=0.3$) than the original sum with just a couple more terms needed near $q=0$...
  oops with $z\sim\pi/2$ we have some issues with the transformed version dying with NaN for $q\sim1$ (though giving much better results until then
  than the original series --- just need to deal with overflow/underflow issues with exp and cosh...
  yes, writing $(q')^{n^2}\cosh(2nz/\phi) = e^{-n^2\pi/\phi + 2nz/\phi}(1 + e^{-4nz/\phi})/2$ gives much better results (though seem to lose a
  bit of precision...))
\end{itemize}
\end{implementation}

%%%%%%%%%%%%%%%%%%%%%%%%%%%%%%%%%%%%%%%%%%%%%%%%%%%%%%%%%%%%%%%%%%%%%%%%%%%%%%%%
\section{Elliptic integrals}
Weierstrass elliptic function $\mathfrak{p}$, inverse weierstrass, elliptic zeta function, etc.
Fock functions, etc.

General notation notes: the ``nome'' $q$ and the ``modulus'' $k$ can be related as follows, where $k'=\sqrt{1-k^2}$:
\begin{eqnarray*}
q &=& e^{-\pi K'(k)/K(k)} \\
k &=& \theta_2^2(0,q) / \theta_3^2(0,q) \\
k' &=& \theta_4^2(0,1) / \theta_3^2(0,q) \\
K(k) &=& \frac{\pi}{2} \theta_3^2(0,q) 
\end{eqnarray*}
(conversions)

%%%%%%%%%%%%%%%%%%%%%%%%%%%%%%%%%%%%%%%%
\subsection{Elliptic integrals}
First kind: ($x=\sin\phi$), $0\leq k\leq 1$
\[ \O{F}(\phi,k) = \int_0^\phi \frac{d\theta}{\sqrt{1-k^2\sin^2\theta}} = \int_0^x \frac{dt}{\sqrt{(1-t^2)(1-k^2t^2)}} \]
Second kind: ($x=\sin\phi$), $0\leq k\leq 1$
\[ \O{E}(\phi,k) = \int_0^\phi \sqrt{1-k^2\sin^2\theta}\,d\theta = \int_0^x\frac{\sqrt{1-k^2t^2}}{\sqrt{1-t^2}}\,dt \]
Third kind: ($x=\sin\phi$), $0\leq k\leq 1$
\[ \O{\Pi}(\phi,k,c) = \int_0^\phi\frac{d\theta}{(1-c\sin^2\theta)\sqrt{1-k^2\sin^2\theta}} = \int_0^x\frac{dt}{(1-ct^2)\sqrt{(1-t^2)(1-k^2t^2)}} \]

(In Mathematica notation, $n=c$, $m=k^2$).

Special case:
\[ \O{\Pi}(\phi,0,k) = \int_0^\phi\frac{d\theta}{1 - \xi\sin^2\theta} = \frac{\atan(\sqrt{1-\xi} \tan(\phi))}{\sqrt{1-\xi}} \]

\begin{implementation}
To implement $\O{F}(\phi,k)$ and $\O{E}(\phi,k)$, use the ascending Landen transforms.

To implement $\O{\Pi}(k,c,\phi)$, the Gauss transform is effective.
\end{implementation}


%%%%%%%%%%%%%%%%%%%%%%%%%%%%%%%%%%%%%%%%
\subsection{Complete elliptic integrals}
\[ \O{K}(k) = \O{F}(\pi/2,k) \]
\[ \O{E}(k) = \O{E}(\pi/2,k) \]
\[ \O{\Pi}(k,c) = \O{\Pi}(\pi/2,k,c) \]

We use AGM approaches to compute these
\begin{implementation}
To implement $\O{\Pi}(k,c)$, we use the AGM approach: for $-\infty<k^2<1$, $-\infty<c<1$, we have
\[ \O{\Pi}(c,k) = \frac{\pi}{4 M(1,k')}\left(2 + \frac{c}{1-c}\sum_{n=0}^\infty Q_n\right) \]
where $a_0 = 1$, $b_0 = k'$, $p_0^2 = 1-c$, $Q_0 = 1$,
and
\begin{eqnarray*}
p_{n+1} &=& \frac{p_n^2 + a_n b_n}{2 p_n} \\
Q_{n+1} &=& \frac12 Q_n \eps_n \\
\eps_{n} &=& \frac{p_n^2 - a_n b_n}{p_n^2 + a_n b_n}
\end{eqnarray*}
\end{implementation}

Also $Z, FK, CE, \Lambda_0, RC, RD, RF, RJ$ ...

%%%%%%%%%%%%%%%%%%%%%%%%%%%%%%%%%%%%%%%%
\subsection{Complementary complete elliptic integrals}
Let $k' = \sqrt{1-k^2}$. Then
\[ \O{K}'(k) = \O{K}(k') \]
\[ \O{E}'(k) = \O{E}(k') \]

%%%%%%%%%%%%%%%%%%%%%%%%%%%%%%%%%%%%%%%%
\subsection{Carlson's symmetric elliptic integrals}

\[ \O{R}_C(x,y) = \frac12 \int_0^\infty \frac{dt}{\sqrt{t+x}(t+y)} \qquad x\in\CC\setminus(-\infty,0), y\in\CC\setminus\{0\} \]

\begin{implementation}
To compute $\O{R}_C(x,y)$ for $x,y\in\RR$, we simply use the closed-form expressions
\begin{itemize}
\item $\O{R}_C(x,x) = x^{-1/2}$
\item For $0\leq x<y$:
    \[ \O{R}_C(x,y) = \frac{1}{\sqrt{y-x}}\atan\sqrt{\frac{y-x}{x}} = \frac{1}{\sqrt{y-x}}\acos\sqrt{\frac{x}{y}} \]
\item For $0<y<x$: (note there is a typo in DLMF for this formula)
    \[ \O{R}_C(x,y) = \frac{1}{\sqrt{x-y}}\atanh\sqrt{\frac{x-y}{x}} = \frac{1}{\sqrt{x-y}}\ln\frac{\sqrt{x} + \sqrt{x-y}}{\sqrt{y}} \]
\item For $y<0\leq x$:
    \[ \O{R}_C(x,y) = \sqrt{\frac{x}{x-y}} \O{R}_C(x-y,-y) = \frac{1}{\sqrt{x-y}}\atanh\sqrt\frac{x}{x-y} = \frac{1}{\sqrt{x-y}}\ln\frac{\sqrt{x}+\sqrt{x-y}}{\sqrt{-y}} \]
\end{itemize}
\end{implementation}

Let $s(t)=\sqrt{t+x}\sqrt{t+y}\sqrt{t+z}$, $x,y,z\in\CC\setminus(-\infty,0]$, and $p\neq0$, then
\[ \O{R}_F(x,y,z) = \frac{1}{2} \int_0^\infty \frac{dt}{s(t)} \]
\[ \O{R}_J(x,y,z,p) = \frac{3}{2} \int_0^\infty \frac{dt}{s(t)(t+p)} \]
\[ \O{R}_G(x,y,z) = \frac{1}{4\pi} \int_0^{2\pi}\int_0^\pi
  \left\{ x\sin^2\theta \cos^2\theta + y\sin^2\theta\sin^2\phi + z\cos^2\theta \right\}^{1/2}\sin\theta\,d\theta\,d\phi \]
\[ \O{R}_D(x,y,z) = \O{R}_J(x,y,z,z) \]
\[ \O{R}_C(x,y) = \O{R}_F(x,y,y) \]
and
\[ \O{R}_{-a}(\vec{b}; \vec{z}) = \O{R}_{-a}(b_1,\dots,b_n; z_1,\dots,z_n) = \frac{1}{B(a,a')}\int_0^\infty t^{a'-1} \prod_{j=1}^n(t+z_j)^{-b_j}\,dt \]
where $a' = -a + \sum_{j=1}^n b_j$.

%%%%%%%%%%%%%%%%%%%%%%%%%%%%%%%%%%%%%%%%
\subsection{Bulirsch's elliptic integrals}
For $a,b,p\in\RR$, $p\neq0$, $k_c\neq0$,
\[ \O{cel}(k_c,p,a,b) = \int_0^{\pi/2} \frac{a \cos^2\theta + b\sin^2\theta}{\cos^2\theta + p\sin^2\theta} \frac{1}{\sqrt{\cos^2\theta+k_c\sin^2\theta}}\,d\theta \]
\[ \O{el}_2(x,k_c,a,b) = \int_0^{\atan x} \frac{a+b\tan^2\theta}{\sqrt{(1+\tan^2\theta)(1+k_c\tan^2\theta)}},d\theta \]
Noting $k_c=k'$, $p=1-\al^2$, $x=\tan\phi$, then
\[ \O{el}_1(x,k_c) = F(\phi,k) \]
If $x^2\neq-1/p$,
\[ \O{el}_3(x,k_c,p) = \int_0^{\atan x}\frac{d\theta}{(\cos^2\theta+p\sin^2\theta)\sqrt{\cos^2\theta+k_c^2\sin^2\theta}} = \Pi(\phi, \al^2, k) \]

Let $r=1/x^2$, then
\[ \O{cel}(k_c,p,a,b) = a \O{R}_F(0,k_c^2,1) + \frac13(b-p a)\O{R}_J(0,k_c^2,1,p) \]
\[ \O{el}_1(x,k_c) = \O{R}_F(r, r+k_c^2, r+1) \]
\[ \O{el}_2(x,k_c,a,b) = a \O{el}_1(x,k_c) + \frac13(b-a)\O{R}_D(r,r+k_c^2,r+1) \]
\[ \O{el}_3(x,k_c,p) = \O{el}_1(x,k_c) + \frac13(1-p)\O{R}_J(r,r+k_c^2,r+1,r+p) \]

%%%%%%%%%%%%%%%%%%%%%%%%%%%%%%%%%%%%%%%%
\subsection{Jacobian elliptic functions}
We have implicit parameter $k$ in the following
\[ \O{sn}(u) = x \text{\quad s.t.\quad} u = F(k,x) = \int_0^x \frac{dt}{\sqrt{(1-t^2)(1-k^2t^2)}} \]
\[ \O{cn}(u) = \sqrt{1-\O{sn}^2 u} \]
\[ \O{dn}(u) = \sqrt{1-k^2\O{sn}^2 u} \]
I.e. $\O{sn}^2 + \O{cn}^2 = 1$ and $\O{dn}^2 + k^2\O{sn}^2 = 1$.
other inverse?

\begin{implementation}
From~\cite{DLMF};
\begin{itemize}
\item Three basic techniques to compute: use $\theta$ functions, use AGM, use Landen transformations and then series in $k$
\item Theta functions: let $\zeta=\frac{\pi z}{2 K(k)}$
  \begin{eqnarray*}
  \O{sn}(z,k) &=& \frac{\theta_3(0,q)}{\theta_2(0,q)} \frac{\theta_1(\zeta,q)}{\theta_4(\zeta,q)} = \frac{1}{ns} \\
  \O{cn}(z,k) &=& \frac{\theta_4(0,q)}{\theta_2(0,q)} \frac{\theta_2(\zeta,q)}{\theta_4(\zeta,q)} = \frac{1}{nc} \\
  \O{dn}(z,k) &=& \frac{\theta_4(0,q)}{\theta_3(0,q)} \frac{\theta_3(\zeta,q)}{\theta_4(\zeta,q)} = \frac{1}{nd} \\
  \O{sd}(z,k) &=& \frac{\theta_3(0,q)^2}{\theta_2(0,q)\theta_4(0,q)} \frac{\theta_1(\zeta,q)}{\theta_3(\zeta,q)} = \frac{1}{ds} \\
  \O{cd}(z,k) &=& \frac{\theta_3(0,q)}{\theta_2(0,q)} \frac{\theta_1(\zeta,q)}{\theta_2(\zeta,q)} = \frac{1}{dc} \\
  \O{sc}(z,k) &=& \frac{\theta_3(0,q)}{\theta_4(0,q)} \frac{\theta_1(\zeta,q)}{\theta_2(\zeta,q)} = \frac{1}{cs} \\
  \end{eqnarray*}
\item AGM:
  \begin{enumerate}
  \item Let $a_0=1$, $b_0=k'\in(0,1)$
  \item Compute $a_N, b_N, c_N$ via AGM
  \item Let $\phi_N = 2^N a_N x$
  \item and $\phi_{n-1} = \tfrac12(\phi_n + \asin( \frac{c_n}{a_n} \sin \phi_n))$
  \item Then $\O{sn}(x,k)=\sin\phi_0$, $\O{cn}(x,k)=\cos\phi_0$, and $\O{dn}(x,k) = \cos\phi_0 / \cos(\phi_1-\phi_0)$
  \end{enumerate}
\end{itemize}
\end{implementation}

%%%%%%%%%%%%%%%%%%%%%%%%%%%%%%%%%%%%%%%%%%%%%%%%%%%%%%%%%%%%%%%%%%%%%%%%%%%%%%%%
\section{Parabolic cylinder functions}
\[ \O{D}_\nu(z) = \O{U}(a,z) =  \qquad\nu=-a-\tfrac12\]
\[ \O{V}_\nu(z) = \O{V}(a,z) =  \qquad\nu=-a-\tfrac12\]
\[ \O{W}(a,\pm z) = \]

{\tt Dnu, Vnu, Wa\_p/m, dDnu, ..., Ua, Va, ...}

%%%%%%%%%%%%%%%%%%%%%%%%%%%%%%%%%%%%%%%%
\subsection{Lebedev}
(According to Lebedev~\cite{lebedev}, solutions to $u'' + (2\nu + 1 - z^2)u = 0$ are the {\em parabolic cylinder functions}.)

According to Lebedev: solutions to $v'' - 2zv' + 2\nu v=0$ are {\em Hermite functions}
\[ \O{H}_\nu(z) = \frac{2^\nu \Gam(1/2)}{\Gam((1-\nu)/2)} \Phi[-\nu/2, 1/2; z^2]
  + \frac{2^\nu \Gam(-1/2)}{\Gam(-\nu/2)} \Phi[(1-\nu)/2, 3/2; z^2] \]
And if $\nu=n=0,1,2,\dots$ then we get the Hermite Polynomials

%%%%%%%%%%%%%%%%%%%%%%%%%%%%%%%%%%%%%%%%
\subsection{NIST handbook}
According to NIST handbook~\cite{nist}:
Parabolic cylinder functions are solutions of the differential equation $w'' + (az^2+bz+c)w = 0$
with three distinct standard forms:
\begin{enumerate}
\item $w''-(z^2/4 + a)w=0$ with solutions $U(a,\pm z)$, $V(a,\pm z)$, $\bar{U}(a,\pm x)$ (not complex conjugate), and $U(-a,\pm\ii z)$;
  for real values of $z(=x)$, numerically satisfactory pairs of solutions are $U(a,x)$ and $V(a,x)$ for $x>0$ and
  $U(a,-x)$, $V(a,-x)$ for $x<0$.
\item $w''+(z^2/4-a)w=0$ with solutions $W(a,\pm x)$;
  for all real $x$, a numerically satisfactory pair is $W(a,x)$, $W(a,-x)$.
\item and $w''+(\nu+1/2-z^2/4)w=0$ with solutions $D_\nu(\pm z)$ where $D_\nu(z)=U(-1/2-\nu,z)$ (Whittaker function);
\end{enumerate}
In $\CC$, for $j=0,1,2,3$, a numerically satisfactory pair of solutions is given by
$U((-)^{j-1}a,(-\ii)^{j-1}z)$ and $U(-)^ja,(-\ii)^jz)$ in the half-plane $\tfrac14(2j-3)\pi\leq\O{ph}z\leq\tfrac14(2j+1)\pi$.

We can express $U$ and $V$ in terms of $D$ via
\begin{eqnarray*}
U(a,x) &=& D_{-a-1/2}(x) \\
V(a,x) &=& \frac{\Gam(a+\tfrac12)}{\pi}\left(\sin(\pi a) D_{-a-1/2}(x) + D_{-a-1/2}(-x)\right)
\end{eqnarray*}

All solutions are entire functions of $z$ and $a$ or $\nu$.

Values at $z=0$ are given by
\begin{eqnarray*}
U(a,0)  &=& \frac{\sqrt\pi}{2^{a/2+1/4}\Gam(\tfrac34+\tfrac12a)} \\
U'(a,0) &=& -\frac{\sqrt\pi}{2^{a/2-1/4}\Gam(\tfrac14+\tfrac12a)} \\
V(a,0)  &=& \frac{\pi 2^{a/2+1/4}}{\Gam^2(\tfrac34-\tfrac12a)\Gam(\tfrac14+\tfrac12a)}
    = \frac{2^{1/4 + a/2} \sin(\pi(\tfrac34 - \tfrac12a))}{\Gam(\tfrac34 - \tfrac12a)} \\
V'(a,0) &=& \frac{\pi 2^{a/3+3/4}}{\Gam^2(\tfrac14-\tfrac12a)\Gam(\tfrac34+\tfrac12a)}
    = \frac{2^{3/4 + a/2} \sin(\pi(\tfrac14 - \tfrac12a))}{\Gam(\tfrac14 - \tfrac12a)} \\
W(a,0) &=& 2^{-3/4}\left|\frac{\Gamma(\tfrac14+\tfrac12a\ii}{\Gamma(\tfrac34+\tfrac12a\ii}\right|^{1/2} \\
W'(a,0) &=& -2^{-1/4}\left|\frac{\Gamma(\tfrac34+\tfrac12a\ii}{\Gamma(\tfrac14+\tfrac12a\ii}\right|^{1/2} \qquad\text{?sign?}
\end{eqnarray*}
{\tt initial\_value\_U, ...}

Series expansions converging for all $z$ are given by
\begin{eqnarray*}
U(a,z) &=& U(a,0)u_1(a,z) + U'(a,0)u_2(a,z) \\
V(a,z) &=& V(a,0)u_1(a,z) + V'(a,0)u_2(a,z) \\
W(a,z) &=& W(a,0)w_1(a,z) + W'(a,0)w_2(a,z)
\end{eqnarray*}
where
\begin{eqnarray*}
u_1(a,z)
  &=& e^{-z^2/4} \O{M}(\tfrac14 + \tfrac12a, \tfrac12, \frac{z^2}{2}) 
      = e^{-z^2/4}\left(1+(a+\tfrac12)\frac{z^2}{2!}+(a+\tfrac12)(a+\tfrac52)\frac{z^4}{4!}+\cdots\right) \\
  &=& e^{z^2/4} \O{M}(\tfrac14 - \tfrac12a, \tfrac12, \frac{-z^2}{2}) 
      = e^{z^2/4}\left(1+(a-\tfrac12)\frac{z^2}{2!}+(a-\tfrac12)(a-\tfrac52)\frac{z^4}{4!}+\cdots\right) \\
u_2(a,z)
  &=& z e^{-z^2/4} \O{M}(\tfrac34 + \tfrac12a, \tfrac32, \frac{z^2}{2}) 
      = e^{-z^2/4}\left(z+(a+\tfrac32)\frac{z^3}{3!}+(a+\tfrac32)(a+\tfrac72)\frac{z^5}{5!}+\cdots\right) \\
  &=& z e^{z^2/4} \O{M}(\tfrac34 - \tfrac12a, \tfrac32, \frac{-z^2}{2}) 
      = e^{z^2/4}\left(z+(a-\tfrac32)\frac{z^3}{3!}+(a-\tfrac32)(a-\tfrac72)\frac{z^5}{5!}+\cdots\right)
\end{eqnarray*}
or we have the series
\begin{eqnarray*}
u_1(a,z) &=& 1 + a\frac{z^2}{2!} + (a^2 + \tfrac12)\frac{z^4}{4!} + (a^3 + \tfrac72a)\frac{z^6}{6!} + \cdots \\
u_2(a,z) &=& z + a\frac{z^3}{3!} + (a^2 + \tfrac32)\frac{z^5}{5!} + (a^3 + \tfrac{13}2a)\frac{z^7}{7!} + \cdots \\
\end{eqnarray*}
where the coefficients of $x^n/n!$ are given by $a_{n+2} = a a_n + \tfrac14n(n-1)a_{n-2}$.


Also,
\begin{eqnarray*}
w_1(a,z) &=& \sum_{n=0}^\infty \al_n(a) \frac{x^{2n}}{(2n)!} \\
w_2(a,z) &=& \sum_{n=0}^\infty \beta_n(a) \frac{x^{2n+1}}{(2n+1)!}
\end{eqnarray*}
with $\al_0(a)=\beta_0(a)=1$, $\al_1(a)=\beta_1(a)=a$, and
\begin{eqnarray*}
\al_{n+2} = a \al_{n+1} - \frac12(n+1)(2n+1)\al_n \\
\beta_{n+2} = a \beta_{n+1} - \frac12(n+1)(2n+3)\beta_n
\end{eqnarray*}
(Can get recursions for the full series terms directly: $\hat\al_n = \al_n/(2n)!$ then
$\hat\al_{n+2} = \frac{a \hat\al_{n+1} - \frac14\hat\al_n}{(2n+3)(2n+4)}$, etc.

Recurrence relations are
\[ z U(a,z) - U(a-1,z) + (a+\tfrac12)U(a+1,z) = 0 \]
\[ z V(a,z) - V(a+1,z) + (a-\tfrac12)V(a-1,z) = 0 \]

Asymptotic expansions for $z\to\infty$ ($\delta$ is an arbitrarily small positive constant)
\begin{eqnarray*}
U(a,z) &\sim& e^{-z^2/4} z^{-a-1/2} \sum_{s=0}^\infty(-)^s\frac{(\tfrac12+a)_{2s}}{s!(2z^2)^s} \qquad |\O{ph}z|\leq\tfrac34\pi-\delta \\
U(a,z) &\sim& e^{-z^2/4}z^{-a-1/2}\sum_{s=0}^\infty(-)^s\frac{(\tfrac12+a)_{2s}}{s!(2z^2)^s} 
  \pm\ii\frac{\sqrt{2\pi}}{\Gam(\tfrac12+a)}e^{\mp\ii\pi a}e^{z^2/4}z^{a-1/2}\sum_{s=0}^\infty\frac{(\tfrac12-a)_{2s}}{s!(2z^2)^s} \\
  &&\qquad \tfrac14\pi+\delta\leq\pm\O{ph}z\leq\tfrac54\pi-\delta \\
V(a,z) &\sim& \sqrt{\frac2\pi}e^{z^2/4}z^{a-1/2} \sum_{s=0}^\infty\frac{(\tfrac12-a)_{2s}}{s!(2z^2)^s} \qquad |\O{ph}z|\leq\tfrac14\pi-\delta \\
V(a,z) &\sim& \sqrt{\frac2\pi}e^{z^2/4}z^{a-1/2}\sum_{s=0}^\infty\frac{(\tfrac12-a)_{2s}}{s!(2z^2)^s} 
  \pm\ii\frac{1}{\Gam(\tfrac12-a)}e^{-z^2/4}z^{-a-1/2}\sum_{s=0}^\infty(-)^s\frac{(\tfrac12+a)_{2s}}{s!(2z^2)^s} \\
  &&\qquad -\tfrac14\pi+\delta\leq\pm\O{ph}z\leq\tfrac34\pi-\delta \\
\end{eqnarray*}

\begin{implementation}
Initial testing seems to indicate that, at least for small $a$, the series expansions can give decent results, though
not necessarily a full 15 digits (with $a=-20$, looks like only 13 digits, even for small $z$; with $a=-2$, getting around 15 digits).
\end{implementation}



%%%%%%%%%%%%%%%%%%%%%%%%%%%%%%%%%%%%%%%%%%%%%%%%%%%%%%%%%%%%%%%%%%%%%%%%%%%%%%%%
\section{Lambert functions}
The Lambert $W$ function is defined to be a solution to $W(z)e^{W(z)}=z$, i.e.
\[ \O{W}(z) = \rho \text{\quad s.t.\quad} \rho e^\rho = z \]
Taking logarithms gives $\log(W(z))) + W(z) = \log(z)$, that is, to solve $y+\log y=z$
we can take $y=W(-e^z)$.
%Similarly to solve $x\log x=z$, take $x=W(e^z)$. %% I DON'T THINK THIS IS CORRECT
\[ \O{W}_{[k]}(z) = \text{$k$'th branch} \]
The related ``tree function'' is given by
\[ T(v) = -W(-v) = \sum_{n=1}^\infty \frac{n^{n-1} v^n}{n!} \]

The Lambert function has two real branches $W_{[0]}:[-1/e,\infty)\to[-1,\infty)$
and $W_{[1]}:[-1/2,0)\to(-\infty,-1]$.

For the principal branch we have the following power-series (with radius of convergence $1/e$)
\[ W_{[0]}(z) = \sum_{n=1}\frac{(-n)^{n-1}}{n!} z^n \]

Note that the derivative satisfies (where $x\neq0$ for the second equality),
\[ W'(x) = \frac{1}{e^{W(x)}(1+W(x))} = \frac{W(x)}{x(1+W(x))} \]
More generally, we have, for $n\geq1$,
\[ \frac{d^n}{dx^n} W(x) = \frac{e^{n W(x)}}{(1+W(x))^{2n-1}} p_n(W(x)) \]
where $p_1(w)=1$ and
\[ p_{n+1}(w) = (-nw + 3n-1) p_{n}(w) + (1+w)p_n'(w) \]

For the integral, we have
\[ \int W(x)\,dx = (W^2(x) - W(x) + 1)e^{W(x)} + C = x(W(x)-1+\frac{1}{W(x)})+C \]

Using Halley's method for solving $we^w=z$, we use the iteration
\[ w' = w - \frac{w e^w - z}{e^w(w+1)- \frac{(w+2)(we^w-z)}{2w+2}} \]

Asymptotics for all non-principal branches, both at $0$ and $\infty$ are given by:
\[ W(z) = \O{Log} z - \log\O{Log} z + \sum_{k=0}^\infty\sum_{m=1}^\infty c_{km} (\log\O{Log} z)^m (\O{Log} z)^{-k-m} \]
where we have (with Stirling cycle numbers)
\[ c_{km} = \frac{(-)^k}{m!}\begin{bmatrix}k+m\\k+1\end{bmatrix} \]

%%%%%%%%%%%%%%%%%%%%%%%%%%%%%%%%%%%%%%%%%%%%%%%%%%%%%%%%%%%%%%%%%%%%%%%%%%%%%%%%
\section{Hypergeometric, Meijer G functions}
$\mathcal{G}$-function, $\mathcal{H}$-function, Macdonald $E$ function, Carr $\phi[\al,\beta\gam|x,y]$ function,

{\tt 0f0, 1f0, 0f1, 1f1, 1f2, 2f1, 2f2, 3f2, 3f3, 2f3, pfq, ...}
Confluent $M$ and $U$ (``first'' and ``second'' kinds) [have alternative ode solutions for all hypergeom funcs?]

Utility functions for hypergeometric-type {\em series}(?): odd/even parts of series, alternating odd/even parts, 
mixed odd/even ($x^n y^{2n}$) etc.?

``Gamma series'' variations (and modifications as above): (?)
\[ \sum_{n=0}^\infty \frac{\Gam(a_1+n)\cdots\Gam(a_p+n)}{\Gam(b_1+n)\cdots\Gam(b_q+n)} x^n \]

For example,
\[ \sum_{n=0}^\infty\frac{x^n}{\Gam(\al+n)} = \frac{1}{\Gam(\al)}\sum_{n=0}^\infty\frac{x^n}{(\al)_n}
    = \frac{1}{\Gam(\al)} \Hyper{1}{1}{1}{\al}{x} \]

Hybrid series, e.g. $\sum_{n=0}^\infty \frac{(a)_n}{\Gam(b+n)}\frac{x^n}{n!}$ etc. (works for $b=0,-1,-2,...$ also!)

Hypergeometric {\em functions} vs. hypergeometric {\em series}

Scaled versions...

Define
\[ \Hyper{2}{1}{\al, \beta}{\gam}{z} = \sum_{n=0}^\infty \frac{(\al)_n (\beta)_n}{(\gam)_n} \frac{z^n}{n!}
    = \frac{\Gam(\gam)}{\Gam(\al)\Gam(\beta)}\sum_{n=0}^\infty \frac{\Gam(\al+n)\Gam(\beta+n)}{\Gam(\gam+n)} \frac{x^n}{n!} \]
This series is defined for $|z|<1$ and $\gam\neq0,-1,-2,\dots$.  It can be extended to an analytic function
on $\CC\setminus[1,\infty]$ via the integral
\[ \Hyper{2}{1}{\al, \beta}{\gam}{z} = \frac{\Gam(\gam)}{\Gam(\beta)\Gam(\gam-\beta)}\int_0^1 t^{\beta-1}(1-t)^{\gam-\beta-1}(1-tz)^{\al}\,dt \]
for $\Re\gam>\Re\beta>0$ and $|\arg 1-z|<\pi$.

The hypergeometric polynomials are given by $F(-n,\beta;\gam;z)$.

Note that Lebedev defines the hypergeometric function of the second kind as $G(\al,\beta;\gam;z) = \dots$ which is the
other solution of the hypergeometric equation $z(1-z)u'' + [\gam-(\al+\beta+1)z]u' - \al\beta u=0$ (for $|z|<1$.)

The generalized hypergeometric function is:
\[ \Hyper{p}{q}{\vec{\al}}{\vec{\gam}}{z} = \sum_{n=0}^\infty \frac{(\al_1)_n\cdots(\al_p)_n}{(\gam_1)_n\cdots(\gam_q)_n} \frac{z^n}{n!} \]

\[ \O{M}(\al, \beta; z) = \Hyper{1}{1}{\al}{\beta}{z} = \sum_{n=0}^\infty \frac{(\al)_n}{(\beta)_n} \frac{z^n}{n!} \]
FOr $\beta\neq0,-1,-2,\dots$, $|Z|<\infty$ (defines an entire function of $z$ on $\CC$).
(In Lebedev's notation we have $\Phi(\al,\beta;z)$.)
(Lebedev defines the confluent hypergeometric function of the second kind as $\Psi(\al,\gam;z) = \dots$.)

\[ \O{U}(\al, \beta; z) = \frac{\pi}{\sin \pi\beta}\left( \frac{\O{M}(\al, \beta; z)}{\Gam(\al-\beta+1)\Gam(\beta)}
  - z^{1-\beta}\frac{\O{M}(\al-\beta+1, 2-\beta; z)}{\Gam(\al) \Gam(2-\beta)} \right) \]

%%%%%%%%%%%%%%%%%%%%%%%%%%%%%%%%%%%%%%%%
\subsection{Whittaker functions}
\[ \O{M}_{k,\mu}(z) = z^{\mu+1/2}e^{-z/2} \O{M}(\mu+\tfrac12-k, 2\mu+1; z) \]
\[ \O{W}_{k,\mu}(z) = \frac{\Gam(-2\mu)}{\Gam(1/2 - \mu - k)}\O{M}_{k,\mu}(z) + \frac{\Gam(2\mu)}{\Gam(1/2 + \mu - k)}\O{M}_{k,-\mu}(z) \]

%%%%%%%%%%%%%%%%%%%%%%%%%%%%%%%%%%%%%%%%%%%%%%%%%%%%%%%%%%%%%%%%%%%%%%%%%%%%%%%%
\section{Zeta functions}
Need {\em zeros} also

Hurwitz Zeta

Lerch Phi

%%%%%%%%%%%%%%%%%%%%%%%%%%%%%%%%%%%%%%%%
\subsection{Riemann zeta function}
\[ \O{\zeta}(s) = \sum_{n=1}^\infty n^{-s} \qquad \Re s>1 \]
\[ \O{\eta}(s) =  \sum_{n=1}^\infty (-)^{n-1}n^{-s} = (1-2^{1-s})\zeta(s) \qquad \Re s>1 \]
\[ \O{\lambda}(s) =  \sum_{n=1}^\infty (2n-1)^{-s} = (1-2^{-s})\zeta(s) \qquad \Re s>1 \]
\[ \O{\beta}(s) =  \sum_{n=1}^\infty (-)^{n-1}(2n-1)^{-s} \qquad \Re s>1 \]
We should have implementation of $\zeta(s)-1$ for large $s$, (especially for $s\in\NN$).
(and for other such funcs) [also $\zeta(s)-2^{-s}$, etc.??]
Also, a version of just $zeta(\tfrac12 + \ii t)$ for real $t$...

Reflection:
\[ \zeta(1-s) = \frac{2}{(2\pi)^s}\cos(s\pi/2)\Gam(s)\zeta(s) \]
\[ \zeta(s) = 2(2\pi)^{s-1}\sin(s\pi/2)\Gam(1-s)\zeta(1-s) \]
we can write this as $\xi(s) = \xi(1-s)$ where we define
\[ \xi(s) = \frac{s(s-1)}{2}\Gam(s/2)\pi^{-s/2}\zeta(s) \]

\[ \zeta(-n) = -\frac{B_{n+1}}{n+1} \]

\begin{implementation}
\begin{itemize}
\item[\bf(A)] For all $s\in\CC\setminus\{1\}$,
  \[ \zeta(s) = \frac12 + \frac{1}{s-1} + 2\int_0^\infty\frac{\sin(s \atan t)}{(1+t^2)^{s/2}(e^{2\pi t}-1)}\,dt \]
\item[\bf(B)] For $s\in\CC$,
  \[ \pi^{-s/2}\Gam(s/2)\zeta(s) = -\frac{1}{s} - \frac{1}{1-s} + \frac12\int_1^\infty(t^{(1-s)/2}+t^{s/2})(\theta_3(e^{-\pi t})-1)\frac{dt}{t} \]
\item[\bf(B)] Following from {\bf(B)}, we get
  \[ \zeta(s)\Gam(s/2) = \frac{\pi^{s/2}}{s(s-1)} + \sum_{n=1}^\infty n^{-s}\Gam(\frac{s}{2}, \pi n^2)
      + \pi^{s-1/2}\sum_{n=1}^\infty n^{s-1}\Gam(\frac{1-s}{2}, \pi n^2) \]
\end{itemize}

\cite{DLMF}~claims that a principal approach is using (Euler-Maclaurin)
\[ \zeta(s) = \sum_{k=1}^N\frac{1}{k^s} + \frac{N^{1-s}}{s-1} - \frac{N^{-s}}{2} + \sum_{k=1}^n\binom{s+2k-2}{2k-1}\frac{B_{2k}}{2k} N^{1-s-2k}
    - \binom{s+2n}{2n+1}\int_N^\infty\frac{\widetilde{B}_{2n+1}(x)}{x^{s+2n+1}}\,dx \]
for $\Re s>-2n; n,N=1,2,3,\dots$  (A quick test shows that just the first 2 terms after the initial sum improve the accuracy significantly...)
[INCLUDE EXPLICIT FIRST FEW TERMS...]

Notes on using the Euler-Maclaurin expansion above:
\begin{itemize}
\item stopping criterion is a little bit tricker (but in practice not a bit problem)
\item improves convergence speed significantly, especially for small $s>1$, chart below gives some sample data
    (running with long doubles, indicating number of extra E-M terms): gives terms needed to convergence and achieved digits of accuracy
\end{itemize}
\end{implementation}
{\small
\[\begin{array}{c|ccccccc} 
s&\text{raw series}&\text{2-terms + n=0}&\text{2-terms + n=1}&\text{2-terms + n=2}&\text{2-terms + n=3}&\text{2-terms + n=4}&\text{2-terms + n=5}\\\hline
2  & >9999/3.8 & >9999/12.6 & 1089/16.5 & 201/17.6 & 77/18.1 & 40/18.6 & 26/18.3 \\
5  & >9999/15.2 & 910/16.7 & 204/17.5 & 78/17.7 & 48/18.6 & 33/18.6 & 24/18.7 \\
10 & 170/18 & 83/19.1 & 40/18.7 & 34/18.5 & 25/19 & 22/18.5 & 23/18.5 \\
40 & 8/full & 8/full & 8/full & 8/full & 8/full & 9/full & 8/full
\end{array}\]
}

Alternative approach for computation of Riemann zeta function.
First, define $\zeta_N(s) = \sum_{r=1}^{N-1} r^{-s}$, then we have
\[ \zeta(s) = \zeta_N(s) + \frac{N^{1-s}}{s-1} + \frac{N^{-s}}{2} + \sum_{j=1}^K T_j(N,s) + R(N,K,s) \]
where we have
\[ T_j(N,s) = \frac{B_{2j}}{(2j)!} \frac{s(s+1)\cdots(s+2j-2)}{N^{s+2j-1}} \]
Then we define our algorithm for computation of $\zeta(s;\epsilon)$ (with accuracy $\epsilon>0$) via:
\begin{itemize}
  \item if $s=0$ then return $-1/2$
  \item else if $s=1$ then return $\infty$
  \item else if $\Re(s)<0$, then return $2(2\pi)^{s-1} \sin(s\pi/2) \Gamma(1-s) \zeta(1-s;\epsilon)$
  \item else
  \begin{itemize}
    \item let $\al=q/p$ where $p$ is the cost of computing $T_j(N,s)$ and $q$ is the cost of $j^{-s}$
    \item let $k$ be the positive integer closets to the solution of $\frac{d}{dk} H_\al(k)=0$
    \item let $n=\lceil n(k) \rceil$
    \item return $\zeta_n(s) + \frac{n^{1-s}}{s-1} + \frac{n^{-s}}{2} + \sum_{j=1}^k T_j(n,s)$
  \end{itemize}
\end{itemize}
Now, we have $H_\al(k) = \al\cdot n(k) + k$ with $\al>0$, where $n(k)=H_r(k)$ for $s\in\RR$ and $n(k)=H_c(k)$ for $s\in\CC\setminus\RR$,
with
\[ H_r(k) = \left( \frac{2}{\epsilon\Gam(s)} \frac{\Gam(s+2k-1)}{(2\pi)^{2k}} \right)^{1/(s+2k-1)} + k \]
and if $s=\sigma+\ii\tau$,
\[ H_c(k) = \frac{|s+2k-1}{e} \left(\frac{2\sqrt{2\pi}}{\epsilon|\Gam(s)|e^{\tau\arg(s)}} \frac{\sqrt{|s+2k-1|}}{(2\pi e)^{2k}}\right)^{1/(\sigma+2k-1)}  + k
  \qquad\text{??}\]
This should work very well.  From the paper [????] (TODO: add reference here!)

\begin{implementation}
Notes on the Dirichlet Beta function $\beta(\nu)$:
\begin{itemize}
\item $\beta(-n) = \tfrac12 E_n$ for $n=0,1,2,\dots$
\item $\beta(n) = (\tfrac\pi2)^n\frac{|E_{n-1}|}{2(n-1)!}$ for $n=1,3,5,\dots$
\item $\beta(\nu) = \frac{1}{2\Gam(\nu)}\int_0^\infty t^{\nu-1}\O{sech}(t)\,dt \qquad(\nu>0)$
\item $\beta(1-\nu) = \frac{\Gam(\nu)\sin(\nu\pi/2)}{(\pi/2)^\nu}\beta(\nu) \qquad \beta\neq0,-1,-2,\dots$
\item $\beta(\nu) = 4^{-\nu}\left(\zeta(\nu,1/4) - \zeta(\nu,3/4)\right)$ (Hurwitz zeta function)
\item $\beta(\nu) = 2^{-\nu}\eta(\nu,1/2)$ (bivariate eta function - from~\cite{atlas:oldham})
\item It is claimed in~\cite{atlas:oldham} that this is the Euler-Maclaurin expansion, but it works very poorly:
    \[ \beta(\nu) = \lim_{J\to\infty}\left\{  \sum_{j=1}^J \frac{(-)^{j+1}}{(2j-1)^\nu}
        + \sum_{k=1}^K \frac{2^{k-1}(2^k-1)(\nu)_{k-1} B_k}{k! (2J-1)^{k+\nu-1}} \right\}\]
\item A much more effective Euler-Maclaurin expansion is given by
    \begin{eqnarray*}
    \beta(\nu) &=& \sum_{k=0}^{2N}\frac{(-)^k}{(2k+1)^\nu}
            + \frac{(4N+1)^{1-\nu} - (4N+3)^{1-\nu}}{4(\nu-1)} - \frac{(4N+1)^{-\nu} - (4N+3)^{-\nu}}{2} \\
        && {} - \sum_{j=2}^K(-4)^{j-1}(\nu)_{j-1}\frac{B_j}{j!}\left[(4N+1)^{-\nu-j+1} - (4N+3)^{-\nu-j+1}\right]
    \end{eqnarray*}
\end{itemize}
\end{implementation}

%%%%%%%%%%%%%%%%%%%%%%%%%%%%%%%%%%%%%%%%
\subsection{Hurwitz' zeta function}
\[ \zeta(s,a) = \sum_{n=0}^\infty \frac{1}{(n+a)^s} \qquad \Re s>1, a\neq0,-1,-2,\dots \]
(with continuation in s to $\CC$ with only a simple pole at $s=1$).

We have the following simple identities
\[ \zeta(s,a) = \zeta(s,a+1) + a^{-s} \]
\[ \zeta(s,a) = \zeta(s,a+m) + \sum_{n=0}^{m-1} (n+a)^{-s} \]
\[ \frac{\partial}{\partial a}\zeta(s,a) = -s \zeta(s+1,a) \]

We have a series expansion for $s\neq1$, $|a-1|<1$:
\[ \zeta(s,a) = \sum_{n=0}^\infty \frac{\Gam(n+s)}{\Gam(s)} \frac{(1-a)^n}{n!}\zeta(n+s) \]

As $a\to0$ with $s\neq1$ fixed
\[ \zeta(s,a+1) = \zeta(s) - s \zeta(s+1)a + O(a^2) \]

We have the following Euler-Maclaurin-type expansions
\[ \zeta(s,a) = \sum_{n=0}^N(n+a)^{-s} + \frac{(N+a)^{1-s}}{s-1} - s\int_N^\infty \cdots \]
\[ \zeta(s,a) = a^{-s}(\frac12 + \frac{a}{s-1}) - s(s+1)\int_0^\infty \cdots \]
\[ \zeta(s,a) = a^{-s} + (1+a)^{-s}(\frac12 + \frac{1+a}{s-1}) + \sum_{k=1}^\infty\binom{s+2k-2}{2k-1}\frac{B_{2k}}{2k} \frac{1}{(1+a)^{s+2k-1}} - R\cdots \]

%%%%%%%%%%%%%%%%%%%%%%%%%%%%%%%%%%%%%%%%
\subsection{Lerch's transcendent}
\[ \O{\Phi}(z,s,a) = \sum_{n=1}^\infty \frac{z^n}{(a+n)^s} \qquad a\neq0,-1,-2,\dots; |z|<1; (|z|=1 \implies \Re(s)>1) \]
(To implement, try out Euler-Maclaurin...)

We have the following identities
\[ \Phi(z,s,a) = z^m \Phi(z,s,a+m) + \sum_{n=0}^{m-1}\frac{z^n}{(a+n)^s} \]
\[ \Phi(z,s,a) = \frac{1}{\Gam(s)}\int_0^\infty \frac{x^{s-1} e^{-ax}}{1-z e^{-x}}\,dx \]
(for $\Re s>0, \Re a>0, z\in\CC\setminus[1,\infty)$.)

%%%%%%%%%%%%%%%%%%%%%%%%%%%%%%%%%%%%%%%%
\subsection{Dirichlet L-function}
\[ \O{L}(s,\chi) = \sum_{n=1}^\infty \frac{\chi(n)}{n^s} \qquad \Re(s)>1, \text{$\chi$ a Dirichlet character (mod $k$)} \]
(express as sum of zeta-functions, allow a general $k$-periodic character~$\chi$.)

The following is valid for all $s$ if $\chi\neq\chi_1$; otherwise valid for all $s\neq1$:
\[ \O{L}(s,\chi) = k^{-s}\sum_{r=1}^{k-1}\chi(r)\zeta(s,\tfrac{r}{k} \]

%%%%%%%%%%%%%%%%%%%%%%%%%%%%%%%%%%%%%%%%
\subsection{Polylogarithms and related}

%%%%%%%%%%%%%%%%%%%%
\subsubsection{Dilogarithm}
The dilogarithm is defined via
\[ \O{Li}_2(z) = \sum_{n=1}^\infty \frac{z^n}{n^2} \qquad |z|\leq1 \]
and can be analytically continued to the complex plane split along the real line from $+1$ to $\infty$ as
\[ \O{Li}_2(z) = -\int_0^z \ln(1-t) \frac{dt}{t} \qquad z\in\CC\setminus(1,\infty) \]
(Compare also with Spence integral)
{\tt Dilog, Li\_2, Polylog, Li\_n, ...} generalized polylog $\O{Li}_s(z)$...

Noodling around:
\begin{eqnarray*}
\O{Li}_2(z) &=& \sum_{n=1}^\infty \frac{z^n}{n^2} \\
\frac{\O{Li}_2(z^2)}{4} &=& \sum_{n=1}^\infty \frac{z^{2n}}{(2n)^2} \\
\O{Li}_2(z) - \frac{\O{Li}_2(z^2)}{4} &=& \sum_{n=1}^\infty \frac{z^{2n+1}}{(2n+1)^2} \\
\frac{1}{\ii}\left(\O{Li}_2(\ii z) - \frac{\O{Li}_2(-z^2)}{4}\right) &=& \sum_{n=1}^\infty (-)^n \frac{z^{2n+1}}{(2n+1)^2}
\end{eqnarray*}

We can derive the following Euler-Maclaurin expansion for the power-series to get
\[ \O{Li}_2(z) = \sum_{k=1}^N\frac{z^k}{k^2} + \frac{\O{E}_2(-N\ln z)}{N} - \frac{z^N}{2 N^2} - \sum_{j=2}^K \frac{B_j}{j!}f^{(j-1)}(N) + R_k \]
where $\O{E}_2(z)$ is the exponential integral, and $f(x) = z^x/x^2$.  This improves convergence, but the cost of computing the exponential integral
outweighs any benefits.
Note that we have
\begin{eqnarray*}
f'(n) &=& \frac{z^n}{n^2}\left( \ln z - \frac{2}{n} \right) \\
f^{(2)}(n) &=& \frac{z^n}{n^2}\left( (\ln z)^2 - 4\frac{\ln z}{n} + \frac{6}{n^2} \right) \\
f^{(3)}(n) &=& \frac{z^n}{n^2}\left( (\ln z)^3 - 6\frac{(\ln z)^2}{n} + 18\frac{\ln z}{n^2} - \frac{24}{n^3} \right) \\
f^{(4)}(n) &=& \frac{z^n}{n^2}\left( (\ln z)^4 - 8\frac{(\ln z)^3}{n} + 36\frac{(\ln z)^2}{n^2}
    - 96\frac{\ln z}{n^3} + \frac{120}{n^4} \right) \\
f^{(5)}(n) &=& \frac{z^n}{n^2}\left( (\ln z)^5 - 10\frac{(\ln z)^4}{n} + 60\frac{(\ln z)^3}{n^2}
    - 240\frac{(\ln z)^2}{n^3} + 600\frac{\ln z}{n^4} - \frac{720}{n^5} \right)
\end{eqnarray*}

\begin{implementation}
Many identities are useful:
\[ \O{Li}_2(z) + \O{Li}_2(\frac{z}{z-1}) = -\frac12(\ln 1-z)^2 \qquad z\in\CC\setminus[1,\infty) \]
\[ \O{Li}_2(z) + \O{Li}_2(\frac{1}{z}) = -\frac{\pi^2}{6} - \frac12(\ln{-z})^2 \qquad z\in\CC\setminus[0,\infty) \]
\[ \O{Li}_2(x) + \O{Li}_2(1-x) = \frac{\pi^2}{6} - (\ln x)(\ln 1-x) \qquad 0<x<1 \]
\[ \O{Li}_2(z^m) = m\sum_{k=0}^{m-1}\O{Li}_2(z e^{2\pi\ii k/m}) \qquad |z|<1, m=1,2,3,\dots \]
When $z=e^{\ii\theta}$, the original definition becomes
\[ \O{Li}_2(e^{\ii\theta}) = \sum_{n=1}^\infty\frac{\cos n\theta}{n^2} + \ii\sum_{n=1}^\infty\frac{\sin n\theta}{n^2}
    = (\frac{\pi^2}{6} - \frac{\pi\theta}{2} + \frac{\theta^2}{4}) - \ii\int_0^\theta \ln(2\sin(x/2))\,dx \]
(and notice the final integal is Clausen's integral).
\end{implementation}

\begin{titled-frame}{{\tt dilog(z)}${}=\O{Li}_2(z)$}
\small%
\begin{verbatim}
## -*- texinfo -*-
## @deftypefn {Function File} {@var{res} =} sf_dilog (@var{z})
## Compute the dilogarithm $Di_2(z) = \sum_(n=1)^\infty z^n/n^2$
## @end deftypefn
function res = sf_dilog(z)
  if (nargin < 1) print_usage; endif
  res = zeros(size(z));
  for n = 1:prod(size(z))
    res(n) = sf_dilog_1(z(n));
  endfor
endfunction
function res = sf_dilog_1(z)
  # branch cut on positive real axis [1,+\infty)
  if (imag(z)==0 && z>=1) res = NaN; return; endif

  if (imag(z)==0 && z>1/2)
    # for efficiency for real values
    # L(z) + L(1-z) = pi^2/6 - (ln z)(ln 1-z)
    res = pi^2/6 - sf_log(z)*sf_log_p1(-z) - power_series(1-z);
  elseif (abs(z) <= 2/3)
    # direct series (radius of convergence for |z|<1)
    res = power_series(z);
  elseif (abs(z) >= 3/2)
    # reflect large values to small
    # L(z) + L(1/z) = -pi^2/6 - ln(-z)^2/2
    res = -pi^2/6 - sf_log(-z)^2/2 - power_series(1/z);
  else
    # clean up the rest
    # L(z) + L(z/(z-1)) = -ln(1-z)^2/2
    res = -sf_log(1-z)^2/2 - power_series(z/(z-1));
  endif
endfunction
function res = power_series(z)
  res = 0.0; n = 1; term = 1.0;
  do
    term *= z;
    old_res = res;
    res += term / (n^2);
    ++n; if (n>999) break; endif
  until (res == old_res)
endfunction
\end{verbatim}
\end{titled-frame}


%%%%%%%%%%%%%%%%%%%%
\subsubsection{Polylogarithm}
For real or complex $s$ and $z$, the polylogarithm is defined by
\[ \O{Li}_s(z) = \sum_{n=1}^\infty \frac{z^n}{n^s} \]
For fixed $s\in\CC$, the series defines an analytic function of $z$ for $|z|<1$.  The series also converges
when $|z|=1$ if $\Re s>1$.  (For other values of $z$, define by analytic continuation).

We have the integral representation for $\Re s>0$ and $|\O{ph} 1-z|<\pi$, or $\Re s>1$ and $z=1$:
\[ \O{Li}_s(z) = \frac{z}{\Gam(s)} \int_0^\infty \frac{x^{s-1}}{e^x - z}\,dx \]

Finally we have the properties
\[ \O{Li}_s(z) = \Gam(1-s)(\ln 1/z)^{s-1} + \sum_{n=0}^\infty \zeta(s-n)\frac{(\ln z)^n}{n!}
    \qquad |\ln z|<2\pi, s\neq1,2,3,\dots \]
and
\[ \O{Li}_s(e^{2\pi\ii a}) + e^{\pi\ii s}\O{Li}_s(e^{-2\pi\ii a}) = \frac{(2\pi)^s e^{\pi\ii s/2}}{\Gam(s)}\zeta(1-s, a) \]
valid when $\Re s>0$, $\Im a>0$; or $\Re s>1$, $\Im a=0$.

We can derive the Euler-Maclaurin expansion as follows:
\begin{eqnarray*}
  \sum_{n=a}^\infty\frac{z^n}{n^s}
    &=& \int_a^\infty\frac{z^t}{t^s}\,dt - \frac{z^a}{2a^s}
      + \sum_{i=2}^\infty\frac{B_i}{i!}\left[-\frac{d^{i-1}}{dn^{i-1}} \frac{z^n}{n^s}\right|_{n=a} \\
    &=& -a^{1-s}\Gam(1-s,a\log z)(-a\log z)^{s-1} -\frac{z^a}{2a^s}
      - z^a\sum_{m=1}^\infty\frac{B_{2m}}{(2m)!}\left[ \sum_{k=0}^{2m-1}(-)^k (s)_k a^{-s-k} (\log z)^{2m-1-k} \right]
\end{eqnarray*}


%%%%%%%%%%%%%%%%%%%%
\subsubsection{Fermi-Dirac and Bose-Einstein integrals}
\[ \O{FD}(k,\eta,\theta) = \int_0^\infty \frac{x^k (1+\theta x/2)^{1/2}}{e^{x-\eta} + 1} \,dx \qquad k>-1; \theta\geq0\]

The DLMF~\cite{DLMF} defines the Fermi-Dirac integrals:
\[ \O{F}_s(x) = \frac{1}{\Gam(s+1)} \int_0^\infty \frac{t^s}{e^{t-x} + 1}\,dt \qquad(s>-1) \]
and the Bose-Einstein integrals:
\[ \O{G}_s(x) = \frac{1}{\Gam(s+1)} \int_0^\infty \frac{t^s}{e^{t-x} - 1}\,dt \qquad(s>-1, x<0; \text{\ or\ } s>0,x\leq0) \]
(``Sometimes the Gamma factor is omitted.'')

In terms of polylogarithms we have: $\O{F}_s(x) = -\O{Li}_{s+1}(-e^{x})$ and $\O{G}_s(x) = \O{Li}_{s+1}(e^{x})$.

\begin{implementation}
We can integrate the definition directly using numerical quadrature ...

From~\cite{approximation-fermi-dirac} we have:
\begin{itemize}
\item Start with an expansion for $\Re(x)<0$:
    \[ F_q(x) = \sum_{n=1}^\infty(-)^{n-1}\frac{e^{xn}}{n^{q+1}} \]
\item Consider the representation (with $p=q+1$), where $\mathcal{L}$ is a vertical contour cutting the real
    $s$-axis between $0$ and $1$:
    \[ F_{p-1}(x) = \frac{1}{2\ii} \int_{\mathcal{L}} \frac{e^{xs}}{s^p \sin\pi s}\,ds \qquad p>0, x>0 \]
\item (etc.)
\end{itemize}
\end{implementation}

%%%%%%%%%%%%%%%%%%%%%%%%%%%%%%%%%%%%%%%%%%%%%%%%%%%%%%%%%%%%%%%%%%%%%%%%%%%%%%%%
\section{Coulomb wave functions}
These are solutions $w(\rho)$ to the Coulomb (wave) equation, for $\ell=0,1,2,\dots$; $\eta\in\RR$; $\rho>0$:
\[ w'' + \left( 1 - \frac{2\eta}{\rho} - \frac{\ell(\ell+1)}{\rho^2}\right) w = 0 \]

The {\em normalizing constant} is defined
\[ C_\ell(\eta) = \frac{2^\ell e^{-\pi\eta/2} |\Gam(\ell+1+\eta\ii)|}{(2\ell+1)!}
    = \frac{2^\ell \left( \frac{2\pi\eta}{e^{2\pi\eta} - 1} \prod_{k=1}^\ell(\eta^2 + k^2)\right)^{1/2}}{(2\ell+1)!} \]
[use exp(log(...)) in the product form to avoid some blow-up for large $\ell$.]
The {\em Coulomb phase shift} is defined (where the branch of phase is chosen as zero when $\eta=0$)
\[ \sig_\ell(\eta) = \O{ph} \Gam(\ell+1+\eta\ii) \]
and we define
\[ \theta_\ell(\eta,\rho) = \rho - \eta\ln(2\rho) - \frac\pi2\eta + \sig_\ell(\eta) \]

The solution, recessive at $\rho=0$ is $\O{F}_\ell(\eta,\rho)$ (analytic on $\rho\in(0,\infty)$ and for $\eta\in\RR$)
defined via (where $\O{M}_{\kap,\mu}(z)$ is the Whittaker function and $\O{M}(a,b,z)$ is the Kummer function)
\begin{eqnarray*}
\O{F}_\ell(\eta,\rho)
    &=& C_\ell(\eta) \left(\frac{\mp\ii}{2}\right)^{\ell+1} \O{M}_{\pm\eta\ii, \ell+1/2}(\pm2\rho\ii) \\
    &=& C_\ell(\eta) \rho^{\ell+1} e^{\mp\rho\ii} \O{M}(\ell+1\mp\eta\ii, 2\ell+2, \pm2\rho\ii)
\end{eqnarray*}
The irregular solutions (both analytic on $\rho\in(0,\infty)$ and with $e^{\mp\sig\ii}H$ analytic on $\eta\in\RR$) are
\begin{eqnarray*}
\O{H}_\ell^\pm(\eta,\rho)
    &=& \O{G}_\ell(\eta,\rho) \pm \ii \O{F}_\ell(\eta,\rho) \\
    &=& (\mp\ii)^{\ell} e^{(\pi\eta/2)\pm\ii\sig_\ell(\eta)} \O{W}_{\mp\eta\ii, \ell+1/2}(\mp2\rho\ii) \\
    &=& (\mp2\rho\ii)^{\ell+1\pm\eta\ii} e^{\pm\ii\theta_\ell(\eta,\rho)} U(\ell+1\pm\eta\ii, 2\ell+2, \mp2\rho\ii)
\end{eqnarray*}

Interrelations:
$\mathcal{W}\{G_\ell,F_\ell\} = \mathcal{W}\{H^\pm_\ell, F_\ell\} = 1$
and
$F_{\ell-1} G_\ell - F_\ell G_{\ell-1} = \frac{\ell}{\sqrt{\ell^2+\eta^2}}$.

We have various recursion formula.  Let $R_\ell = \sqrt{1 + \eta^2/\ell^2}$, $S_\ell = \ell/\rho + \eta/\ell$, and $T_\ell=S_\ell + S_{\ell+1}$,
then with $X_\ell$ being $\O{F}_\ell(\eta,\rho)$ or $\O{G}_\ell(\eta,\rho)$, or $\O{H}_\ell^\pm(\eta,\rho)$, we have
\[ R_\ell X_{\ell-1} - T_\ell X_\ell + R_{\ell+1} X_{\ell+1} = 0 \qquad (\ell\geq1) \]
\[ X_\ell' = R_\ell X_{\ell-1} - S_\ell X_\ell \qquad (\ell\geq1) \]
\[ X_\ell' = S_{\ell+1} X_\ell - R_{\ell+1} X_{\ell+1} \qquad (\ell\geq0) \]
[These recursions should be used in decreasing $\ell$ for regular solutions and increasing $\ell$ for irregular solutions.]

We have a series in $\rho$:
\[ \O{F}_\ell(\eta,\rho) = C_\ell(\eta) \sum_{k=\ell+1}^\infty A^\ell_k(\eta) \rho^k \]
\[ \O{F}_\ell'(\eta,\rho) = C_\ell(\eta) \sum_{k=\ell+1}^\infty k A^\ell_k(\eta) \rho^{k-1} \]
where
$A^\ell_{\ell+1} = 1$, $A^\ell_{\ell+2} = \frac{\eta}{\ell+1}$, and
$(k+\ell)(k-\ell-1)A^\ell_k = 2\eta A^\ell_{k-1} - A^\ell_{k-2}$ for $k=\ell+3, \ell+4, \dots$.
(Alternatively, can write this as $A^\ell_k(\eta) = \frac{(-\ii)^{k-\ell-1}}{(k-\ell-1)!} \Hyper{2}{1}{\ell+1-k, \ell+1-\eta\ii}{2\ell+2}{2}$.)

Various limiting forms:

As $\rho\to\infty$ with $\eta$ fixed:
$\O{F}_\ell(\eta,\rho) = \sin( \theta_\ell(\eta,\rho) ) + o(1)$,
$\O{G}_\ell(\eta,\rho) = \cos( \theta_\ell(\eta,\rho) ) + o(1)$, and
$\O{H}^\pm_\ell(\eta,\rho) \sim \exp( \pm\ii\theta_\ell(\eta,\rho) )$.

For $\eta=0$,
$\O{F}_\ell(0,\rho) = \rho \O{j}_\ell(\rho) = \sqrt{\rho\pi/2} \O{J}_{\ell+\tfrac12}(\rho)$,
$\O{G}_\ell(0,\rho) = -\rho \O{y}_\ell(\rho) = -\sqrt{\rho\pi/2} \O{Y}_{\ell+\tfrac12}(\rho)$, and
$C_\ell(0) = \frac{2^\ell \ell!}{(2\ell+1)!} = \frac{1}{(2\ell+1)!!}$.
(Note thus that $\O{F}_0(0,\rho) = \sin\rho$, $\O{G}_0(0,\rho) = \cos\rho$, and $\O{H}^\pm_0(0,\rho)=e^{\pm\rho\ii}$.)

etc.

%%%%%%%%%%%%%%%%%%%%%%%%%%%%%%%%%%%%%%%%%%%%%%%%%%%%%%%%%%%%%%%%%%%%%%%%%%%%%%%%
\section{Orthogonal polynomials}
%\[\begin{array}{l|lccc|cc}
%{}&\multicolumn{4}{c|}{g_2 w'' + g_1 w' + g_0 w = 0} &{}&{}\\
%\text{Name} & w(x) & g_2(x) & g_1(x) & g_0(x) & \text{weight} & \text{interval} \\\hline
%\text{Chebyshev, 1st} & T_n(x) & 1-x^2 & -x & n^2 & (1-x^2)^{-1/2} & (-1,1) \\
%\text{Chebyshev, 2nd} & U_n(x) & 1-x^2 & -3x & n(n+2) & (1-x^2)^{1/2} & (-1,1) \\
%\text{Gegenbauer} & C^{(\al)}_n(x) & 1-x^2 & -(2\al+1)x & n(n+2\al) & (1-x^2)^{\al-1/2} & (-1,1) \\
%\text{Hermite} & H_n(x) & 1 & -2x & 2n & e^{-x^2} & (-\infty,\infty) \\
%\text{Laguerre} & L^{(\al)}_n(x) & x & \al+1-x & n & e^{-x}x^\al & (0,\infty) \\
%\text{Legendre} & P_n(x) & 1-x^2 & -2x & n(n+1) & 1 & (-1,1) \\
%\text{Jacobi} & P_n^{(\al,\beta)}(x) & 1-x^2 & \beta-\al-(\al+\beta+2)x & n(n+\al+\beta+1) & (1-x)^\al(1+x)^\beta & (-1,1) \\
%\end{array}\]
\begin{itemize}
\item name, symbol, parameters (and restrictions)
\item differential equation (and second solution to DE); integral equation(s)
\item integral representation(s)
\item recurrence relation(s), ``Rodrigues' formula''
\item representation(s) as hypergeometric series
\item weight, interval --- orthogonality relations, orthonormality weightings,
  ortho functions (basically including sqrt of weight)
\item alternative tricks: chebyshev in terms of trig functions, etc.
\item zeros, polynomial coeffs, ... Gauss integration weights, ...
\item relationships (Jacobi $\to$ Gegenbauer $\to$ Chebyshev1\&2/Legendre
\end{itemize}
Computation of {\em general} orthogonal polynomials --- compute coefficients / zeros / evaluate / etc.
(Give arbitrary weight function or arbitrary zero locations or ...)
Evaluate functions given as a squence of orthogonal polynomials; compute such series representations.

{\tt function-evaluation, coefficients, zeros, weights, etc.}
Other orthogonal polynomials (...) generalized ... etc.

Note that {\em orthonormal} $p_n$ satisfy
\[ p_{n+1} - (a_n x + b_n)p_n + c_n p_{n-1} = 0 \]
where $c_0=0$, $c_n = a_n / a_{n-1}$; $b_n = -a_n\langle x p_n, p_n \rangle$; $a_n = k_{n+1} / k_n$,
with $k_n$ being the coefficient of $x^n$ in $p_n$.

Let
\[ K_n(x,y) = \sum_{k=0}^n p_k(x) p_k(y) = \frac{k_n}{k_{n+1}} \frac{p_n(y)p_{n+1}(x) - p_n(x) p_{n+1}(y)}{x-y} \]
then we have
\[ K_n(x,x) = \frac{k_n}{k_{n+1}}\left( p_n(x) p_{n+1}'(x) - p_n'9x) p_{n+1}(x) \right) = \sum_{k=0}^n p_k^2(x) > 0 \]

For Gaussian quadrature we have
\[ \int w(x) f(x) \,dx \approx \sum_{k=1}^n \lam_{k,n} f(x_k) \]
where $x_1, \dots, x_n$ being the zeros of $p_n$ and where
\[ \lam_{k,n} = \left( \sum_{j=0}^{n-1} p_j(x_k)^2 \right)^{-1} \]

To find the zeros of orthogonal polynomials $p_n$ (from SF book):
suppose that $p_n$ satisfy the recurrence (with $c_0 p_{-1} = 0$):
\[ x p_k = a_k p_{k+1} + b_k p_k + c_k p_{k01} \]
then the zeros of $p_n$ are exactly the eigenvalues of the Jacobian matrix
\[\begin{pmatrix}
b_0 & a_0 &   0 \\
c_1 & b_1 & a_1 &  0 \\
  0 & c_2 & b_2 & a_2 &  0 \\
    &  0  & \ddots & \ddots & \ddots & 0\\
    &     &   0 & a_{n-2} & b_{n-2} & a_{n-2} \\
    &     &     &       0 & c_{n-1} & b_{n-1}
\end{pmatrix}\]
For instance, for the generalized Laguerre polynomials we have $a_n = -(n+1)$, $b_n = (2n+\al+1)$, $c_n = -(n+\al)$.
For Hermite $H_n$ we have $a_n = 1/2$, $b_n = 0$, $c_n = n$.
Note that due to numerical issues, an eigenvalue solver may find complex eigenvalues for large values of $n$
(for some cases - Hermite especially, whereas Laguerre works much more effectively).
Even if all eigenvalues found are real, it is useful to ``polish'' the accuracy by applying root-finders to improve the accuracy
of the roots found (such as Newton --- Halley's method hasn't been found to be very effective).

Alternative approach might be an iterative method using the roots of $p_{n-1}$ to bracket the roots of $p_n$ for a general root-finder.

An alternative symmetric matrix (whose eigenvalues give roots) is given by
\[\begin{pmatrix}
\beta_0 & \al_0 &   0 \\
\al_1 & \beta_1 & \al_1 &  0 \\
  0 & \al_2 & \beta_2 & \al_2 &  0 \\
    &  0  & \ddots & \ddots & \ddots & 0\\
    &     &   0 & \al_{n-2} & \beta_{n-2} & \al_{n-2} \\
    &     &     &       0 & \al_{n-1} & \beta_{n-1}
\end{pmatrix}\]
(for details see NR4SF book...)

%%%%%%%%%%%%%%%%%%%%%%%%%%%%%%%%%%%%%%%%
\subsection{Chebyshev polynomials}

%%%%%%%%%%%%%%%%%%%%
\subsubsection{First kind}
$\O{T}_n(z)$
\begin{itemize}
\item $\O{T}_0 = 1$, $\O{T}_1 = x$
\item Weight $w(x) = (1-x^2)^{-1/2}$ on interval $(-1,1)$
\item Differential equation $\left(1-x^2\right) \O{T}_n'' + \left(-x\right) \O{T}_n' + \left(n^2 \right) \O{T}_n = 0$
\item Normalization $(\pi/2)^{-1/2}$ for $n\neq0$ and $(\pi)^{-1/2}$ for $n=0$
\item Relation $\O{T}_n(x) = \frac{n!}{(1/2)_n} \O{P}_n^{(-1/2, -1/2)}(x)$
\item Recurrence $\O{T}_n = 2 x \O{T}_{n-1} - \O{T}_{n-2}$
\item Derivative $\O{T}_n' = \frac{n}{1-x^2}\left( \O{T}_{n-1} - x \O{T}_{n} \right)$
\item Zeros
\item Weights
\item Rodrigues' formula $\O{T}_n(z) = (-2)^n \frac{n!}{(2n)!} (1-x^2)^{1/2} \frac{d^n}{dx^n}\left[ (1-x^2)^{n-1/2} \right]$
\item $\O{T}_n(x) = \cos(n \acos x)$.
\end{itemize}
Shifted: $\O{T}^*_n(x) = T_n(2x-1)$ on $(0,1)$ with weight $(x(1-x))^{-1/2}$

%%%%%%%%%%%%%%%%%%%%
\subsubsection{Second kind}
$\O{U}_n(z)$
\begin{itemize}
\item $\O{U}_0 = 1$, $\O{U}_1 = 2x$
\item Weight $w(x) = (1-x^2)^{1/2}$ on interval $(-1,1)$
\item Differential equation $\left(1-x^2\right) \O{U}_n'' + \left(-3x\right) \O{U}_n' + \left(n(n+2)\right) \O{U}_n = 0$
\item Normalization $ $
\item Relation $\O{U}_n(x) = \frac{(n+1)!}{(3/2)_n} \O{P}_n^{(1/2, 1/2)}(x)$
\item Recurrence $\O{U}_n = 2 x \O{U}_{n-1} - \O{U}_{n-2}$
\item Derivative $\O{U}_n' = \frac{1}{1-x^2}\left( (n+1)\O{U}_{n-1} - n x \O{U}_{n} \right)$
\item Zeros
\item Weights
\item Rodrigues' formula $\O{U}_n(z) = (-2)^n\frac{n+1}{2n+1}\frac{n!}{(2n)!} (1-x^2)^{-1/2} \frac{d^n}{dx^n}\left[ (1-x^2)^{n+1/2} \right]$
\item 
\end{itemize}
Shifted: $\O{U}^*_n(x) = U_n(2x-1)$ on $(0,1)$ with weight $(x(1-x))^{1/2}$

%%%%%%%%%%%%%%%%%%%%%%%%%%%%%%%%%%%%%%%%
\subsection{Legendre polynomials}
$\O{P}_n(z)$
\begin{itemize}
\item $\O{P}_0 = 1$, $\O{P}_1 = x$
\item Weight $w(x) = 1$ on interval $(-1,1)$
\item Differential equation $\left(1-x^2\right) \O{P}_n'' + \left(-2x\right) \O{P}_n' + \left(n(n+1)\right) \O{P}_n = 0$
\item Normalization $\sqrt{\frac{2}{2n+1}}$
\item Relation $\O{P}_n(x) = \O{P}_n^{(0,0)}(x)$
\item Recurrence $\O{P}_n = \frac{(2n-1)x}{n}\O{P}_{n-1} - \frac{n-1}{n}\O{P}_{n-2}$
\item Derivative $\O{P}_n' = $
\item Zeros
\item Weights
\item Rodrigues' formula $\O{P}_n(z) = \frac{1}{2^n n!}\frac{d^n}{dx^n}\left[ (x^2-1)^n \right]$
\item 
\end{itemize}

%%%%%%%%%%%%%%%%%%%%%%%%%%%%%%%%%%%%%%%%
\subsection{Laguerre (generalized) polynomials}
$\O{L}^{(\al)}_n(z)$ with $\al>-1$
\begin{itemize}
\item $\O{L}^{(\al)}_0 = 1$, $\O{L}^{(\al)}_1 = 1+\al-x$
\item Weight $w(x) = e^{-x} x^\al$ on interval $(0,\infty)$
\item Differential equation $\left(x\right) \O{L}_n'' + \left(\al+1-x\right) \O{L}_n' + \left(n\right) \O{L}_n = 0$
\item Normalization $\sqrt{\frac{n!}{\Gam(n+\al+1)}}$
\item Relation
\item Recurrence $\O{L}^{(\al)}_n(z) = \frac{2n+a-1-z}{n}\O{L}^{(\al)}_{n-1}(z) - \frac{n+\al-1}{n}\O{L}^{(\al)}_{n-2}(z)$
\item Derivative $\O{L}' = \frac{n}{x}\left(\O{L}_{n} - \O{L}_{n-1}\right)$ [TODO: generalized form]
\item Zeros
\item Weights
\item Rodrigues' formula $\O{L}^{(\al)}_n(z) = e^x \frac{x^{-\al}}{n!} \frac{d^n}{dx^n}\left[ e^{-x} x^{n+\al} \right]$
\item 
\end{itemize}

%%%%%%%%%%%%%%%%%%%%%%%%%%%%%%%%%%%%%%%%
\subsection{Gegenbauer (ultraspherical) polynomials}
$\O{C}^{(\gam)}_n(z)$ with $\gam...$
\begin{itemize}
\item Weight $w(x) = (1-x^2)^{\gam-1/2}$ on interval $(-1,1)$
\item Differential equation $\left(1-x^2\right) \O{C}_n'' + \left(-(2\gam+1)x\right) \O{C}_n' + \left(n(n+2\gam)\right) \O{C}_n = 0$
\item Normalization $\left(\frac{(2\gam)_n}{(\gam+1/2)_n} \frac{2^\gam \Gam(n+\gam+1/2)}{\sqrt{2(n+\gam)n!\Gam(n+2\gam)}}\right)^{-1}$
\item Relation $\O{C}^{(\gam)}_n(x) = \frac{(2+\gam)_n}{(\gam+1/2)_n} \O{P}_n^{(\gam-1/2, \gam-1/2)}(x)$
\item Recurrence
\item Zeros
\item Weights
\item Rodrigues' formula $\O{C}^{(\gam)}_n(z) = \frac{d^n}{dx^n}\left[  \right]$
\item Eigenvalue matrix terms $\beta_j = 0$,
    \[ \al_j = \frac{2}{2j+2\gam-1} \sqrt{\frac{j(j+\gam-1/2)^2(j+2\gam-1)}{(2j+2\gam)(2j+2\gam-2)}} \qquad j\geq1 \]
\end{itemize}

%%%%%%%%%%%%%%%%%%%%%%%%%%%%%%%%%%%%%%%%
\subsection{Jacobi polynomials}
$\O{P}^{(\al,\beta)}(z)$ with $\al,\beta>-1$
\begin{itemize}
\item Weight $w(x) = (1-x)^\al(1+x)^\beta$ on interval $(-1,1)$
\item Differential equation $\left(1-x^2\right) \O{P}_n'' + \left(\beta-\al-(\al+\beta+2)x\right) \O{P}_n' + \left(n(n+\al+\beta+1)\right) \O{P}_n = 0$
\item Normalization $\sqrt{\frac{2^{\al+\beta+1}}{2n+\al+\beta+1} \frac{\Gam(n+\al+1)}{\Gam(n+1)} \frac{\Gam(n+\beta+1)}{\Gam(n+\al+\beta+1)}}$
\item Relation 
\item Recurrence
\item Zeros
\item Weights
\item Rodrigues' formula $\O{P}^{(\al,\beta)}_n(z) = \frac{(-)^n}{2^n n!}(1-x)^{-\al}(1+x)^{-\beta}\frac{d^n}{dx^n}\left[ (1-x)^{n+\al}(1+x)^{n+\beta} \right]$
\item Eigenvalue matrix terms $\beta_0 = \frac{\beta-\al}{\al+\beta+2}$,
    \[ \beta_j = \frac{\beta^2-\al^2}{(2j+\al+\beta)(2j+\al+\beta+2)} \qquad j\geq1 \]
    \[ \al_j = \frac{2}{2j+\al+\beta}\sqrt{\frac{j(j+\al)(j+\beta)(j+\al+\beta)}{(2j+\al+\beta+1)(2j+\al+\beta-1)}} \qquad j\geq1 \]
\end{itemize}
Shifted: $\O{R}^{(\al,\beta)}(x)=\O{P}^{(\al,\beta)}_n(2x-1)$ on $(0,1)$ with weight $(1-x)^\al x^\beta$

%%%%%%%%%%%%%%%%%%%%%%%%%%%%%%%%%%%%%%%%
\subsection{Hermite polynomials}

%%%%%%%%%%%%%%%%%%%%
\subsubsection{Physicists'}
$\O{H}_n(z)$
\begin{itemize}
\item $\O{H}_0 = 1$, $\O{H}_1 = 2x$
\item Weight $w(x) = e^{-x^2}$ on interval $(-\infty,\infty)$
\item Differential equation $\left(1\right) \O{H}_n'' + \left(-2x\right) \O{H}_n' + \left(2n\right) \O{H}_n = 0$
\item Normalization $(\sqrt{pi}2^n n!)^{-1/2}$
\item Relation 
\item Recurrence $\O{H}_n = 2x\O{H}_{n-1} - 2(n-1)\O{H}_{n-2}$
\item Derivative $\O{H}_n' = 2n \O{H}_{n-1}$
\item Zeros
\item Weights
\item Rodrigues' formula $\O{H}_n(z) = (-)^n e^{x^2} \frac{d^n}{dx^n}\left[ e^{-x^2} \right]$
\item Eigenvalue matrix terms $\beta_j = 0$, $\al_j = \sqrt{j/2}$, $\mu_0 = \sqrt\pi$
\end{itemize}

%%%%%%%%%%%%%%%%%%%%
\subsubsection{Probabilists'}
$\O{He}_n(z)$
\begin{itemize}
\item Weight $w(x) = e^{-x^2/2}$ on interval $(-\infty,\infty)$
\item Differential equation $\left(2\right) \O{He}_n'' + \left(-4x\right) \O{He}_n' + \left(2n\right) \O{He}_n = 0$
  [double-check this]
\item Normalization $(\sqrt{2pi}n!)^{-1/2}$
\item Relation $\O{He}_n(z) = 2^{-n/2}\O{H}_n(x/\sqrt2)$
\item Recurrence
\item Zeros: zeros of $\O{He}_n(x)$ are $\sqrt{2} \times$ zeros of $\O{H}_n(x)$
\item Weights
\item Rodrigues' formula $\O{He}_n(z) = (-)^n e^{x^2/2} \frac{d^n}{dx^n}\left[ e^{-x^2/2} \right]$
\item 
\end{itemize}

%%%%%%%%%%%%%%%%%%%%%%%%%%%%%%%%%%%%%%%%%%%%%%%%%%%%%%%%%%%%%%%%%%%%%%%%%%%%%%%%
\section{Misc functions}

synchrotron, transport, Einstein, Lobachevsky, Erdelyi $\mu$/$\nu$,

Heun

Lam\'e

Painlev\'e

%%%%%%%%%%%%%%%%%%%%%%%%%%%%%%%%%%%%%%%%
\subsection{Anger, Weber functions}
[Move this section???]

%%%%%%%%%%%%%%%%%%%%
\subsubsection{Lommel functions}
For $\mu\pm\nu\neq\pm1,\pm3,\pm5,\dots$
\[ S_{\mu,\nu}(z) = \frac{z^{\mu+1}}{(\mu+1)^2 - \nu^2} \sum_{k=0}^\infty t_k \]
where $t_0 = 1$ and $t_k = t_{k-1}\frac{-z^2}{(\mu+2k+1)^2 - \nu^2}$.

And the second Lommel function given via asymptotic expansion:
\[ s_{\mu,\nu}(z) \sim \sum_{k=0}^\infty u_k \]
where $u_0=1$, $u_k = u_{k-1}\frac{-(\mu-2k+1)^2 - \nu^2}{z^2}$.

%%%%%%%%%%%%%%%%%%%%
\subsubsection{Anger function}
\[ \Ob{J}_\nu(z) = \int_0^\pi\frac{\cos(\nu\theta - z\sin\theta)}{\pi}\,d\theta \]
We have
\[ \Ob{J}_\nu(z) = \frac{\sin\nu\pi}{\pi}S_{0,\nu}(z) - \nu\frac{\sin\nu\pi}{\pi}S_{-1,\nu}(z) \]
and
\[ \Ob{J}_\nu(z) = \O{J}_\nu(z) + \frac{\sin\nu\pi}{\pi z}s_{0,\nu}(z) - \nu\frac{\sin\nu\pi}{\pi z^2}s_{-1,\nu}(z) \]

Power-series if $|z|<10$ or $|z/\nu|<q$??
\begin{implementation}
From~\cite{atlas:thompson}: to get an array of values for index $\nu, \nu+1, \dots, \nu+m$ with $0<\nu<1$:
\begin{itemize}
\item Compute $\Ob{J}_\nu(x)$, $\Ob{J}_{\nu+1}(x)$ by power-series for $x<30$ and by asymptotic expansion for $x\geq30$
\item Use upwards recurrence in $\nu$ --- for small $x$ this can be bad, so we can use a downwards recurrence
    (with modifications to the previous step)
\item For the special case of $x=0$, use the fact that $\Ob{J}_\nu(0) = \sin(\nu\pi)/(\nu\pi)$ exactly
\end{itemize}
\end{implementation}

%%%%%%%%%%%%%%%%%%%%
\subsubsection{Weber function}
\[ \Ob{E}_\nu(z) = \int_0^\pi\frac{\sin(\nu\theta - z\sin\theta)}{\pi}\,d\theta \]
This is computed the same basic way as $\Ob{J}_\nu(x)$ above...~\cite{atlas:thompson}.

We have
\[ \Ob{E}_\nu(z) = -\frac{1+\cos\nu\pi}{\pi}S_{0,\nu}(z) - \nu\frac{1-\cos\nu\pi}{\pi}S_{-1,\nu}(z) \]
and
\[ \Ob{E}_\nu(z) = -\O{Y}_\nu(z) + \frac{1+\cos\nu\pi}{\pi}s_{0,\nu}(z) - \nu\frac{1-\cos\nu\pi}{\pi z^2}s_{-1,\nu}(z) \]
[CHECK THE DENOMINATOR ON THE FIRST FRACTION IN THE PREVIOUS EQUATION! (compare with Anger function...)]

\[ \Ob{E}_\nu(0) = \frac{\cos(\pi/\nu) - 1}{\nu\pi} \]

%%%%%%%%%%%%%%%%%%%%%%%%%%%%%%%%%%%%%%%%
\subsection{Howland integrals}
\[ I_k = \frac{1}{2(k!)}\int_0^\infty \frac{w^k}{\O{sinh}w + w}\,dw \qquad (k\geq1) \]
\[ I^*_k = \frac{1}{2(k!)}\int_0^\infty \frac{w^k}{\O{sinh}w - w}\,dw \qquad (k\geq3) \]

%%%%%%%%%%%%%%%%%%%%%%%%%%%%%%%%%%%%%%%%
\subsection{Mathieu functions}
(elliptic cylinder functions)
[This needs to be cleaned up!!]
[MOVE TO ITS OWN SECTION!]

{\tt characteristic values, expansion coefficients, sem, cem dsem, dcem, McM, dMcM, MsM, dMsM, etc. ...}

From Zhang-Jin~\cite{zj}
Mathieu's differential equation is
\[  y'' + (\lam - 2q\cos 2z)y = 0 \]
And note that due to periodicity requirements, there are conditions on $\lam=\lam(q)$ as a function of $q$.
Let $\lam(0) = m^2$, then we get the following Mathieu functions of order~$m$:
\[ \O{ce}_m(z,q) = \sum_{k=0}^\infty A^m_k(q)\cos kz \]
\[ \O{se}_m(z,q) = \sum_{k=0}^\infty B^m_k(q)\cos kz \]
for some coeffs $A$, $B$...

%%%%%%%%%%%%%%%%%%%%
\subsubsection{Modified Mathieu functions}

\[  y'' - (\lam - 2q\cosh 2z)y = 0 \]
\[ \O{Ce}_m(z,q) = \O{ce}_m(z,q)(\ii z,q) \]
\[ \O{Se}_m(z,q) = -\ii \O{se}_m(z,q)(\ii z,q) \]

\[ \O{Mc}^{(j)}_m(z,q) = (j=1..4)\]
\[ \O{Ms}^{(j)}_m(z,q) = (j=1..4)\]
Matheiu-Hankel for $j=3,4$

? others: ?

\[ \O{Cc}^{(1)}_m(z,q) = \]
\[ \O{Cs}^{(1)}_m(z,q) = \]
\[ \O{Cc}^{(2)}_m(z,q) = \]
\[ \O{Cs}^{(2)}_m(z,q) = \]
\[ \O{Cc}^{(3)}_m(z,q) = \]
\[ \O{Cs}^{(3)}_m(z,q) = \]
\[ \O{Cc}^{(4)}_m(z,q) = \]
\[ \O{Cs}^{(4)}_m(z,q) = \]
\[ \lambda(q) = \]

%%%%%%%%%%%%%%%%%%%%%%%%%%%%%%%%%%%%%%%%
\subsection{Debye functions}
\[ \O{D}_n(x) = \int_0^x\frac{t^n}{e^t-1}\,dt \]
\[ \O{\widetilde{D}}_n(x) = \frac{n}{x^n}\O{D}_n(x) \]
\begin{implementation}
A series representation (compare with generating function for Bernoulli numbers) (for $|x|<2\pi$ and $n\geq1$):
\[ \int_0^x\frac{t^n}{e^t-1}\,dt = x^n\left( \frac{1}{n} - \frac{x}{2(n+1)} + \sum_{k=1}^\infty \frac{B_{2k}}{(2k+n)(2k)!}x^{2k}\right) \]
This series works ok, but requires increasing number of terms as $n$ or $x$ increase (for $n=1$, $x=1$ we already need 13 terms, for $n=6$, $x=3$ we need 31 terms).

Reflection:
\[ \O{D}_n(-x) = (-)^n \O{D}_n(x) + (-)^n\frac{x^{n+1}}{n+1} \]
\[ \O{\widetilde{D}}_n(-x) = \O{\widetilde{D}}_n(x) + \frac{n}{n+1}x \]

Also for the complementary integral (for $x>0$ and $n\geq1$):
\[ \int_x^\infty\frac{t^n}{e^t-1}\,dt = \sum_{k=1}^\infty e^{-kx}\left( \frac{x^n}{k} + \frac{n x^{n-1}}{k^2} + \frac{n(n-1)x^{n-2}}{k^3} + \cdots + \frac{n!}{k^{n+1}} \right) \]
Note that
\[ \int_0^\infty\frac{t^n}{e^t-1}\,dt = n!\zeta(n+1) \]
The complementary integral is useful for $|x|\geq2$ but for large $n$ (say $n>20$) we get bad cancellation.

Note
\[ \O{D}_n(x) = n!\zeta(n+1) - \int_x^\infty\frac{t^n}{e^t-1}\,dt = n!\left\{\left[\zeta(n+1)-1\right]
    + e^{-x}\left[\sum_{j=n+1}^\infty\frac{z^j}{j!}\right] - \sum_{k=2}^\infty \frac{e^{-kx}}{k^{n+1}}\left[\sum_{j=0}^n\frac{z^j}{j!}\right] \right\} \]
which might offer improved numerical stability (with bracketed terms computed individually with high precision)

Asymptotic expansion:
\[ \O{D}_n(x) \sim n!\zeta(n+1) - e^{-x}\cdot n\cdot n!\sum_{k=0}^n\frac{x^k}{k!} + \log(1-\cosh(x)+\sinh(x))x^n \]
\end{implementation}

%%%%%%%%%%%%%%%%%%%%%%%%%%%%%%%%%%%%%%%%
\subsection{Sievert integral}
\[ \O{S}(\theta,x) = \int_0^\theta e^{-x\sec\phi}\,d\phi = \int_0^\theta e^{-x/cos\phi}\,d\phi \]
\[ \O{\widetilde{S}}(\theta,x) = \frac{\O{S}(\theta,x)}{\theta} \]

From~\cite{a&s}:
As $x\to\infty$,
\[ \O{S}(x,\theta) \sim \sqrt{\frac{\pi}{2x}} e^{-x} \erf(\theta\sqrt{x/2}) \]
This can be derived by expanding $\sec\phi$ as a power-series at zero, taking the first 2 terms:
$\sec\phi = 1+\frac{\phi^2}{2} + O(\phi^4)$ and integrating the resulting integral:
$\int_0^\theta e^{-z(1+t^2/2)}\,dt$.

Also, if $a_0=1$, $a_n=\frac{1\cdot3\cdot5\cdots(2n-1)}{2\cdot4\cdot6\cdots(2n)}$ then
\[ \O{S}(x,\theta) = \O{S}(x,\pi/2) - \sum_{k=0}^\infty \al_k \cos(\theta)^{2k+1} E_{2k+2}(\frac{x}{\cos\theta}) \]
for $x\geq0$, $0<\theta<\pi/2$.  ($E_n(z)=\int_1^\infty e^{-zt} t^{-n}\,dt$ are exponential integrals.)

\[ \O{S}(x,\pi/2) = \O{Ki}_1(x) = \int_x^\infty \O{K}_0(t)\,dt \]

We can derive the following expansion by expanding the exponential and integrating,
but it doesn't seem particularly helpful for calculations:
\[ \O{S}(x,\theta) = \sum_{n=0}^\infty(-x)^n \Hyper{2}{1}{1/2, (n+1)/2}{3/2}{\sin^2 \theta} \]

%%%%%%%%%%%%%%%%%%%%%%%%%%%%%%%%%%%%%%%%
\subsection{Abramowitz functions}
\[ \O{f}_m(x) = \int_0^\infty t^me^{-t^2-x/t}\,dt \]
for $m=0,1,2,\dots$

\begin{implementation}
A few potentially useful facts (from~\cite{a&s}):
\begin{itemize}
\item A recurrence: $2\O{f}_m = (m-1)\O{f}_{m-2} + x\O{f}_{m-3}$
\item $f'_m = -f_{m-1}$ ($m=1,2,\dots$)
\item A series expansion for $\O{f}_1$:
  \[ \O{f}_1(x) = \frac12\sum_{k=0}^\infty(a_k\ln x + b_k)x^k\]
  where
  \begin{itemize}
  \item $a_0 = a_1 = 0$; $a_2 = -b_0$
  \item $b_0 = 1$; $b_1 = -\sqrt\pi$; $b_2 = \tfrac32(1-\gamma)$
  \item $a_k = \frac{-2a_{k-2}}{k(k-1)(k-2)}$
  \item $b_k = \frac{-2b_{k-2} - (3k^2-6k+2)a_k}{k(k-1)(k-2)}$
  \end{itemize}
\item An asymptotic expansion, let $v=3(x/2)^{2/3}$, then
  \[ \O{f}_{m}(x) \sim \sqrt{\frac\pi3}3^{-m/2}v^{m/2}e^{-v}\left(a_0 + \frac{a_1}{v} + \frac{a_2}{v^2} + \cdots \right) \]
  where $a_0 = 1$; $a_1 = \tfrac{1}{12}(3m^2+3m-1)$; and
  $12(k+2)a_{k+2} = -(12k^2+36k-3m^2-3m+25)a_{k+1} + \tfrac12(m-2k)(2k+3-m)(2k+3+2m)a_k$
\end{itemize}
\end{implementation}

%%%%%%%%%%%%%%%%%%%%%%%%%%%%%%%%%%%%%%%%
\subsection{Abramowitz 2 functions}
\[ \O{f}(x) = \int_0^\infty\frac{e^{-t^2}}{t+x}\,dt \]
Representation in terms of exponential integral and Dawson's integral:
\[ \O{f}(x) = -\frac12e^{-x^2}\O{Ei}(x^2) + \sqrt{\pi}e^{-x^2}\int_0^x e^{t^2}\,dt \]
Series expansion:
\[ \O{f}(x) = -e^{-x^2}\ln x + e^{-x^2}\left[ \sqrt{\pi}\sum_{k=0}^\infty \frac{x^{2k+1}}{k!(2k+1)}
    - \sum_{k=1}^\infty \frac{x^{2k}}{k!(2k)} - \frac\gamma2 \right] \]
or a series with the digamma function:
\[ \O{f}(x) = -e^{-x^2}\ln x + \frac12\sum_{k=0}^\infty\frac{\psi(k+1)x^{2k}}{k!}
    + \sqrt{\pi}\sum_{k=0}^\infty\frac{(-2)^k x^{2k+1}}{1\cdot3\cdot5\cdots(2k+1)} \]

%%%%%%%%%%%%%%%%%%%%%%%%%%%%%%%%%%%%%%%%
\subsection{Clausen integral}
[and generalizations...]
For $0\leq\theta\leq\pi$:
\[ \O{C}(\theta) = -\int_0^\theta \ln(2\sin t/2)\,dt = \sum_{n=1}^\infty \frac{\sin n\theta}{n^2} = \Im(\O{Li}_2(e^{\ii\theta}))\]

Another series:
\[ \O{C}(\theta) = -\theta\ln\theta + \theta + \sum_{k=1}^\infty(-)^{k-1}\frac{B_{2k}}{(2k)!}\frac{\theta^{2k+1}}{2k(2k+1)} \]
or equivalently:
\[ \frac{\O{C}(\theta)}{\theta} = 1 - \ln\theta + \sum_{k=1}^\infty\frac{\zeta(2k)}{n(2n+1)}\left(\frac{\theta}{2\pi}\right)^{2k} \]

Another series (from wikipedia)
\[ \O{C}(\theta) = 3\theta - \theta\ln\left(\theta(1-\frac{\theta^2}{4\pi^2})\right)
    -2\pi\ln(\frac{2\pi+\theta}{2\pi-\theta}) + \theta\sum_{n=1}^\infty \frac{\zeta(2n)-1}{n(2n+1)}\left(\frac{\theta}{2\pi}\right)^{n} \]
[check and test this...]

Relation:
\[ \O{C}(\pi-\theta) = \O{C}(\theta) - \frac12\O{C}(2\theta) \qquad (0\leq\theta\leq\pi/2) \]

%%%%%%%%%%%%%%%%%%%%%%%%%%%%%%%%%%%%%%%%
\subsection{Voigt (plasma dispersion) function}
\[ \O{V}(u,\al) = \frac1\pi \int_{-\infty}^\infty \frac{e^{-(\al y-u)^2}}{y^2+1}\,dy \]
\[ \O{V}(u,\al) = \Re(e^{z^2}\erfc(z)) \qquad z=\al+u\ii \]

The versions from DLMF:
\[ \O{U}(x,t) = \frac{1}{\sqrt{4\pi t}} \int_{-\infty}^\infty \frac{e^{-(x-y)^2/4t}}{1 + y^2}\,dy \]
\[ \O{V}(x,t) = \frac{1}{\sqrt{4\pi t}} \int_{-\infty}^\infty \frac{y e^{-(x-y)^2/4t}}{1 + y^2}\,dy \]
then with $z=\frac{1-\ii x}{2\sqrt{t}}$ we have
\[ \O{U}(x,t) + \ii\O{V}(x,t) = \sqrt{\frac{\pi}{4t}}e^{z^2}\erfc(z) \]

The ``line-broadening function'' (from DLMF) is defined by (and related to Voight functions via):
\[ \O{H}(a,u) = \frac{a}{\pi}\int_{-\infty}^\infty\frac{e^{-t^2}}{(u-t)^2 + a^2}\,dt
    = \frac{1}{a\sqrt\pi} \O{U}(\frac{u}{a}, \frac{1}{4a^2}) \]

%%%%%%%%%%%%%%%%%%%%%%%%%%%%%%%%%%%%%%%%
\subsection{Spence integral}
\[ \O{S}(x) = -\int_1^x\frac{\ln t}{t-1}\,dt = \O{Li}_2(1-x) \qquad \Re{x}\geq0 \vee x\notin\RR\]

\begin{implementation}
Implementation notes for $\O{S}(z)$, $z\geq0$:
\begin{itemize}
\item Recall $\O{Li}_2(z)=\sum_{k=1}^\infty\frac{z^k}{k^2}$ (with $z<1$). Recall the following identities (and their applicability when $z\in\RR$)
  \begin{itemize}
  \item [\bf(A)] $\O{Li}_2(z) + \O{Li}_2(\frac{1}{z}) = -\frac{\pi^2}{6} - \frac{\ln(-z)^2}{2}$ for $z<0$
  \item [\bf(B)] $\O{Li}_2(z) + \O{Li}_2(1-z) = \frac{\pi^2}{6} - \ln(z)\ln(1-z)$ for $0<z<1$
  \item [\bf(C)] $\O{Li}_2(z) + \O{Li}_2(\frac{z}{z-1}) = -\frac{\ln(1-z)^2}{2}$ for $z<1$
  \end{itemize}
\item Using these identities, we have the following computations:
  \begin{itemize}
  \item For $x=0$ use $\O{S}(0) = \frac{\pi^2}{6}$
  \item For $0<x<0.5$, use {\bf(B)} and compute $\O{S}(x) = \frac{\pi^2}{6} - \ln(x)\ln(1-x) - \sum_{k=1}^\infty \frac{x^k}{k^2}$
  \item For $0.5\leq x<1$, use $\O{S}(x)=\O{Li}_2(1-x)$ and compute $\O{S}(x) = \sum_{k=1}^\infty\frac{(1-x)^k}{k^2}$
  \item For $1\leq x<2.5$, use {\bf(C)} and compute $\O{S}(x) = -\frac{\ln(x)^2}{2} - \sum_{k=1}^\infty\frac{1}{k^2}\left(\frac{x-1}{x}\right)^k$
  \item For $2.5\leq x$, use {\bf(A)} and compute $\O{S}(x) = -\frac{\pi^2}{6} - \frac{\ln(x-1)^2}{2} - \sum_{k=1}^\infty \frac{(1-x)^{-k}}{k^2}$
  \end{itemize}
\item This gives 15 digits precision everywhere but up to 7 ulps near $1$
\item Generally have less than 50 terms needed everywhere in the domain, as $x\to\infty$, need fewer terms
\item Atlas recommends series near $0$ and $1$ and in other cases to integrate $\O{S}(x) = -\int_0^{x-1}\frac{\ln 1+u}{u}\,dt$ using Simpson's rule
  (140 points for $x\leq2$ and more points for larger $x$)
\end{itemize}
\end{implementation}

%%%%%%%%%%%%%%%%%%%%%%%%%%%%%%%%%%%%%%%%
\subsection{Angular momentum coupling coefficients}
Define $\Delta(abc) = \sqrt{\frac{(a+b-c)!(a-b+c)!(-a+b+c)!}{(a+b+c+1)!}}$

%%%%%%%%%%%%%%%%%%%%
\subsubsection{$3-j$ coefficients}
The sum below is over all nonnegative integers $s$ such that factorial arguments are nonnegative:
\begin{eqnarray*}
 \begin{pmatrix}a&b&c\\\al&\beta&\gam\end{pmatrix}
 &=& (-)^{a-b-\gam} \Delta(abc)\sqrt{(a+\al)!(a-\al)!(b+\beta)!(b-\beta)!(c+\gam)!(c-\gam)!} \\
 && {}\cdot\sum_s\frac{(-)^s}{s!(a+b-c-s)!(a-\al-s)!(b+\beta-s)!(c-b+\al+s)!(c-a-\beta+s)!}
\end{eqnarray*}

Alternatively: 
\begin{eqnarray*}
 \begin{pmatrix}a&b&c\\\al&\beta&\gam\end{pmatrix}
  &=& \delta_{\al+\beta+\gam,0}(-)^{a-b-\gam} \\
  && {}\cdot\sqrt{\frac{(c+a-b)!(c-a+b)!(a+b-c)!(c-\gam)!(c+\gam)!}{(a+b+c+1)!(a-\al)!(a+\al)!(b-\beta)!(b+\beta)!}} \\
  && {}\cdot{\sum_k}'\frac{(-)^{k+b+\beta}(b+c+\al-k)!(a-\al+k)!}{k!(c-a+b-k)!(c-\gam-k)!(k+a-b+\gam)!}
\end{eqnarray*}

%%%%%%%%%%%%%%%%%%%%
\subsubsection{$6-j$ coefficients}
\begin{eqnarray*}
\left\{\begin{matrix}a&b&c\\d&e&f\end{matrix}\right\}
  &=& (-)^{a+b+c+d}\Delta(abe)\Delta(acf)\Delta(bdf)\Delta(cde) \\
  && {}\cdot{\sum_k}'(a+b+c+d+1-k)! \\
  && {}\qquad \left(k!(e+f-a-d+k)!(e+f-b-c+k)!\right)^{-1} \\
  && {}\qquad \left((a+b-e-k)!(c+d-e-k)!(a+c-f-k)!(b+d-f-k)!\right)^{-1}
\end{eqnarray*}

%%%%%%%%%%%%%%%%%%%%
\subsubsection{$9-j$ coefficients}
\[\left\{\begin{matrix}a&b&c\\d&e&f\\g&h&i\end{matrix}\right\} = \sum_k(-)^{2k}(2k+1)
 \left\{\begin{matrix}a&i&k\\h&d&g\end{matrix}\right\}
 \left\{\begin{matrix}b&f&k\\d&h&e\end{matrix}\right\}
 \left\{\begin{matrix}a&i&k\\f&b&c\end{matrix}\right\}
\]

%%%%%%%%%%%%%%%%%%%%%%%%%%%%%%%%%%%%%%%%%%%%%%%%%%%%%%%%%%%%%%%%%%%%%%%%%%%%%%%%
\section{Combinatorial functions and numbers}
Fibonacci numbers/polynomials, Lucas numbers/polynomials,
Stirling numbers (1/2), factorial, binomial, multinomial,
Pochhammer symbol, rising/falling factorial, 
generalized Lucas/Fibonacci numbers,
((brackets and numbers from ``concrete mathematics''))

integer \& generalized versions

Bell numbers = \# of partitions of set.  Touchard's congruence: $B_{p+k}\cong B_k + B_{k+1}\pmod{p}$
where $p$ is prime and $B_n$ is a Bell number.  $e^{e^x-1} = \sum_{n=0}^\infty B_n \frac{x^n}{n!}$.
Bell polynomials --- Dobinski's formula: $B_n(x) = e^{-x}\sum_{k=0}^\infty k^n \frac{x^k}{k!}$.

%%%%%%%%%%%%%%%%%%%%%%%%%%%%%%%%%%%%%%%%
\subsection{Harmonic numbers}

\[ H_n = \sum_{k=1}^n \frac{1}{k} \]
We have the asymptotic expansion
\[ H_n \sim \log(n) + \gam + \frac{1}{2n} - \sum_{k=1}^\infty\frac{B_{2k}}{2k n^{2k}}
  = \log(n) + \gam + \frac{1}{2n} - \frac{1}{12n^2} + \frac{1}{120 n^4} - \cdots \]
(This expansion to $n^{-4}$ works perfectly for $n>1111$ in double-precision.)

We can also define one form of generalized harmonic numbers
\[ H^{(m)}_n = \sum_{k=1}^n \frac{1}{k^m} \]

Another generalization is given, for $\al\in\RR$, by
\[ H_{\al} = \int_0^1 \frac{1-x^\al}{1-x}\,dx \qquad \al\in(0,1) \]
and
\[ H_{\al} = H_{\al-1} + \frac{1}{\al} \]
\[ H_{1-\al} - H_{\al} = \pi\cot(\pi\al) - \frac{1}{\al} + \frac{1}{1-\al} \]
Note that we have, for $x>0$,
\[ H_x = x\sum_{k=1}^\infty\frac{1}{k(x+k)} \]

%%%%%%%%%%%%%%%%%%%%%%%%%%%%%%%%%%%%%%%%
\subsection{Bernoulli numbers, polynomials}

\[ \frac{t}{e^t-1} = \sum_{n=0}^\infty \frac{t^n}{n!} B_n \]
\[ \frac{t e^{xt}}{e^t-1} = \sum_{n=0}^\infty \frac{t^n}{n!} B_n(x) \]

Fact: $\zeta(2n) = \frac{(2\pi)^{2n}}{2} \frac{|B_{2n}|}{(2n)!}$ which implies
that $|B_{2n}\sim \frac{2(2n)!}{(2\pi)^{2n}}$ for large $n$...
** Use $\zeta$ to generate $B_n$ for large/scaled $n$ (very easy to compute)...

Compute: $B_n$, $B_n/n!$, $\ln|B_n|$, etc.

\[ B_n(x) = \sum_{k=0}^n\binom{n}{k}B_k x^{n-k} \]
\[ B_n(1-x) = (-)^n B_n(x) \]
\[ B_n(x+1) - B_n(x) = n x^{n-1} \]

Another approach for Bernoulli numbers (due to Ramanujan?):
\[ B_n = frac{a_n - s_n}{\binom{n+3}{n}} \]
where
\[ a_n = \begin{cases}\frac{n+3}{2} & n\cong0,2(6) \\ -\frac{n+3}{6} & n\cong4(6)\end{cases} \]
and
\[ s_n = \sum_{k=1}^{\lfloor n/6\rfloor} \binom{n+3}{n-6k} B_{n-6k} \]
Note that the case $n=1$ requires a special case.)

%%%%%%%%%%%%%%%%%%%%%%%%%%%%%%%%%%%%%%%%
\subsection{Euler numbers, polynomials}

\[ \frac{1}{\cosh t} = \frac{2 e^{t}}{e^{2t}+1} = \sum_{n=0}^\infty \frac{t^n}{n!} E_n  \qquad|z|<\frac\pi2\]
\[ \frac{2 e^{xt}}{e^t+1} = \sum_{n=0}^\infty \frac{t^n}{n!} E_n(x) \qquad|z|<\pi\]

\[ E_n = 2^n E_n(\tfrac12)\in\ZZ \]
\[ E_n(x) = \sum_{k=0}^n\binom{n}{k}\frac{E_k}{2^k} (x-\tfrac12)^{n-k} \]
\[ E_n(1-x) = (-)^n E_n(x) \]
\[ E_n(x+1) + E_n(x) = 2 x^n \]

We also have the {\em secant numbers},
with $\sec x = \sum_{k=0}^\infty S_k \frac{x^{2k}}{(2k)!}$:
\[ S_n = |E_{2k}| = E^*_k \]

%%%%%%%%%%%%%%%%%%%%%%%%%%%%%%%%%%%%%%%%
\subsection{Tangent numbers, polynomials}

\[ T_{2n-1} = \frac{2^{2n}(2^{2n}-1)|B_{2n}|}{2n} = (-)^{n-1}\frac{2^{2n}(2^{2n}-1)B_{2n}}{2n} \in \ZZ \]
and $T_{2n} = 0$ with $\tan x = \sum_{k=1}^\infty T_k \frac{x^k}{k!}$
Note that

%%%%%%%%%%%%%%%%%%%%%%%%%%%%%%%%%%%%%%%%
\subsection{Genocchi numbers, polynomials}

\[ \frac{2t}{e^t+1} = \sum_{n=1}^\infty G_n \frac{t^n}{n!} \]
whence $G_n = 2(1-2^n) B_n$.  We can define the scaled numbers also $G_n^* = G_n/n!$.


%%%%%%%%%%%%%%%%%%%%%%%%%%%%%%%%%%%%%%%%
\subsection{Number theoretic functions}

for unsigned long long (64-bit): isprime, factor (note <=16 distinct factors of $n\leq2^{64}$), nth\_prime, 

Chinese Remainder Theorem, M\"obius function, Jacobi function, $\sig$, $\tau$, etc.


%%%%%%%%%%%%%%%%%%%%%%%%%%%%%%%%%%%%%%%%%%%%%%%%%%%%%%%%%%%%%%%%%%%%%%%%%%%%%%%%
\section{Mathematical finance}

digital, power, log,
OT, NT, DOT, DNT, KI, KO, DKI, DKO, sequentials, windowed,
aisan, lookback, ratchet, cliquet,
forward-start, chooser, compound, reset, extendible,
fader, accumulator, accrual, digital-accrual, ..., fadelets, ...,
quanto, basket,


implied-volatility function

various associated functionals of Brownian motion: pdf, hitting-time pdfs, BF with drift,
lognormal, local-times, etc.

%%%%%%%%%%%%%%%%%%%%%%%%%%%%%%%%%%%%%%%.%
\subsection{Brownian motion}
pdf/cdf of BM at time $T$ (with starting point); with drift/volatility;
pdf/cdf of hitting time BH (w/drift) for level $L$, $H$, $L$ and $H$, $L$ before $H$, etc.;
conditional density of $B_T$ on not hitting barrier(s);
pdf/cdf of minimum/maximum before $T$;
joint pdf of $B_T$, $M_T$, $m_T$ (\& combinations);
occupation times;
average densities (Asians...);
joint densities of $B_{T_1}$, $B_{T_2}$, etc.;

Brownian bridge; lognormal;

copulas?

sampling?; generating Weiner paths?; generic SDE simulator? (Euler/Milstein/etc.)

%%%%%%%%%%%%%%%%%%%%%%%%%%%%%%%%%%%%%%%.%
\subsection{Bachelier model}

%%%%%%%%%%%%%%%%%%%%%%%%%%%%%%%%%%%%%%%.%
\subsection{Black-Scholes-Merton (BS) model}

\[ d_{\pm}(k,t,s,r,g,\sig) = \frac{\ln(s/k) + (r-g\pm\sig^2/2)t}{\sig\sqrt{t}} \]

%%%%%%%%%%%%%%%%%%%%
\subsubsection{Vanilla}
\[ \O{Prem}^\text{V}_\text{BS}(k, t, \om; s, r, g; \sig)
  = \om e^{-gt}s\O{N}(\om d_+) - \om e^{-rt}k\O{N}(\om d_-) \]
\[ \O{\Delta}^\text{V}_\text{BS}(k, t, \om; s, r, g; \sig) = \frac{\partial\O{P}}{\partial s}
  = \om e^{-gt}\O{N}(\om d_+) \]
\[ \O{\Gam}^\text{V}_\text{BS}(k, t, \om; s, r, g; \sig) = \frac{\partial^2\O{P}}{\partial s^2}
  = \O{n}(d_+)\frac{e^{-gt}}{\sig\sqrt{t}} \]
\[ \O{Ve}^\text{V}_\text{BS}(k, t, \om; s, r, g; \sig) = \frac{\partial\O{P}}{\partial \sig}
  = \O{n}(d_+) s e^{-gt}\sqrt{t} \]
\[ \O{Va}^\text{V}_\text{BS}(k, t, \om; s, r, g; \sig) = \frac{\partial^2\O{P}}{\partial s\partial\sig}
  = \O{n}(d_+)\frac{-e^{-gt}d_-}{\sig} \]
\[ \O{Vo}^\text{V}_\text{BS}(k, t, \om; s, r, g; \sig) = \frac{\partial^2\O{P}}{\partial \sig^2}
  = \O{Ve}\frac{d_+ d_-}{\sig} \]
\[ \O{\rho}^\text{V}_\text{BS}(k, t, \om; s, r, g; \sig) = \frac{\partial\O{P}}{\partial r}
  = \om t k e^{-rt} \O{N}(\om d_-) \]
\[ \O{\phi}^\text{V}_\text{BS}(k, t, \om; s, r, g; \sig) = \frac{\partial\O{P}}{\partial g}
  = -\om t s e^{-gt} \O{N}(\om d_+) \]
\[ \O{\theta}^\text{V}_\text{BS}(k, t, \om; s, r, g; \sig) = -\frac{\partial\O{P}}{\partial t}
  = -\frac{s \sig e^{-gt} \O{n}(d_+)}{2\sqrt{t}} + \om g s e^{-gt}\O{N}(d_+) - \om r k e^{-rt}\O{N}(d_-) \]

%%%%%%%%%%%%%%%%%%%%%%%%%%%%%%%%%%%%%%%%
\subsection{Heston stochastic-volatility model}

\[ \O{Prem}^\text{Van}_\text{H}(k, t; s, r, g; \dots) = \]

%%%%%%%%%%%%%%%%%%%%%%%%%%%%%%%%%%%%%%%%
\subsection{Variance-gamma (VG) model}

\[ \O{\Psi}(a, b, \gam) = \frac{1}{\Gam(\gam)}\int_0^\infty \O{N}[ a u^{-1/2} + b u^{1/2} ] u^{\gam-1} e^{-u} \,du \]
\[ \O{\Phi}(\al, \beta, \gam; x, y) = \frac{\Gam(\gam)}{\Gam(\al)\Gam(\gam-\al)} \int_0^1 u^{\al-1} (1-u)^{\gam-\al-1} (1-ux)^{-\beta} e^{uy} \,du \]

\[ \O{Prem}^\text{Van}_\text{VG}(k, t; s, r, g; \dots) = \]

%%%%%%%%%%%%%%%%%%%%%%%%%%%%%%%%%%%%%%%%
\subsection{Constant-elasticity-of-variance (CEV) model}
\[ \O{Prem}^\text{Van}_\text{CEV}(k, t; s, r, g; \dots) = \]


%%%%%%%%%%%%%%%%%%%%%%%%%%%%%%%%%%%%%%%%%%%%%%%%%%%%%%%%%%%%%%%%%%%%%%%%%%%%%%%%
\section{Probability functions}
pdf, cdf, co-cdf, inverse cdf, inverse co-cdf, shifted/scaled versions (give mean/variance {\em or} location/scale parameters), moments (mean, variance, general),

sampling, generation, integration against density, fitting, etc.

normal, lognormal, t, gamma, exponential, logistic, F, $\chi^2$, beta, cauchy, Weibull, Bessel (I),
$\al$-stable, Pareto, Laplace, inverse-gamma, inverse-normal, Bernoulli, Cauchy-Lorentz, ``extreme-value distribution'',
erlang, inverse $\chi^2$, non-centered beta/F/t, Rayleigh, triangular, uniform, ``logarithmic series distribution''


bivariate normal, trivariate normal, n-variate normal,

binomial, poisson, hypergeometric, Kolmogorov-Smirnov, negative binomial, geometric

copulas

%%%%%%%%%%%%%%%%%%%%%%%%%%%%%%%%%%%%%%%%%%%%%%%%%%%%%%%%%%%%%%%%%%%%%%%%%%%%%%%%
\section{Useful techniques}

%%%%%%%%%%%%%%%%%%%%%%%%%%%%%%%%%%%%%%%%
\subsection{Misc}

%%%%%%%%%%%%%%%%%%%%
\subsubsection{Euler-Maclaurin summation}

A few results for possible use:
\begin{eqnarray*}
\int\frac{1}{x^s}\,dx &=& \frac{x^{1-s}}{1-s} \\
\int\frac{1}{(x+a)^s}\,dx &=& \frac{(x+a)^{1-s}}{1-s} \\
\int\frac{z^x}{x^s}\,dx &=& -x^{1-s}\Gam(1-s,-x\ln z)(-x\ln z)^{s-1} \\
\int\frac{z^x}{(x+a)^s}\,dx &=& -(x+a)^{1-s}\Gam(1-s,-(x+a)\ln z)(-(x+a)\ln z)^{s-1}
\end{eqnarray*}

The Euler-Maclaurin summation formula
\[ \sum_{n=a}^b f(n) = \int_a^b f(t)\,dt - \frac{f(b)+f(a)}{2}
    + \sum_{j=2}^k\frac{B_j}{j!}\left( f^{(j-1)}(b) - f^{(j-1)}(a) \right) - \int_a^b \frac{B_k(\{1-t\})}{k!}f^{(k)}(t)\,dt \]
where $\{x\}\in[0,1)$ is the fractional part of $x$ and $B_j$, $B_j(x)$ are Bernoulli numbers, polynomials (resp.).

If $f$ and its derivatives go to zero as $x\to\infty$ then we have
\[ \sum_{n=a}^\infty f(n) = \int_a^\infty f(t)\,dt - \frac{f(a)}{2}
    - \sum_{j=2}^k\frac{B_j}{j!}f^{(j-1)}(a) - \int_a^\infty \frac{B_k(\{1-t\})}{k!}f^{(k)}(t)\,dt \]

This is very useful for getting accurate approximations to the tail of truncated summations...

%%%%%%%%%%%%%%%%%%%%
\subsubsection{Boole summation}
Let $h\in[0,1]$, $f:[m,\infty)\to\RR$ with $k$ continuous derivatives, $f^{(i)}(x)\to0$ as $x\to\infty$ for $i=0,\dots,k$,
then
\[ \sum_{i=m}^\infty (-)^{i-m}f(h+i) = \frac12 \sum_{i=0}^{k-1} \frac{E_i(h)}{i!} f^{(i)}(m) + R_k \]
(recall that the Euler numbers $E_n$ are related to the Euler polynomials $E_n=E_n(1/2)$ and are
$E_{2n+1}=0$, $E_0=1$, $E_2=-1$, $E_4=5$, $E_6=-61$, $E_8=1385$, \dots).


%%%%%%%%%%%%%%%%%%%%
\subsubsection{Borel summation}
(From Wikipedia)

Let $y(z)=\sum_{k=0}^\infty y_k z^k$ be a formal power series in $z$.  Define the Borel transform of $y$
\[ \mathcal{B}_y(t) = \sum_{k=0}^\infty \frac{y_k}{k!} t^k \]
Suppose that $\mathcal{B}_y$ converges to an analytic function near $0$ that can be analytically continued
along the positive real axis to a function growing sufficiently slowly that the following integral is
well-defined (as an improper integral).  Then the {\em Borel sum} of $y$ is given by
\[ \int_0^\infty e^{-t}\mathcal{B}_y(tz)\,dt \]

A slightly weaker form of Borel's summation method gives the Borel sum of $y$ as
\[ \lim_{t\to\infty}e^{-t}\sum_{n}\frac{t^n}{n!}\cdot\sum_{k\leq n} y_k z^k \]
If the sum exists in this sense then it also exists in the previous sense and is the same, but there are some
series that can be summed with the previous method but not with this method.

Example: $y(z)=\sum_{k=0}^\infty z^k = \frac{1}{1-z}$ for $|z|<1$.  Then $\mathcal{B}_y(t) = \sum_{k=0}^\infty\frac{1}{k!}t^k = e^t$ and
the Borel sum is $\int_0^\infty e^{-t}e^{tz}\,dt = \frac{1}{1-z}$ for $\Re z<1$, giving analytic continuation of
the original series to a larger region.

Example: $y(z)=\sum_{k=0}^\infty k!(-z)^k$ diverges for all $z\neq0$.  Then $\mathcal{B}_y(t) = \sum_{k=0}^\infty(-t)^k = \frac{1}{1+t}$ and
the Borel sum is $\int_0^\infty\frac{e^{-t}}{1+tz}\,dt = \frac1z e^{1/z} \Gam(0,1/z)$.

%%%%%%%%%%%%%%%%%%%%
\subsubsection{Kahan's compensated summation trick}
Note that optimization will destroy these algorithms!!

To sum up the numbers $x_1, \dots, x_n$:
\begin{verbatim}
s = 0; e = 0;
for i = 1:n
  temp = s
  y = x[i] + e
  s = temp + y
  e = (temp - s) + y
end
s = s + e // optional correction
\end{verbatim}
Then $s$ is a more accurate summation.
%
%Alternative (equivalent) formulation (from Wikipedia)
%\begin{verbatim}
%sum = 0; c = 0;
%for i = 1 to N
%  y = x[i] - c
%  t = sum + y
%  c = (t - sum) - y
%  sum = t
%end for
%return sum
%\end{verbatim}

%%%%%%%%%%%%%%%%%%%%%%%%%%%%%%%%%%%%%%%%
\subsection{Continued fractions}

Add Wallis algorithm also?

%%%%%%%%%%%%%%%%%%%%
\subsubsection{Pincherle's theorem}
(From Temme: "Numerical Aspects of Special Function")
Pincherle's theorem: given a three-term recurrence relation
\[ y_{n+1} + b(n) y_n + a(n) y_{n-1} = 0 \]
then the continued fraction
\[ \frac{-a_k}{b_k + } \frac{-a_{k+1}}{b_{k+1} + } \cdots \]
converges iff the recurrence relation possesses a minimal solution.  Furthermore,
if $f_n$ is a minimal solution, then the continued fraction converges to $f_k/f_{k-1}$.

%%%%%%%%%%%%%%%%%%%%
\subsubsection{Modified Lentz algorithm}
(cribbed from Numerical Recipes)
Suppose $f(x) = b_0 + \frac{a_1}{b_1 +} \frac{a_2}{b_2 +} \frac{a_3}{b_3 +} \cdots$.
Then compute the following (where $\eps\sim10^{-15}$ and $\zeta\sim10^{-30}$):
\begin{itemize}
\item Set $f_0 = b_0$ (if $b_0=0$ then set $f_0=\zeta$)
\item Set $C_0 = f_0$
\item Set $D_0 = 0$
\item For $j=1, 2, \dots$
  \begin{itemize}
  \item Set $D_j = b_j + a_j D_{j-1}$ (if $D_j = 0$ then set $D_j = \zeta$)
  \item Set $C_j = b_j + a_j / C_{j-1}$ (if $C_j = 0$ then set $C_j = \zeta$)
  \item Set $D_j = 1/D_j$
  \item Set $\Delta_j = C_j D_j$
  \item Set $f_j = f_{j-1} \Delta_j$
  \item if $|\Delta_j -1|<\eps$ then exit.
  \end{itemize}
\end{itemize}

%%%%%%%%%%%%%%%%%%%%
\subsubsection{Quotient-difference algorithm}
Given a series $f(z)=c_0 + c_1 + c_2 z^2$, we can find an equivalent continued fraction via the table
\[\begin{array}{cccccc}
      & q^0_1 \\
e^1_0 &       & e^0_1 \\
      & q^1_1 &       & q^0_2 \\
e^2_0 &       & e^1_1 &       & e^0_2 \\
\vdots& \vdots& \vdots& \vdots& &\ddots \\
\end{array}\]
where $e^n_0 = 0$, $q^n_1 = \frac{c_{n+1}}{c_n}$ and we continue via the rules
\begin{eqnarray*}
e^k_j &=& e^{k+1}_{j-1} + \left(q^{k+1}_j - q^k_j\right) \\
q^k_{j+1} &=& q^{k+1}_j \frac{e^{k+1}_j}{e^k_j}
\end{eqnarray*}
Then we get the continued fraction
\[ c = \frac{a_0}{1 -} \frac{a_1 z}{1 -} \frac{a_2 z}{1 -} \cdots \]
where $a_0 = c_0$, $a_1 = q^0_1$, $a_2 = e^0_1$, $a_3 = q^0_2$, $a_4 = e^0_2$, \dots

%%%%%%%%%%%%%%%%%%%%%%%%%%%%%%%%%%%%%%%%
\subsection{Sequence acceleration}

%%%%%%%%%%%%%%%%%%%%
\subsubsection{Aitken acceleration}
Given $\{x_n\}_n$, produce
\[ \al_n = \frac{x_{n+2}x_n - x_{n+1}^2}{x_{n+2} - 2x_{n+1} + x_n} = x_n - \frac{(\Delta x_n)^2}{\Delta^2 x_n} \]
where $\Delta x_n = x_{n+1} - x_n$ and $\Delta^2 x_n = x_{n+2} - 2x_{n+1} + x_n$.

%%%%%%%%%%%%%%%%%%%%
\subsubsection{Cohen-Villegas-Zagier convergence acceleration of alternating series}
Suppose we have $S=\sum_{k=0}^\infty(-)^k a_k$.  Then fixing $N$, we compute
\begin{verbatim}
Let d = (3 + sqrt(8))^N;
    d = (d + 1/d) / 2;
    b = -1;  c = -d;  s = 0;
    for k = 0 : (N-1)
      c = b - c;
      s = s + c * a_k;
      b = b * ((k+N) * (k-N))/((k+1/2) * (k+1));
output s/d;
\end{verbatim}
Note that this uses universal rational coefficients $c^{(N)}_k/d^{(N)}$ independent of
the particular sequence (so they could be pre-computed...)

This algorithm gives relative accuracy $\sim 5.828^{-N}$, so to get $D$ decimal digits,
we need $N\sim 1.31 D$ (so for 16 digits, need $N\sim21$).  (Assuming $a_k$ are the moments
of a positive measure on $[0,1]$, equivalent to ...)

Can also use the partial sums $s_m=\sum_{k=0}^{m-1}(-)^k a_k$ (rather than the individual terms) and compute
\[ \sum_{k=0}^N c^{(N)}_k a_k = \sum_{m=1}^N \frac{N}{n+m}\binom{N+m}{2m} 2^{2m} s_m \]

%%%%%%%%%%%%%%%%%%%%%%%%%%%%%%%%%%%%%%%%
\subsection{Root finding}

%%%%%%%%%%%%%%%%%%%%
\subsubsection{Newton's method}
\[ x_{n+1} = x_n - \frac{f(x_n)}{f'(x_n)} \]

%%%%%%%%%%%%%%%%%%%%
\subsubsection{Halley's method}
\[ x_{n+1} = x_n - \frac{2f(x_n)f'(x_n)}{2f'(x_n)^2 - f(x_n)f''(x_n)} \]

%%%%%%%%%%%%%%%%%%%%%%%%%%%%%%%%%%%%%%%%
\subsection{Quasi-random sequences}

Sobol' sequence, ...

Faure: use largest prime $p$ greater than the number of dimensions, scramble, ...

Halton sequences:
\begin{itemize}
  \item $1/2,1/4,3/4,1/8,3/8,5/8,7/8,\dots$
  \item $1/3,2/3,1/9,2/9,4/9,5/9,7/9,8/9,\dots$
  \item $1/5,2/5,3/5,4/5,1/25,2/25,\dots$
\end{itemize}
Use a different prime for each dimension.
Let $n=\sum_{k=0}^{l-1}d_k b^k$ be the base-$b$ expansion of $n\geq1$ then generate $\sum_{k=0}^{l-1} d_k b^{-k-1}$.

%%%%%%%%%%%%%%%%%%%%%%%%%%%%%%%%%%%%%%%%%%%%%%%%%%%%%%%%%%%%%%%%%%%%%%%%%%%%%%%%
\section{Notes on implementations}
tolerance, accuracy, error estimate (max, expected), iterations, time, cost (mul,div,plus,sub,exp,... ops)
\begin{itemize}
\item Taylor series
  \begin{itemize}
  \item at fixed points
  \item at arbitrary points
  \item analytic continuation
  \item (modified series)
  \end{itemize}
\item (For inverse functions)
  \begin{itemize}
  \item root-finders
  \item direct approximation / inverse interpolation
  \item power series (Lagrange inversion, Dominici techniques)
  \end{itemize}
\item quadrature
  \begin{itemize}
  \item real
  \item complex (contour)
  \item ODE (system)
  \end{itemize}
\item local approximation
  \begin{itemize}
  \item polynomial (power-series, Chebyshev, other orthogonal polynomial)
  \item Pad\'e (rational)
  \item other series
  \end{itemize}
\item continued fraction
  \begin{itemize}
  \item fixed depth, direct
  \item iteration (P,Q)
  \end{itemize}
\item asymptotic expansion
\item recurrence (backward, forward, scaling trick)
\item expansion in other special functions
\end{itemize}

%%%%%%%%%%%%%%%%%%%%%%%%%%%%%%%%%%%%%%%%%%%%%%%%%%%%%%%%%%%%%%%%%%%%%%%%%%%%%%%%
\section{Notes on testing}
\begin{enumerate}
\item tabulated values (Mathematica, Maple, etc.) [$\erf 0=0$]
\item comparison for special cases (relations between functions) [$\erf z=\gamma(z^2,1/2)$]
\item identities [$\cos^2z+\sin^2z=1$]
\item comparison among alternative implementations [power series vs. Pad\'e approximation vs. continued fraction vs. integration]
\item properties of function (e.g. check monotonicity --- strictest test is 1ulp bump...; inequalities, etc.)
\item for functions with many parameters, use a lattice (say Sobol' qrng) to generate points that uniformly sample the
  space for testing specific values
\end{enumerate}

%%%%%%%%%%%%%%%%%%%%%%%%%%%%%%%%%%%%%%%%%%%%%%%%%%%%%%%%%%%%%%%%%%%%%%%%%%%%%%%%
\section{General mathematical library}
\begin{enumerate}
\item Data-types
  \begin{enumerate}
  \item double
  \item double-double, quad-double
  \item arbitrary precision
  \item complex<>
  \item vector<>, matrix<>, matrix\_ndim<>
  \item Common Lisp has: Number=Complex/Real; Real=Float/Rational; Float=Short/Single/Double/Long; Rational=Ratio/Integer;
    Integer=Fixnum/Bignum
  \item Scheme (R6RS) has: Complex - Real - Rational - Integer and distinguishes Exact/Inexact and Flonum/Fixnum
  \end{enumerate}
\item Monte-carlo methods
  \begin{enumerate}
  \item Random-number generation (Mersenne Twister prng, Sobol' qrng)
  \item Distribution sampling (all major types as well as generic; discrete and continuous)
  \end{enumerate}
\item Root-finding
  \begin{enumerate}
  \item $1$-dimension: bracketer, bisection, secant, Newton, inverse-quadratic, Halley, Brent-Dekker, (monotonic)
  \item $2$-,$n$-dimension: conjugate-gradient, homotopy
  \item polynomial methods --- specialized methods to find {\em all} real/complex roots, etc.
  \end{enumerate}
\item Sequence acceleration (Aitken, Wynn, $\Delta$, etc.)
\item Differentiation (basic, $n$-th order, Savitzy-Golay, adaptive; weight generators)
\item Optimization
  \begin{enumerate}
  \item $1$-dim minimization / maximization
  \item $n$-dim minimization / maximization
  \item least-squares optimization (linear / non-linear)
  \end{enumerate}
\item Integration
  \begin{enumerate}
  \item Quadrature ($1$-dim)
    \begin{enumerate}
    \item Open/closed/infinite intervals, singularities
    \item Trapezoidal, Simpson, Gauss-type rules (\& weight generators)
    \item Contour integration in $\CC$ (branch tracking?)
    \end{enumerate}
  \item Cubature ($2$-dim)
  \item High-dimensional integration (lattice, MC, quasi-MC; dimension reduction)
  \item ODE
    \begin{enumerate}
    \item $1$-dim second-order equation
    \item $n$-dim first-order (RK (4 \& others), adaptive, Boer-Stoerlisch, etc.)
    \end{enumerate}
  \item PDE (generic IC, BC, coeffs)
    \begin{enumerate}
    \item $1$-dim parabolic
    \item $2$-dim parabolic
    \item $2$-,$3$-dim elliptic
    \end{enumerate}
  \end{enumerate}
  \item Solving integral equations ...
\item Interpolation
  \begin{enumerate}
  \item $1$-dim (splines (cubic, B-, Akima, general), Chebyshev, RBF, etc.)
  \item $2$-dim (bicubic, scattered data, etc.)
  \item $n$-dim (scattered, etc.)
  \item probability distribution
  \end{enumerate}
\item Linear algebra
  \begin{enumerate}
  \item ...
  \end{enumerate}
\end{enumerate}

\begin{itemize}
\item function grouping / classification
\item name, symbol
\item parameters, arguments: largest domain of definition, branch cuts, etc.
\item different representations (and domains of validity)
\item relations
\item implementation notes
  --- for different parameter/argument restrictions (e.g. $\NN$ vs $\RR$ vs $\CC$)
  --- for scaled versions, etc.
\item NR/ISML/Boost/GSL/C library/NAG/Pari/cephes/\&c. availability
\end{itemize}

Generic functions templated with input-type, working-type, output-type
(which can be allowed to be real or complex variants).  Can be useful for balancing
needed accuracy vs. speed (especially inside other routines...)

Precision: [also 20/25/30 digits for double-double, etc.; arbitrary precision? (template implementation...)]
\begin{itemize}
\item Green: $\ge15$ digits
\item Yellow: $\ge10$ digits
\item Red: $\geq5$ digits
\item Grey: $<5$ digits and/or restricted domain
\item Black: not supported
\end{itemize}
Timing: 
\begin{itemize}
\item $\leq10$ms
\item $\leq100$ms
\item $\leq1$s
\item $>1$s
\end{itemize}
Domain: 
\begin{itemize}
\item integer
\item real interval
\item reals
\item complex domain
\item complex
\item principal value + branches
\end{itemize}
Platform: 
\begin{itemize}
\item Octave
\item C
\item C++
\end{itemize}
Modify with $+/-$ for some discrepancies (lower in extreme ranges, etc.)

Also, intrinsic liminations (over/under-flow) [solve with scaled versions also]
vs. extrinsic (unstable algorithm, etc.)

Idea: number representation as Integer $\pm$ Fractional-part, where both of the two components
have an independent exponent/mantissa representation.  (Similar to the quad/double-double representations
in two parts...)  This allows to represent $1+\eps$ with high precision in $\eps$...

Estimation of cancellation errors in summations --- compare magnitude of result to largest term / partial-sum.

Test data-files: \# comment or nnn|nnn|nnn (to many digits of accuracy, minimum 72 (for quad-double+guards)).

Plot test points --- color-coded for error/etc. (for 2 parameter cases).

%%%%%%%%%%%%%%%%%%%%%%%%%%%%%%%%%%%%%%%%%%%%%%%%%%%%%%%%%%%%%%%%%%%%%%%%%%%%%%%%
%%%%%%%%%%%%%%%%%%%%%%%%%%%%%%%%%%%%%%%%%%%%%%%%%%%%%%%%%%%%%%%%%%%%%%%%%%%%%%%%
%\bibliography{special_functions}{}
%\bibliographystyle{plain}

\begin{thebibliography}{WWWWW}
\bibitem[A\&S]{a&s}
  Abramowitz and Stegun,
  \emph{Handbook of ...}.
\bibitem[Concrete]{ConcreteMath}
R.L. Graham, D.E. Knuth, and O. Patashnik, \emph{Concrete
mathematics}, Addison-Wesley, Reading, MA, 1989.
\bibitem[Luke1]{luke1}
  Yudell Luke,
  \emph{The Special Functions and Their Approximations, Volume 1}.
\bibitem[Atlas]{atlas:oldham}
  K. Oldham, J. Myland, and J. Spanier,
  \emph{An Atlas of Special Functions: With Equator, the Atlas Function Calculator}, second edition
  Springer, 2009.
\bibitem[NIST]{nist}
  eds F.W.J. Olver, D.W. Lozier, R.F. Boisvert, and C.W. Clark,
  \emph{NIST Handbook of Mathematical Functions},
  NIST and Cambridge University Press, 2010.
\bibitem[NR]{nr}
  W.H. Press, S.A. Teukolsky, W.T. Vetterling, and B.P. Flannery,
  \emph{Numerical Recipes: The art of scientific computing}, third edition,
  Cambridge University Press, 2007.
\bibitem[ACMF]{atlas:thompson}
  W.J. Thompson,
  \emph{Atlas for Computing Mathematical Functions: An Illustrated Guide for Practitioners; With Programs in Fortran 90 and Mathematica},
  John Wiley \& Sons, Inc., 1997.
\bibitem[ZJ]{zj}
  S. Zhang and J. Jin,
  \emph{Computation of Special Functions},
  John Wiley \& Sons, Inc., 1996.
\bibitem[DLMF]{DLMF}
  \emph{DLMF}
\bibitem[Lebedev]{lebedev}
  Lebedev
  \emph{Special Functions...}
\bibitem[Marsaglia]{marsaglia}
  G. Marsaglia,
  ``Evaluating the Normal Distribution''
  \emph{Journal of Statistical Software},
  July 2004, Volume 11, Issue 4.
\bibitem[Precise]{precise-numerical}
  M. Sofroniou and G. Spaletta,
  ``Precise Numerical Computation'',
  in \emph{...}.
\bibitem[AppFD]{approximation-fermi-dirac}
  N.M. Temme and A.B. Olde Daalhuis,
  ``Approximation of Fermi-Dirac integrals''
  in \emph{Journal of Computational and Applied Mathematics},
  {\bf 31} (1990) 383--387, North-Holland.
\bibitem[ZetaE]{edwards-zeta}
  Edwards ...
  \emph{Riemann Zeta Function}.
\end{thebibliography}


%%%%%%%%%%%%%%%%%%%%%%%%%%%%%%%%%%%%%%%%%%%%%%%%%%%%%%%%%%%%%%%%%%%%%%%%%%%%%%%%
%%%%%%%%%%%%%%%%%%%%%%%%%%%%%%%%%%%%%%%%%%%%%%%%%%%%%%%%%%%%%%%%%%%%%%%%%%%%%%%%
%%%%%%%%%%%%%%%%%%%%%%%%%%%%%%%%%%%%%%%%%%%%%%%%%%%%%%%%%%%%%%%%%%%%%%%%%%%%%%%%
%%%%%%%%%%%%%%%%%%%%%%%%%%%%%%%%%%%%%%%%%%%%%%%%%%%%%%%%%%%%%%%%%%%%%%%%%%%%%%%%
%%%%%%%%%%%%%%%%%%%%%%%%%%%%%%%%%%%%%%%%%%%%%%%%%%%%%%%%%%%%%%%%%%%%%%%%%%%%%%%%
\end{document}
%%%%%%%%%%%%%%%%%%%%%%%%%%%%%%%%%%%%%%%%%%%%%%%%%%%%%%%%%%%%%%%%%%%%%%%%%%%%%%%%
%%%%%%%%%%%%%%%%%%%%%%%%%%%%%%%%%%%%%%%%%%%%%%%%%%%%%%%%%%%%%%%%%%%%%%%%%%%%%%%%
%%%%%%%%%%%%%%%%%%%%%%%%%%%%%%%%%%%%%%%%%%%%%%%%%%%%%%%%%%%%%%%%%%%%%%%%%%%%%%%%
%%%%%%%%%%%%%%%%%%%%%%%%%%%%%%%%%%%%%%%%%%%%%%%%%%%%%%%%%%%%%%%%%%%%%%%%%%%%%%%%
%%%%%%%%%%%%%%%%%%%%%%%%%%%%%%%%%%%%%%%%%%%%%%%%%%%%%%%%%%%%%%%%%%%%%%%%%%%%%%%%
